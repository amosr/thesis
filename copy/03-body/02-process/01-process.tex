\chapter{Processes and networks}
\label{chapter:process:processes}

This chapter introduces a language for describing processes in a Kahn process network.
These process networks execute concurrently and communicate via fixed-size bounded buffers between channels.
Each buffer is restricted to a single element.
The processes in a process network are then fused together to form a single process which computes the whole process, without the need for communication.


\section{Gold panning with processes}
In \autoref{taxonomy} we introduced the @priceAnalyses@ example, which performs statistical analyses over the daily prices of a particular corporate stock and market index.
\autoref{figs/procs/priceAnalyses-again} shows the dependency graph for @priceAnalyses@ with the two input streams, @index@ and @stock@, at the top of the graph.

\begin{figure}
\center
\begin{dot2tex}[dot]
digraph G {
  node [shape="none"];
  stock; index;

  stock -> pom_join;
  index -> pom_join;
  stock -> pot_tps;

  graph [style="rounded corners"];

  subgraph cluster_priceOverTime  {
    lblstyle="right";
    label="priceOverTime";
    pot_tps [label="map"];
    pot_cor [label="correlation"];
    pot_reg [label="regression"];
    pot_tps -> pot_cor;
    pot_tps -> pot_reg;
  };

  subgraph cluster_priceAgainstMarket {
    lblstyle="left";
    label="priceOverMarket";
    pom_join [label="join"];
    pom_price [label="map"];
    pom_cor [label="correlation"];
    pom_reg [label="regression"];
    pom_join -> pom_price;
    pom_price -> pom_cor;
    pom_price -> pom_reg;
  };
}
\end{dot2tex}
\caption{Dependency graph for priceAnalyses example.}
\label{figs/procs/priceAnalyses-again}
\end{figure}

Recall that we cannot execute this example using the pull streams from \autoref{taxonomy/pull}, because the @stock@ input stream is used twice, and pull streams only support a single consumer.
Similarly, we cannot execute this example using the push streams from \autoref{taxonomy/push}, because the @join@ combinator has two inputs, and push streams only support a single producer except for non-deterministic merge.
Rather than just using pull streams, or just using push streams, we wish to be able to perform both pulling and pushing at the same time, in a way that supports multiple consumers and multiple producers.
Kahn process networks \citep{kahn1976coroutines} are a flexible, expressive way of writing streaming computations, where a network is composed of communicating processes.
Executing communicating processes introduces runtime overhead, as stream elements must be passed between processes.
Instead, we wish to take this concurrent process network and convert it back to sequential code, without any runtime scheduling or message passing overhead.

% This additional communication means that a stream element which may have been in cache, has likely been swapped out by the time the consumer receives it.
% Or if the consumer is running on a different processor, it is unlikely to be in the lower level of cache in the first place.
% Furthermore, communication primitives likely require some kind of locking, which will add even more overhead.
%
% A lot of the time, the ideal execution model is actually just a simple imperative loop.
% By using concurrent processes we have gained expressivity, but at the cost of speed.

A \emph{process} in our system is a simple imperative program with a local heap.
A process pulls source values from an arbitrary number of input streams and pushes result values to at least one output stream.
The process language is an intermediate representation we use when fusing the overall dataflow network.
When describing the fusion transform we describe the control flow of the process as a state machine.

A \emph{combinator} is a template for a process which parameterises it over the particular input and output streams, as well as values of configuration parameters such as the worker function used in a @map@ process.
Each process implements a logical \emph{operator} --- so we use ``operator'' when describing the values being computed, but ``process'' when referring to the implementation.


\subsection{Fold combinator}
\begin{figure}
\begin{process}
foldl 
  = \ (k  : b -> a -> b) (z   : b) (j : b -> c)
      (sIn: Stream a)    (sOut: Stream b). 
    / (s  : b) (v : a)   (F0..F4: Label).
    process
     { ins:    { sIn  }
     , outs:   { sOut }
     , heap:   { s = z, v }
     , label:    F0
     , instrs: { F0 = pull  sIn     v  F1[] else F2[]
               , F1 = drop  sIn        F0[s = k s v]

               -- sIn closed
               , F2 = push  sOut (j s) F3[]
               , F3 = close sOut       F4[]
               , F4 = exit } }
\end{process}
\caption{Process implementation of foldl}
\label{figs/procs/impl/foldl}
\end{figure}

The definition of the @foldl@ combinator, used to implement @correlation@ and @regression@ in our @priceAnalyses@ process network, is given in \autoref{figs/procs/impl/foldl}.
The combinator is parameterised by the fold state update function (@k@) and the fold state initialisation (@z@).
In @correlation@ and @regression@, the result must be extracted from the fold state; we extend the standard presentation of @foldl@ with an eject function (@j@) to perform this extraction.
The process reads from an input stream and, at the end of the input stream, produces a single-element output stream containing the fold result.
The \emph{nu-binders} ($\nu@ (s : a) (v : b)@\ldots$) indicate that each time the @foldl@ combinator is instantiated, fresh names must be given to @s@, @v@ and so on, that do not conflict with other insantiations.
The @s@ and @v@ bindings refer to variables in the mutable heap of the process.
The @s@ variable stores the current fold state and is initialised to the initial fold value (@z@); the @v@ variable stores the most recent value from the input stream, and is left uninitialised.

The body of the combinator is a record that defines the process.
The @ins@ field defines the set of input streams, and the @outs@ field defines the set of output streams.
The @heap@ field gives the initial values of each of the local variables; variables without an explicit initial value are given some arbitrary value.
The @instrs@ field contains a set of labelled instructions that define the program, while the @label@ field gives the label of the initial instruction.
In this form, the output stream (@sOut@) is defined via a parameter, rather than being the result of the combinator.

The initial instruction (\lstiproc!pull sIn v F1[] else F2[]!) pulls the next element from the stream @sIn@, writes it into the heap variable @v@, then proceeds to the instruction at label @F1@.
The empty list @[]@ after the target label @F1@ can be used to update heap variables, but as we do not need to update anything yet we leave it empty. 
If the input stream is finished, there are no more elements to pull; execution proceeds to the instruction at label @F2@ instead.

After successfully pulling a new element from the input stream, the instruction at label @F1@ (\lstiproc!drop sIn F0[s = k s v]!) signals that the current element that was pulled from stream @sIn@ is no longer required, before updating the fold state (@s@) by applying the fold update function (@k@).
Execution then proceeds back to the pull instruction at label @F0@.
In \autoref{s:Drop:in:synchrony} we shall see how this @drop@ instruction is used to synchronise processes reading from the same shared stream, ensuring that all processes operate on the same element together without overtaking one another.

When the input stream is finished, the instruction (\lstiproc!push sOut (j s) F3[]!) pushes the eject function applied to the final fold state to the output stream @sOut@.
Execution then proceeds to the instruction at label @F3@.
The comment above the instruction highlights the change in state of the input stream.

Next, the instruction (\lstiproc!close sOut F4[]!) signals that the output stream @sOut@ is finished, and then proceeds to the instruction at label @F4@.

Finally, the instruction (\lstiproc!exit!) signals that the process is finished, and has no further work to do.
The process terminates.

\subsection{Map combinator}

\begin{figure}
\begin{process}
map 
  = \ (f  : a -> b)
      (sIn: Stream a) (sOut: Stream b). 
    / (v  : a)        (M0..M4: Label).
    process
     { ins:    { sIn  }
     , outs:   { sOut }
     , heap:   { v }
     , label:  M0
     , instrs: { M0 = pull  sIn     v  M1[] else M3[]
               , M1 = push  sOut (f v) M2[]
               , M2 = drop  sIn        M0[]

               -- sIn closed
               , M3 = close sOut       M4[]
               , M4 = exit } }
\end{process}
\caption{Process implementation of map}
\label{figs/procs/impl/map}
\end{figure}

The definition of the @map@ combinator, which applies a worker function to every element in the input stream, is given in \autoref{figs/procs/impl/map}.
The combinator is parameterised by the worker function (@f@), and takes one input stream (@sIn@) and produces one output stream (@sOut@).
The heap variable (@v@) is used to store the last value read from the input stream.
The process starts by pulling from the input stream, storing the element in the heap variable (@v@).
It then pushes the transformed element (@f v@) into the output stream, drops the element from the input stream, and pulls again.
When the input stream finishes, the process closes the output stream and terminates.

\subsection{A network of processes}
The @map@ and @foldl@ combinators are sufficient to express the @priceOverTime@ example, which takes a single input stream and computes the correlation and regression.
Here is the list implementation of @priceOverTime@ again:

\begin{haskell}
priceOverTime :: [Record] -> (Line, Double)
priceOverTime stock =
  let timeprices = map (\r -> (daysSinceEpoch (time r), price r)) stock
  in (regression timeprices, correlation timeprices)
\end{haskell}

We can express @priceOverTime@ as a process network by instantiating the above process templates and connecting them together.
A process network is a set of processes that are able to communicate with each other.

\begin{process}
priceOverTime =
  \ (stock : Stream Record)
    (reg_out : Stream Line) (cor_out : Stream Double).
  / (timeprices : Stream (Double,Double)).
     { map    tp_f             stock      timeprices
     , foldl reg_k reg_z reg_j timeprices reg_out
     , foldl cor_k cor_z cor_j timeprices cor_out }
\end{process}

As with the process templates, the network is parameterised by the output streams, which are in this case the output of @regression@ and @correlation@.
We use the nu-binder syntax to instantiate a fresh name for the @timeprices@ internal stream, which is the output of the @map@ combinator.
We implement @regression@ and @correlation@ as folds with eject functions.
The details of the worker functions given to @map@ and @foldl@ are irrelevant for fusion, and are defined externally.

\subsection{Fusing processes together}
\label{s:FusingProcesses}

Our fusion algorithm takes two processes and produces a new one that computes the output of both.
We fuse a pair of processes in the @priceOverTime@ network; to distinguish between the two @foldl@ processes in this network, we refer to them as the @regression@ and @correlation@ processes.
As an example, we fuse the @map@ process with the @regression@ process.
The result process computes the result of both processes as if they were executed concurrently, where the output stream of the @map@ process is used as the input stream of the @regression@ process.

\begin{figure}
\begin{process}
process -- map tp\_f stock timeprices
 { ins:    { stock  }
 , outs:   { timeprices }
 , heap:   { tp_v }
 , label:  M0
 , instrs: { M0 = pull  stock       tp_v        M1[] else M3[]
           , M1 = push  timeprices (tp_f tp_v)  M2[]
           , M2 = drop  stock                   M0[]

           -- stock closed
           , M3 = close timeprices              M4[]
           , M4 = exit } }
\end{process}
\vspace{1em}
\begin{dot2tex}[dot,scale=0.8]
digraph G {
node[shape=none,texmode="raw"];
  I[shape=point];
edge[style="procFstD,thick"];
  M0 [label="\CbF{pull stock tp\_v} (M0)"];
  M1 [label="\CbF{push timeprices (tp\_f tp\_v)} (M1)"];
  M2 [label="\CbF{drop stock} (M2)"];
  M3 [label="\CbF{close timeprices} (M3)"];
  M4 [label="\CbF{exit} (M4)"];

  I -> M0;

  M0 -> M1 [label="have stock "];
  M1 -> M2;
  M2 -> M0;

  M0 -> M3 [label="closed stock "];
  M3 -> M4;
}
\end{dot2tex}
\caption{Instantiated process for map with control flow graph}
\label{figs/procs/instance/pot-timeprices}
\end{figure}

\begin{figure}
\begin{process}
process -- foldl reg\_k reg\_z reg\_j timeprices reg\_out
 { ins:    { timeprices  }
 , outs:   { reg_out }
 , heap:   { reg_s = reg_z, reg_v }
 , label:    F0
 , instrs: { F0 = pull  timeprices reg_v  F1[] else F2[]
           , F1 = drop  timeprices        F0[reg_s = reg_k reg_s reg_v]

           -- timeprices closed
           , F2 = push  reg_out (reg_j reg_s) F3[]
           , F3 = close reg_out       F4[]
           , F4 = exit } }
\end{process}
\vspace{1em}
\begin{dot2tex}[dot,scale=0.8]
digraph G {
node[shape=none];
  node[texmode="raw"];
edge[style="procSndD,thick"];
  I[shape=point];
  F0 [label="\CbS{pull timeprices reg\_v} (F0)"];
  F1 [label="\CbS{drop timeprices [reg\_s = reg\_k reg\_s reg\_v]} (F1)"];
  F2 [label="\CbS{push reg\_out (reg\_j reg\_s)} (F2)"];
  F3 [label="\CbS{close reg\_out} (F3)"];
  F4 [label="\CbS{exit} (F4)"];

  I -> F0;

  F0 -> F1 [label="have timeprices"];
  F1 -> F0;

  F0 -> F2 [label="closed timeprices"];
  F2 -> F3;
  F3 -> F4;
}
\end{dot2tex}
\caption{Instantiated process for fold (regression) with control flow graph}
\label{figs/procs/instance/pot-regression}
\end{figure}

\autoref{figs/procs/instance/pot-timeprices} shows the result of instantiating the @map@ process in the @priceOverTime@ process network.
The combinator parameters have the corresponding argument value substituted in, and the variables and labels are given fresh names as necessary.
We rename the variable name @v@ to @tp_v@, to avoid conflict with variables named @v@ in other processes.
The figure also shows the control flow graph of the process.
\autoref{figs/procs/instance/pot-regression} likewise shows the result of instantiating the @regression@ process.
The instructions and edges in each control flow graph are coloured differently; the same colours will be used to highlight the provenance of each instruction in our informal description of the fusion algorithm.



\subsubsection{Fusing Pulls}
\label{s:Fusion:FusingPulls}

The algorithm proceeds by considering pairs of labels and instructions: one from each of the source processes to be fused.
First, we consider the initial labels of each process and their corresponding instructions.
This situation is shown in \autoref{figs/fsm/fuse-pulls}; instructions from the two source processes are shown side-by-side and the instruction of the fused process is below.
The @map@ process pulls from the @stock@ stream, while the @regression@ process pulls from the @timeprices@ stream.
As the @timeprices@ stream is produced by the @map@ process, the @regression@ process must wait until the @map@ process pushes a value.
If we were to execute the two processes concurrently at this stage, only the @map@ process could make progress, by pulling from the @stock@ input stream.
The corresponding instruction for the fused process pulls from the @stock@ input stream, allowing the @map@ process to execute while the @regression@ process waits.


\begin{figure}
\center
\begin{tabular}{ll||rr}
\begin{dot2tex}[dot]
digraph G {
node[shape=none,texmode="raw"];
edge[style="procFstD,thick"];
  M0 [label="\CbF{pull stock tp\_v} (M0)"];
  M1 [label="... (M1)"];
  M3 [label="... (M3)"];
  M0 -> M1 [label="have stock"];
  M0 -> M3 [label="closed stock"];
}
\end{dot2tex}
& \quad & \quad &
\begin{dot2tex}[dot]
digraph G {
node[shape=none,texmode="raw"];
edge[style="procSndD,thick"];
  F0 [label="\CbS{pull timeprices reg\_v} (F0)"];
  F1 [label="... (F1)"];
  F2 [label="... (F2)"];
  F0 -> F1 [label="have timeprices"];
  F0 -> F2 [label="closed timeprices"];
}
\end{dot2tex}
\end{tabular}
\vspace{1em}
\center
\begin{dot2tex}[dot]
digraph G {
node[shape=none,texmode="raw"];
  M0F0 [label="\CbF{pull stock tp\_v} \FuTiReLa{M0}{F0}{none}"];
  M1F0 [label="... \FuTiReLa{M1}{F0}{none}"];
  M3F0 [label="... \FuTiReLa{M3}{F0}{none}"];
  M0F0 -> M1F0 [label="have stock",style="procFstD,thick"];
  M0F0 -> M3F0 [label="closed stock",style="procFstD,thick"];
}
\end{dot2tex}
\caption{Fusing pull instructions; the left process can pull from an unshared stream, while the right process must wait for the first to produce a value.}
\label{figs/fsm/fuse-pulls}
\end{figure}

Each of the joint result labels represents a combination of two source labels, one from each of the source machines.
For example, the first joint label \FuTiReLa{M0}{F0}{none} represents a combination of the @map@ process being in its initial state @M0@ and the @regression@ process being in its own initial state @F0@. 
We also associate each of the joint labels with the \emph{input state}: a description of whether the @regression@ process has a value available to read from the shared @timeprices@ stream.
There is no value available, so the input state for @timeprices@ is set to @none@.
This extra information only applies to shared input streams; as such, the input state of the @map@ process is the empty map.

\subsubsection{Fusing Push with Pull}
\label{s:Fusion:FusingPushPull}

Next, \autoref{figs/fsm/fuse-pushpull} shows the @map@ process pushing into the @timeprices@ stream after pulling a value from the @stock@ stream, while the @regression@ process is still trying to pull from the @timeprices@ stream.
After the @map@ process pushes a value, this value becomes available for the @regression@ process.
In the fused process, this situation results in two steps.
First, the @map@ process pushes the value (@tp_f tp_v@), and stores this value in the new local variable (@chan_tp@), so it is available for the @regression@ process.
The input state for the @regression@ process is updated to (@pending@), to signal that there is a value ready to be pulled in the (@chan_tp@) variable.
Next, the @regression@ process reads the @pending@ value, copying from the (@chan_tp@) variable into the (@reg_v@) variable.
The input state for the @regression@ process is updated to (@have@), to signal that the @regression@ process has copied the pulled value and is using it.

In the original process network, before any fusion, the @timeprices@ stream has two consumers: the @regression@ and @correlation@ processes.
Since the fused process implements both @map@ and @regression@ processes, the fused process still pushes to the @timeprices@ stream to allow the @correlation@ process to consume it.

\begin{figure}
\center
\begin{tabular}{ll||rr}
\begin{dot2tex}[dot]
digraph G {
node[shape=none,texmode="raw"];
edge[style="procFstD,thick"];
  M1 [label="\CbF{push timeprices (tp\_f tp\_v)} (M1)"];
  M2 [label="... (M2)"];
  M1 -> M2;
}
\end{dot2tex}
& \quad & \quad &
\begin{dot2tex}[dot]
digraph G {
node[shape=none,texmode="raw"];
edge[style="procSndD,thick"];
  F0 [label="\CbS{pull timeprices reg\_v} (F0)"];
  F1 [label="... (F1)"];
  F2 [label="... (F2)"];
  F0 -> F1[label="have timeprices"];
  F0 -> F2[label="closed timeprices"];
}
\end{dot2tex}
\end{tabular}
\vspace{1em}
\center
\begin{dot2tex}[dot]
digraph G {
node[shape=none,texmode="raw"];
  M1F0 [label="\CbF{push timeprices (tp\_f tp\_v)[chan\_tp=tp\_f tp\_v]} \FuTiReLa{M1}{F0}{none}"];
  M2F0 [label="\CbS{jump [reg\_v=chan\_tp]} \FuTiReLa{M2}{F0}{pending}"];
  M2F1 [label="... \FuTiReLa{M2}{F1}{have}"];
  M1F0 -> M2F0 [style="procFstD,thick"];
  M2F0 -> M2F1 [style="procSndD,thick"];
}
\end{dot2tex}
\caption{Fusing push with pull; the left process produces a value, which the right process consumes.}
\label{figs/fsm/fuse-pushpull}
\end{figure}


\subsubsection{Fusion result}
\autoref{figs/procs/instance/fused-timeprices-regression} shows the final result of fusing the @map@ and @regression@ processes together.
There are similar rules for handling the other combinations of instructions, but we defer the details to \autoref{s:Fusion}.
The result process has one input stream, @stock@, and two output streams: @timeprices@ from @map@, and @reg_out@ from @regression@.

To complete the implementation of @priceOverTime@, we would now fuse this result process with the @correlation@ process.
Note that although the result process has a single shared heap, the heap bindings from each fused process are guaranteed not to interfere, as when we instantiate combinators to create source processes we introduce fresh names. 



\begin{figure}
\begin{lstlisting}[language=process,linebackgroundcolor={
  \hilineFst{2}
  \hilineFst{3}
  \hilineSnd{4}
  \hilineFst{5}
  \hilineSnd{6}
  \hilineCom{7}
  \hilineFst{10}
  \hilineFst{11}
  \hilineFst{12}
  \hilineFst{13}
  \hilineSnd{14}
  \hilineSnd{15}
  \hilineFst{16}
  \hilineFst{17}
  \hilineSnd{18}
  \hilineSnd{19}
  \hilineFst{22}
  \hilineFst{23}
  \hilineSnd{24}
  \hilineSnd{25}
  \hilineSnd{26}
  \hilineSnd{27}
  \hilineSnd{28}
  \hilineSnd{29}
  \hilineCom{30}
  \hilineCom{31}
  }]
process -- \colorbox{procFst}{map tp\_f stock timeprices} / \colorbox{procSnd}{foldl reg\_k reg\_z reg\_j timeprices reg\_out}
 { ins:    { stock  }
 , outs:   { timeprices
           , reg_out }
 , heap:   { tp_v
           , reg_s = reg_z, reg_v
           , chan_tp }
 , label:    M0_F0
 , instrs:
 { M0_F0   = pull stock tp_v M1_F0[] else M3_F0[]
 -- \FuTiReLa{M0}{F0}{none}; (LocalPull)
 , M1_F0   = push timeprices (tp_f tp_v) M2_F0_p[chan_tp = (tp_f tp_v)]
 -- \FuTiReLa{M1}{F0}{none}; (SharedPush)
 , M2_F0_p = jump M2_F1_h[reg_v = chan_tp]
 -- \FuTiReLa{M2}{F0}{pending}; (SharedPullPending)
 , M2_F1_h = drop stock M0_F1_h[] 
 -- \FuTiReLa{M2}{F1}{have}; (LocalDrop)
 , M0_F1_h = jump M0_F0[reg_s = reg_k reg_s reg_v]
 -- \FuTiReLa{M0}{F1}{have}; (ConnectedDrop)

 -- stock closed
 , M3_F0   = close timeprices M4_F0_c[] 
 -- \FuTiReLa{M3}{F0}{none}; (SharedClose)
 , M4_F0_c = jump M4_F2_c[] 
 -- \FuTiReLa{M4}{F0}{closed}; (SharedPullClosed)
 , M4_F2_c = push reg_out (reg_j reg_s) M4_F3_c 
 -- \FuTiReLa{M4}{F2}{closed}; (LocalPush)
 , M4_F3_c = close reg_out M4_F4_c[] 
 -- \FuTiReLa{M4}{F3}{closed}; (LocalClose)
 , M4_F4_c = exit
 -- \FuTiReLa{M4}{F4}{closed}; (LocalExit)
 } }
\end{lstlisting}
\caption{Fusion of \colorbox{procFst}{timeprices} and \colorbox{procSnd}{regression}, along with \colorbox{procCommon}{shared} instructions and variables. }
\label{figs/procs/instance/fused-timeprices-regression}
\end{figure}

\subsection{Join combinator}

\begin{figure}
\begin{process}
join 
  = \ (cmp : a -> b -> Ordering)
      (sA  : Stream a) (sB : Stream b)
      (sOut: Stream (a,b)). 
    / (va : a) (vb : b) (c : Ordering) (...: Label).
    process
     { ins:    { sA, sB }
     , outs:   { sOut }
     , heap:   { va, vb, c }
     , label:  IN0
     , instrs: { IN0 = pull  sA va      IN1[] else DB0[]
               , IN1 = pull  sB vb      IN2[] else DA1[]
               , IN2 = jump             IN3[ c = cmp va vb ]
               , IN3 = case  (c == EQ)  EQ0[] else NE0[]

               -- cmp va vb $=$ EQ
               , EQ0 = push  sOut (a,b) EQ1[]
               , EQ1 = drop  sA         EQ2[]
               , EQ2 = drop  sB         IN0[]

               -- cmp va vb $\not=$ EQ
               , NE0 = case  (c == LT)  LT0[] else GT0[]

               -- cmp va vb $=$ LT
               , LT0 = drop  sA         LT1[]
               , LT1 = pull  sA va      IN2[] else DB1[]

               -- cmp va vb $=$ GT
               , GT0 = drop  sB         GT1[]
               , GT1 = pull  sB vb      IN2[] else DA1[]

               -- sB closed; drain sA
               , DA0 = pull  sA         DA1[] else EX0[]
               , DA1 = drop  sA         DA0[]

               -- sA closed; drain sB
               , DB0 = pull  sB         DB1[] else EX0[]
               , DB1 = drop  sB         DB0[]

               -- sA and sB closed
               , EX0 = close sOut       EX1[]
               , EX1 = exit } }
\end{process}
\caption{Process implementation of join}
\label{figs/procs/impl/join}
\end{figure}

To implement the whole @priceAnalyses@ process network, we also need the @join@ combinator, which pairs together the elements of two sorted input streams.
The combinator is parameterised by the key comparison function, which returns an @Ordering@ describing whether the key of the first argument is equal to the key of the second argument (@EQ@), lesser (@LT@), or greater (@GT@).
The process reads from two input streams (@sA@ and @sB@), and produces one output stream (@sOut@).
Two heap variables are used to store the most recent input elements (@va@ and @vb@), and another is used to store the key comparison (@c@).

In the first group of instructions, the instructions at labels @IN0@ and @IN1@ pull an element from each input stream.
If both pulls are successful, the instruction at label @IN2@ compares the input values using the key comparison function (@cmp va vb@).
Next, the instruction at label @IN3@ checks whether the keys are equal: if so, execution proceeds to the instruction at label @EQ0@; otherwise, instruction proceeds to the instruction at label @NE0@.

The group of instructions at label @EQ0@ execute when the element keys are equal, and pushes the pair of elements to the output stream, before dropping both input streams.
Execution then proceeds back to the instruction at label @IN0@ to pull from the inputs.

The instruction at label @NE0@ executes when the element keys are not equal, and proceeds to the instruction at label @LT0@ if the first element is lesser, or to the instruction at label @GT0@ if the first element is greater.
These two groups of instructions drop the input stream with the lesser key and pull a new value from the same input stream, before returning back to the instruction at label @IN2@ to compare the elements.

The group of instructions at label @DA0@ executes when the @sB@ input stream is finished, while the @sA@ input stream may still have elements.
As with the polarised stream implementation of @join_iii@ in \autoref{taxonomy/polarised}, we must drain the leftover elements from the unfinished stream by repeatedly pulling and dropping until there are no more elements.
This draining is required because each consumer has a one-element buffer for each input stream; the producer can only push when all consumers' buffers are empty.
Without draining, the producer would be blocked indefinitely on a terminated process, and any other consumers of the stream would be unable to receive input values.

As an alternative to draining, we could extend the process network semantics to include a ``disconnect'' instruction, which indicates that a process is no longer interested in consuming a particular input stream.
Here, we use the simpler process semantics without disconnection, at the expense of having to explicitly drain streams.
It may be tempting to instead modify the network semantics so that producers do not push to terminated processes, effectively disconnecting from all inputs upon termination.
However, a fused process, which performs the job of two input processes, only terminates once both processes are terminated; as such, the fused process would only disconnect once both input processes have terminated, which may be later than necessary.

% In the @priceAnalyses@ example, the @stock@ input stream is used by two consumers, the @join@ in @priceOverMarket@, and the @map@ in @priceOverTime@.
% When a stream has multiple consumers, all consumers must agree when to read the next value, otherwise execution might require an unbounded buffer.
% This draining ensures that a producer will not 
% Without draining, the producer of the unfinished input stream would continue pushing to , potentially requiring an unbounded buffer; we look at draining in more detail in \REF{kpn/draining}.

The group of instructions at label @DB0@ executes when the @sA@ input stream is finished, and drains the unfinished @sB@ input stream.

Finally, the group of instructions at @EX0@ close the output stream and terminate the process.

\section{Process definition}

%!TEX root = ../Main.tex

\begin{figure}
\begin{minipage}[t]{0.4\textwidth}
\begin{tabbing}
MMMMMx \TABDEF @MMMMM@  \TABSKIP $\Exp$ \TABSKIP $\Exp$ \TABSKIP $\Exp$ \kill

\Exp,~$e$       \> ::= \> ~~~ $x~|~v~|~e~e $ \\
                \> $\enskip|~$ \> ~~ $ (e~||~e) ~|~ e+e ~|~ e~@/=@~e ~|~ e < e$ \\
\Value,~$v$     \> ::= \> ~~~ $\mathbb{N}~|~\mathbb{B}~|~(\lambda{}x.~e)$ \\
\Heap,~$bs$     \> ::= \> ~~~ $\cdot~|~bs,~x~=~v$ \\
\Updates,~$us$  \> ::= \> ~~~ $\cdot~|~us,~x~=~e$
\\[0.5em]

\Proc,~$p$      \> ::=\> @process@ \\
MMMMMM \= M \= \kill
\> \> @ins:   @  $(\Chan ~\mapsto~ \InputState)$ \\
\> \> @outs:  @  $\sgl{\Chan}$ \\
\> \> @heap:  @  \Heap \\
\> \> @label: @  \Label \\
\> \> @instrs:@  $(\Label ~\mapsto~ \Instr)$ 
\\[0.5em]
\InputState \> ::= \> ~~ @none@~$|$~@pending@~\Value~$|$~@have@
\end{tabbing}

\begin{tabbing}
MMMMMM \TABDEF MMMM \TABSKIP $\Chan$ \TABSKIP $\Chan$ \TABSKIP $\Exp$ \kill

\Var,~$x$       \> $\to$ \> ~~~ (value variable) \\
\Chan,~$c$      \> $\to$ \> ~~~ (channel name) \\
\Label,~$l$     \> $\to$ \> ~~~ (label name) \\ 
\ChannelStates  \> ~ =   \> ~~~ $(\Chan \mapsto \InputState)$ \\
\Action,~$a$   \> ::=    \> ~~~ $\cdot ~~|~~ \Push~\Chan~\Value$ \\[0.5em]

\Instr
    \> ::=\> @pull@  \> \Chan  \> \Var  \> \Next \\
    \TABALT  @push@  \> \Chan  \> \Exp  \> \Next \\
    \TABALT  @drop@  \> \Chan  \>       \> \Next \\
    \TABALT  @case@  \> \Exp   \> \Next \> \Next \\
    \TABALT  @jump@  \>        \>       \> \Next \\
\\[0.5em]

\Next \> = \> $\Label~\times~\Updates$ 
\end{tabbing}
\end{minipage}
\caption{Process definitions}
\label{fig:Process:Def}
\end{figure}



The formal grammar for process definitions is given in \autoref{fig:Process:Def}.
Variables, Channels and Labels are specified by unique names.
We refer to the \emph{endpoint} of a stream as a channel.
A particular stream may flow into the input channels of several different processes, but can only be produced by a single output channel.
For values and expressions we use an untyped lambda calculus with a few primitives.
The `$||$' operator is boolean-or, `+' addition, `/=' not-equal, and `$<$' less-than.
The special @uninitialised@ value is used as a default value for uninitialised heap variables, and inhabits every type.

A $\Proc$ is a record with five fields: the @ins@ field specifies the input channels; the @outs@ field the output channels; the @heap@ field the process-local heap; the @label@ field the label of the instruction currently being executed; and the @instrs@ a map of labels to instructions.
We use the same record when specifying both the definition of a particular process, as well as when giving the evaluation semantics.
In the process definition, the @heap@ field gives the initial heap of the process, and any variables with unspecified values are assumed to be the @uninitialised@ value.
The @label@ field gives the entry-point in the process definition, though during evaluation it is the label of the instruction currently being executed.
Likewise, we usually only list channel names in the @ins@ field in the process definition, though during evaluation they are also paired with their current $\InputState$.
If an $\InputState$ is not specified we assume it is `none'.

In the grammar of \autoref{fig:Process:Def}, the $\InputState$ has four options: @none@, which means no value is currently stored in the associated stream buffer variable; $(@pending@~\Value)$, which gives the current value in the stream buffer variable and indicates that it has not yet been copied into a process-local variable; @have@, which means the pending value has been copied into a process-local variable; and @closed@, which means the producer has signalled that the channel is finished and will not receive any more values.
The $\Value$ attached to the @pending@ state is used when specifying the evaluation semantics of processes.
When performing the fusion transform, the $\Value$ itself will not be known, but we can still reason statically that a process must be in the @pending@ state.
When defining the fusion transform in \autoref{s:Fusion}, we will use a version of $\InputState$ with only this statically known information.

The @instrs@ field of the $\Proc$ maps labels to instructions.
The possible instructions are: @pull@, which tries to pull the next value from a channel into a given heap variable, blocking until the producer pushes a value or closes the channel; @push@, which pushes the value of an expression to an output channel; @close@, which signals the end of an output channel; @drop@, which indicates that the current value pulled from a channel is no longer needed; @case@, which branches based on the result of a boolean expression; @jump@, which causes control to move to a new instruction; and @exit@, which signals that the process is finished.

Instructions include a $\Next$ field containing the label of the next instruction to execute, as well as a list of $\Var \times \Exp$ bindings used to update the heap.
The list of update bindings is attached directly to instructions to make the fusion algorithm easier to specify, in contrast to a presentation with a separate @update@ instruction.

% When lowering process code to a target language, as in \autoref{chapter:process:implementation}, we can safely convert @drop@ to plain @jump@ instructions.
% The @drop@ instructions are used to control how processes should be synchronised, but do not affect the execution of a single process.
% We discuss @drop@s further in \autoref{s:Drop:in:synchrony}.

% This allows us to \emph{deliberately} introduce artificial deadlocks when a process network would require more than one element of buffering.
%%% AR: added to highlight that this rules out networks that require unbounded buffers
%%% BL: We don't have any examples of explicitly introducing deadlocks. The process networks just happen to have them when viewed abstractly.

%%% AR: feels a bit disjointed because drops were only mentioned once a few paragraphs ago. Maybe reword to talk about lowering in general is obvious for most instructions, and drops are just treated as jumps. Or move up.


% -----------------------------------------------------------------------------
\subsection{Execution}
\label{s:Process:Eval}

The dynamic execution of a process network consists of:

\begin{enumerate}
\item \emph{Injection} of an action, which is an optional stream value or stream close message, into a process or a network.
  Each individual process only accepts an injected value when it is ready for it, and injection into a network succeeds only when \emph{all} processes accept it.

\item \emph{Advancing} a single process from one state to another, producing an output action.
  Advancing a network succeeds when \emph{any} of the processes in the network can advance, and the output action can be injected into \emph{all} the other processes.

\item \emph{Feeding} input values from the environment into processes, and collecting outputs of the processes.
  Feeding alternates between Injecting values from the environment and Advancing the network, until all processes have terminated.
    When a process pushes a value to an output channel, we collect this value in a list associated with the output channel.
\end{enumerate}

Execution of a network is non-deterministic.
At any moment, several processes may be able to take a step, while others are blocked.
As with Kahn processes~\cite{kahn1976coroutines}, pulling from a channel is blocking, which enables the overall sequence of values on each output channel to be deterministic.
Unlike Kahn processes, pushing to a channel can also block.
Each consumer has a single element buffer, and pushing only succeeds when that buffer is empty.

%%% AR: what is the distinction between 'execution' and 'evaluation'?  I only have a vague feeling that execution is something a computer does, while evaluation is the mathematical rules. Either way, these should probably be consistent.
%%% BL: "Evaluation" is pure.  E-"value"-ation. Execution has visible actions, like pushing to streams.

% TODO BL: Mention what happens if we choose a different ordering,
% and how the particular ordering chosen is decided upon.
Importantly, it is the order in which values are \emph{pushed to each particular output channel} which is deterministic, whereas the order in which different processes execute their instructions is not.
When we fuse two processes, we choose one particular instruction ordering that enables the network to advance without requiring unbounded buffering.
The single ordering is chosen by heuristically deciding which pair of states to merge during fusion, and is discussed in \autoref{s:EvaluationOrder}.

Each channel may be pushed to by a single process only, so in a sense each output channel is owned by a single process.
The only intra-process communication is via channels and streams.
Our model is ``pure data flow'' as there are no side-channels between processes --- in contrast to ``impure data flow'' systems such as StreamIt~\cite{thies2002streamit}.


%!TEX root = ../Main.tex

\begin{figure}

$$
\arrLR{
  \boxed{\ProcInject{\Proc}{\Value}{\Chan}{\Proc}}
}{
  \boxed{\ProcsInject{\sgl{\Proc}}{\Value}{\Chan}{\sgl{\Proc}}}
}
$$

$$
\ruleIN{
  p[@ins@][c] = @none@
}{
  \ProcInject{p}{v}{c}{p~[@ins@ \mapsto (p[@ins@][c \mapsto @pending@~v]) ] }
}{InjectValue}
$$

$$
\ruleIN{
  c \not\in p[@ins@]
}{
  \ProcInject{p}{v}{c}{p}
}{InjectIgnore}
%
\quad
%
\ruleIN{
  \{~ \ProcInject{p_i}{v}{c}{p'_i} ~\}^i
}{
  \ProcsInject{\sgl{p_i}^i}{v}{c}{\sgl{p'_i}^i}
}{InjectMany}
$$

\caption{Injection of values into input channels}
\label{fig:Process:Eval:Inject}
\end{figure}



% -----------------------------------------------------------------------------
\subsubsection{Injection}
\autoref{fig:Process:Eval:Inject} gives the rules for injecting values into processes.
Injection is a meta-level operation, in contrast to @pull@ and @push@, which are instructions in the object language.
The statement $(\ProcInject{p}{a}{p'})$ reads ``given process $p$, injecting action $a$ yields an updated process $p'$''.
An action $a$ is a message describing the state change that can occur to a channel, with three options: $(\cdot)$, the empty action, used when a process simply updates internal state; $(\Push~\Chan~\Value)$, which encodes the value a process pushes to one of its output channels; and $(\Close~\Chan)$, which denotes the end of the stream.
The @injects@ form is similar to the @inject@ form, operating on a process network.

Rule (InjectPush) injects a single value into a single process. The value is stored as a (@pending@~ v) binding in the $\InputState$ of the associated channel of the process. The $\InputState$ acts as a single element buffer, and must be empty (@none@) for injection to succeed.
Rule (InjectClose) injects a close message and updates the input state in a similar way.

Rules (InjectNopPush) and (RuleNopClose) allow processes that do not use a particular named channel to ignore messages injected into that channel.
Rule (InjectNopInternal) allows processes to ignore empty messages.

Rule (InjectMany) injects a single value into a network.
We use the single process judgment form to inject the value into all processes, which must succeed for all of them.
To inject a message into a process network, all the processes which do not ignore the message must be ready to accept the message by having the corresponding $\InputState$ set to @none@; otherwise, the process would require more than a single-element buffer to store multiple messages.
% Once a value has been injected into all consuming processes that require it, the producing process no longer needs to retain it.


%!TEX root = ../Main.tex

% -----------------------------------------------------------------------------
\begin{figure}
\begin{tabbing}
MM \= MMMMMMM \= MM \= MMMMMMMMM\kill
\end{tabbing}


% -----------------------------
$$
  \boxed{
    \ProcBlockShake
      {\Instr}
      {\ChannelStates}
      {\Heap}
      {\Action}
      {\Label}
      {\ChannelStates}
      {\Updates}
  }
$$

$$
\ruleIN{
  is[c] = @pending@~v
}{
  \ProcBlockShake
        {@pull@~c~x~(l,us)~(l',us')}
        {is}
        {\Sigma}
        {\cdot}
        {l}
        {is[c \mapsto @have@]}
        {(us, x = v)}
}{PullPending}
$$

$$
\ruleIN{
  is[c] = @closed@
}{
  \ProcBlockShake
        {@pull@~c~x~(l,us)~(l',us')}
        {is}
        {\Sigma}
        {\cdot}
        {l'}
        {is}
        {us'}
}{PullClosed}
$$

$$
\ruleIN{
  \ExpEval{\Sigma}{e}{v}
}{
  \ProcBlockShake
        {@push@~c~e~(l,us)}
        {is}
        {\Sigma}
        {\Push~c~v}
        {l}
        {is}
        {us}
}{Push}
$$

$$
\ruleAx{
  \ProcBlockShake
        {@close@~c~(l,us)}
        {is}
        {\Sigma}
        {\Close~c}
        {l}
        {is}
        {us}
}{Close}
$$


$$
\ruleIN{
  is[c] = @have@
}{
  \ProcBlockShake
        {@drop@~c~(l,us)}
        {is}
        {\Sigma}
        {\cdot}
        {l}
        {is[c \mapsto @none@]}
        {us}
}{Drop}
\ruleIN{
}{
  \ProcBlockShake
        {@jump@~(l,us)}
        {is}
        {\Sigma}
        {\cdot}
        {l}
        {is}
        {us}
}{Jump}
$$

$$
\ruleIN{
  \ExpEval{\Sigma}{e}{@True@}
}{
  \ProcBlockShake
        {@case@~e~(l_t,us_t)~(l_f,us_f)}
        {is}
        {\Sigma}
        {\cdot}
        {l_t}
        {is}
        {us_t}
}{CaseT}
$$
$$
\ruleIN{
  \ExpEval{\Sigma}{e}{@False@}
}{
  \ProcBlockShake
        {@case@~e~(l_t,us_t)~(l_f,us_f)}
        {is}
        {\Sigma}
        {\cdot}
        {l_f}
        {is}
        {us_f}
}{CaseF}
$$

\vspace{2em}

% ----------------
$$
  \boxed{\ProcShake{\Proc}{\Action}{\Proc}}
$$
$$
\ruleIN{
  \ProcBlockShake
    {p[@instrs@][p[@label@]]} 
    {p[@ins@]}
    {p[@heap@]}
    {a}
    {l}
    {is}
    {us}
  \qquad
    \ExpEval{p[@heap@]}{us}{bs}
}{
  \ProcShake
        {p}
        {a}
        {p~[    @label@~ \mapsto ~l
           , ~~ @heap@~  \mapsto (p[@heap@] \lhd bs)
           , ~~ @ins@~   \mapsto ~is]}
}{Advance}
$$

% ---------------------------------------------------------
\vspace{2em}

$$
  \boxed{\ProcsShake{\sgl{\Proc}}{\Action}{\sgl{\Proc}}}
$$

$$
\ruleIN{
  \ProcShake{p_i}{a}{p'_i}
  \qquad
  \forall j~|~j \neq i.~
  \ProcInject{p_j}{a}{p'_j}
}{
  \ProcsShake{
    \sgl{p_0 \ldots p_i \ldots p_n}
  }{a}{
    \sgl{p'_0 \ldots p'_i \ldots p'_n}
  }
}{AdvanceMany}
$$



\caption{Advancing processes}

% Evaluation: shaking allows proceses to take a step from one label to another as well as produce an output message. If the message is a push, the value is injected to all other processes in the network; otherwise it is an internal step.}
\label{fig:Process:Eval:Shake}
\end{figure}




% -----------------------------------------------------------------------------
\subsubsection{Advancing}
\autoref{fig:Process:Eval:Shake} gives the rules for advancing a single process and process networks.
The statement $(\ProcBlockShake{i}{is}{bs}{a}{l}{is'}{us'})$ reads ``instruction $i$, given channel states $is$ and the heap bindings $bs$, passes control to instruction at label $l$ and yields new channel states $is'$, heap update expressions $us'$, and performs an output action $a$.''

Rule (PullPending) takes the @pending@ value $v$ from the channel state and produces a heap update to copy this value into the variable $x$ in the @pull@ instruction.
Control is passed to the first output label, $l$.
We use the syntax ($us,x=v$) to mean that the list of updates $us$ is extended with the new binding ($x=v$).
In the result channel states, the state of the input channel $c$ is updated to @have@, to indicate that the value has been copied into the local variable.

Rule (PullClosed) applies when the channel state is @closed@, passing control to the second output label, $l'$.
As the channel remains closed, there is no need to update the channel state as in the (PullPending) rule.

Rule (Push) evaluates the expression $e$ under heap bindings $bs$ to a value $v$, and produces a corresponding action which carries this value.
The judgment $(bs \vdash e \Downarrow v)$ expresses standard untyped lambda calculus reduction, using the heap $bs$ for the values of free variables.
As this evaluation is completely standard, we omit it to save space.

Rule (Close) emits a @close@ action; once injected, this action will transition the recipients' channel states to @closed@.
Once a channel is closed it can no longer be pushed to, as the recipients' channel states cannot transition back to the @none@ state required by the (InjectPush) rule.

Rule (Drop) changes the input channel state from @have@ to @none@. A @drop@ instruction can only be executed after @pull@ has set the input channel state to @have@.

Rule (Jump) produces a new label and associated update expressions. Rules (CaseT) and (CaseF) evaluate the scrutinee $e$ and emit the appropriate label.

There is no corresponding rule for the @exit@ instruction, which denotes a finished process.

The statement ($\ProcShake{p}{a}{p'}$) reads ``process $p$ advances to new process $p'$, yielding action $a$''. Rule (Advance) advances a single process. We look up the current instruction for the process' @label@ and pass it, along with the channel states and heap, to the above single instruction judgment. The update expressions $us$ from the single instruction judgment are reduced to values before updating the heap. We use $(us \lhd bs)$ to replace bindings in $us$ with new ones from $bs$. As the update expressions are pure, the evaluation can be done in any order.

The statement ($\ProcShake{ps}{a}{ps'}$) reads ``the network $ps$ advances to the network $ps'$, yielding action $a$''.
Rule (AdvanceMany) allows an arbitrary, non-deterministically chosen process in the network to advance to a new state while yielding an output action $a$.
For this to succeed, it must be possible to inject the action into all the other processes in the network.
As all consuming processes must accept the output action at the time it is created, there is no need to buffer it further in the producing process.
When any process in the network produces an output action, we take that as the action of the whole network.

%!TEX root = ../Main.tex

\begin{figure}

\newcommand\vs {\ti{vs}}
\newcommand\ps {\ti{ps}}

$$
  \boxed{
    \ProcsFeed
      {(\Chan \mapsto \overline{Value})~}
      {\sgl{\Proc}}
      {(\Chan \mapsto \overline{Value})~}
      {\sgl{\Proc}}
  }
$$

$$
\ruleIN{
  \ProcsShake
    {ps}
    {\cdot}
    {ps'}
\qquad
  \ProcsFeed{i}{ps'}{o}{ps''}
}{
  \ProcsFeed
    {i}
    {ps}
    {o}
    {ps''}
}{FeedInternal}
$$

$$
\ruleIN{
  \ProcsShake
    {ps}
    {\Push~c~v}
    {ps'}
\qquad
  \ProcsFeed
    {i}
    {ps'}
    {o}
    {ps''}
\qquad
  (c \mapsto \vs) \in o
}{
  \ProcsFeed
    {i}
    {ps}
    {o[c \mapsto ([v] \listappend \vs)]}
    {ps''}
}{FeedPush}
$$

$$
\ruleIN{
  \ProcsShake
    {ps}
    {\Close~c}
    {ps'}
\qquad
  \ProcsFeed
    {i}
    {ps'}
    {o}
    {ps''}
}{
  \ProcsFeed
    {i}
    {ps}
    {o[c \mapsto []]}
    {ps''}
}{FeedClose}
$$





$$
\ruleIN{
%   (\forall p \in \ps.~c \not\in p[@outs@])
% \qquad
  \ProcsInject
    {ps}
    {(\Push~c~v)}
    {ps'}
\qquad
  \ProcsFeed
    {i[c \mapsto \vs]}
    {ps'}
    {o}
    {ps''}
}{
  \ProcsFeed
    {i[c \mapsto ([v] \listappend vs)]}
    {ps}
    {o}
    {ps''}
}{FeedExternalPush}
$$

$$
\ruleIN{
%  (\forall p \in \ps.~c \not\in p[@outs@])
%\qquad
  \ProcsInject
    {ps}
    {(\Close~c)}
    {ps'}
\qquad
  \ProcsFeed
    {i}
    {ps'}
    {o}
    {ps''}
}{
  \ProcsFeed
    {i[c \mapsto [] ]}
    {ps}
    {o}
    {ps''}
}{FeedExternalClose}
$$



\caption{Feeding Process Networks}
\label{fig:Process:Eval:Feed}
\end{figure}



% -----------------------------------------------------------------------------
\subsubsection{Feeding}
\autoref{fig:Process:Eval:Feed} gives the rules for collecting output actions and feeding external input values to the network.
These rules exchange input and output values with the environment in which the network runs.
% The first set of rules concerns feeding values to other processes within the same network, while the second exchanges input and output values with the environment the network is running in.

The statement ($\ProcsFeed{i}{ps}{o}$) reads ``when fed input channel values $i$, network $ps$ executes to termination of all processes, and produces output channel values $o$''.
The input channel values map $i$ contains a list of values for each input channel; these channels are inputs of the overall network, and cannot be outputs of any processes.
The output channel values map $o$ contains the list of values for every output channel in the network.
In a concrete implementation the input and output values would be transported over some IO device, but for the semantics we describe the abstract behavior only.

Rule (FeedExit) terminates execution of a network when all processes have terminated.
We require the input channel values map to be empty, as any leftover input values would be ignored by the network.
The output channel values map is empty.

Rule (FeedInternal) allows the network to perform local computation in the context of the channel values.
This does not affect the input or output values, and execution proceeds with the updated process network.

Rule (FeedPush) collects an output action containing a pushed value (@push@ $c$ $v$) produced by a network.
The input is fed to the updated process, which results in output channel map $o$.
At this point, the output channel map $o$ contains the result of executing the remainder of the process network, after the push has happened.
In the output, the pushed value $v$ is added to the start of the list corresponding to the output channel $c$.

Rule (FeedClose) collects a close output action (@close@ $c$) produced by a network.
The output channel map for the channel $c$ is set to the empty list; earlier pushes will prefix elements to this list using rule (FeedPush).

Rule (FeedEnvPush) injects values from the external environment as push messages.
The updated process network, after having the value injected, is fed the remainder of the input without the pushed value.
% This rule also has the side condition that values cannot be injected from the environment into output channels that are already owned by some process.
% This constraint is required for correctness proofs, but can be ensured by construction in a concrete implementation.

Rule (FeedEnvClose) injects a close message for an external input stream when the corresponding list is empty.
When execution continues with the updated process network, the input stream is removed from the channel map using the ($i \setminus \sgl{c}$) syntax.

% The topology of the dataflow network does not change at runtime, so it only needs to be checked once, before execution.


% -----------------------------------------------------------------------------
\subsection{Non-deterministic Execution Order}
\label{s:EvaluationOrder}

The execution rules of \autoref{fig:Process:Eval:Feed} are non-deterministic in several ways. Rule (ProcessInternal) allows any process to perform internal computation at any time, without synchronising with other processes in the network; (ProcessPush) allows any process to perform a push action at any time, provided all other processes in the network are ready to accept the pushed value; (FeedExternal) also allows new values to be injected from the environment, provided all processes that use the channel are ready to accept the value.

In the semantics, allowing the execution order of processes to be non-deterministic is critical, as it defines a search space where we might find an order that does not require unbounded buffering. For a direct implementation of concurrent processes using message passing and operating system threads, an actual, working, execution order would be discovered dynamically at runtime. In contrast, the role of our fusion system is to construct one of these working orders statically. In the fused result process, the instructions will be scheduled so that they run in one of the orders that would have arisen if the network were executed dynamically. Fusion also eliminates the need to pass messages between processes --- once they are fused we can just copy values between heap locations.

% In our system, allowing the execution order of processes to be non-deterministic is critical, as it provides freedom to search for a valid ordering that does not require excessive buffering. Consider the following example, where the @alt2@ operator pulls two elements from its first input stream, then two from the second, before pushing all four to its output stream.
% \begin{code}
%   alternates : S Nat -> S Nat -> S Nat -> S (Nat, Nat)
%   alternates sInA sInB sInC
%    = let  s1   = alt2 sInA sInB
%           s2   = alt2 sInB sInC
%           sOut = zip s1 s2
%      in   sOut
% \end{code}
%
% Note that the middle stream @sInB@ is shared, and the result streams from both @alt2@ operators are zipped into tuples. Given the inputs @sInA@ = @[a1,a2]@, @sInB@ = @[b1,b2]@ and @sInC@ = @[c1,c2]@ the output of @zip@ will be @[(a1,b1),(a2,b2),(b1,c1),(b2,c2)]@, assuming @a1,a2,b1,b2@ and so on are values of type @Nat@.
%
% Now, note that the first @alt2@ process pushes values to its output stream @s1@ two at a time, and the second @alt2@ process also pushes values to its own output stream @s2@ two at a time. However, the downstream @zip@ process needs to pull one value from @s1@ then one from @s2@, then another from @s1@, then another from @s2@, alternating between the @s1@ and @s2@ streams. This will work, provided we can arrange for the two \emph{separate} @alt2@ processes to push to their separate output streams alternatively. They can still push two values at a time to their own outputs, but the downstream @zip@ process needs receive one from each process alternately. Here is a table of intermediate values to help make the explanation clearer:
%
% \begin{code}
%     sInA = [a1, a2, a3, a4, a5 ...]
%     sInB = [b1, b2, b3, b4, b5 ...]
%     sInC = [c1, c2, c3, c4, c5 ...]
%
%     s1   = alt2 sInA sInB
%          = [a1, a2, b1, b2, a3, a4, b3, b4 ...]
%
%     s2   = alt2 sInB sInC
%          = [b1, b2, c1, c2, b3, b4, c3, c4 ...]
%
%     sOut = zip s1 s2
%          = [(a1,b1), (a2,b2), (b1,c1), (b2,c2) ...]
% \end{code}
%
% Considering the last line in the above table, note that @zip@ needs to output a tuple of @a1@ and @b1@ together, then @a2@ and @b2@ together, and so on. The implementation of the @zip@ process will attempt to pull the first value @a1@ from stream @s1@, blocking until it gets it, then pull the next value @b1@ from stream @s2@, blocking until it gets it. While @zip@ is blocked waiting for @b1@, the first @alt2@ process cannot yet push @a2@. The execution order of the overall network is constrained by communication patterns of processes in that network.

% As we cannot encode all possible orderings into the definition of the processes themselves, we have defined the execution rules to admit many possible orderings. In a direct implementation of concurrent processes using message passing and operating system threads, an actual, working, execution order would be discovered dynamically at runtime. In contrast, the role of our fusion transform is to construct one of these working orders statically. In the fused result process, the instructions will be scheduled so that they run in one of the orders that would have arisen if the network was executed dynamically. In doing so, we also eliminate the need to pass messages between processes --- once they are fused we can just copy values between heap locations.

% Although alt2 produces output elems two at a time, the consumer zip need its input elements to arrive alternately. At evaluation time we need the results pushed to sA1 and sA2 in the sA1 sA2 sA1 sA2 order, not sA1 sA1 sA2 sA2. Writing the rules nondeterministically allows the elaborator to discover a usable order, if there is one. This also affects fusion, we don't want to commit to the wrong order too early. We shall see that if we fuse the two alt processes first fusion will not work. We need to start with zip so that the order in which input elems arrive is constrained.





\section{Fusion}
\label{s:Fusion}

Our core fusion algorithm constructs a static execution schedule for a single pair of processes.
In \autoref{ss:Fusing:a:network} we fuse a whole process network by fusing successive pairs of processes until only one remains.

\autoref{fig:Fusion:Types} defines some auxiliary grammar used during fusion. We extend the $\Label$ grammar with a new alternative, $\LabelF \times \LabelF$ for the labels in a fused result process. Each $\LabelF$ consists of a $\Label$ from a source process, paired with a map from $\Chan$ to the statically known part of that channel's current $\InputState$. When fusing a whole network, as we fuse pairs of individual processes the labels in the result collect more and more information. Each label of the final, completely fused process encodes the joint state that all the original source processes would be in at that point.

% The definition of $\Label$ is now recursive.
% These new labels $LabelF$ consist of a pair of source labels, as well as the static part of the $\InputState$ of each input channel.

% If the static $\InputStateF$ is $@pending@_F$, there is a value waiting to be pulled, do not know the actual value.

We also extend the existing $\Var$ grammar with a (@chan@ $c$) form which represents the buffer variable associated with \mbox{channel @c@}. We only need one buffer variable for each channel, and naming them like this saves us from inventing fresh names in the definition of the fusion rules.
We used the name (@chan_tp@) back in \autoref{s:FusingProcesses} to avoid introducing a new mechanism at that point in the discussion, when in fact the fused process would use a buffer variable called (@chan timeprices@).

Still in \autoref{fig:Fusion:Types}, $\ChanTypeTwo$ classifies how channels are used, and possibly shared, between two processes.
Type @in2@ indicates that both processes @pull@ from the same channel, so these actions must be coordinated.
Type @in1@ indicates that only a single process pulls from the channel.
Type @in1out1@ indicates that one process pushes to the channel and the other pulls.
Type @out1@ indicates that the channel is pushed to by a single process.
Each output channel is uniquely owned and cannot be pushed to by more than one process.

%!TEX root = ../Main.tex

\begin{figure}

\begin{tabbing}
@MMMMMMMMMMMM@   \TABDEF \kill

$\Label$        \> ::=  \> ~~~ $\ldots ~|~\LabelF~\times~\LabelF ~|~ \ldots$ \\
$\LabelF$      \> =    \> ~~~ $\Label~\times~\MapType{\Chan}{\InputStateF}$  \\
$\InputStateF$ \> ::=  \> ~~~ $@none@_F ~|~ @pending@_F ~|~ @have@_F ~|~ @closed@_F$    \\
$\Var$          \> ::=  \> ~~~ $\ldots ~|~@chan@~\Chan ~|~ \ldots$ \\
\\

$\ChanType_2$   \> ::=  \> ~~~ $@in2@~|~@in1@~|~@in1out1@~|~@out1@$
\end{tabbing}

\caption{Fusion type definitions.}
% The labels for a fused program consists of both of the original program labels, as well as the statically known part of the input state for each channel. The channels of both processes are classified into inputs and outputs, this describes what coordination is required between the two.
\label{fig:Fusion:Types}
\end{figure}



%!TEX root = ../Main.tex

\begin{figure}

\begin{tabbing}
\ti{fusePair}~@M M@   \TABDEF \kill

\ti{fusePair} \> $~:$ \> $\Proc \to \Proc \to  \Maybe~\Proc$ \\
$\ti{fusePair}~p~q$ \> $=$ \\
@    process@ \\
@        ins: @ $\sgl{c~|~c=t \in \cs,~t \in \sgl{@in1@,@in2@}} $ \\
@       outs: @ $\sgl{c~|~c=t \in \cs,~t \in \sgl{@in1out1@,@out1@}} $ \\
@       heap: @ $p[@heap@]~\cup~q[@heap@]$ \\
@      label: @ $l_0$ \\
@     instrs: @ $\ti{go}~\sgl{}~l_0$ \\
@ where@ \\
MM\=MM\=~=~\=\kill
 \> \cs \> $=$ \> $\ti{channels}~p~q$ \\
 \> $l_0$   \> $=$ \> $
      \big( 
      (p[@label@],~\sgl{c=none_S~|~c~\in~p[@ins@]}),~
      (q[@label@],~\sgl{c=none_S~|~c~\in~q[@ins@]})
      \big)$ \\
 \\
 \> $\ti{go}~\ti{bs}~(l_p,l_q)$ \\
 \> \> $~|$ \> $(l_p,l_q)~\in~\ti{bs}$ \\
 \> \> $=$  \> $\ti{bs}$ \\
 \> \> $~|$ \>
        $b~\in~\ti{tryStepPair}~\cs~l_p~p[@instrs@][l_p]~l_q~q[@instrs@][l_q]$ \\ 
 \> \> $=$ \> $\ti{fold}~\ti{go}~(\ti{bs}~\cup~\sgl{(l_p,l_q)=b})~(\ti{outlabels}~b)$
\end{tabbing}

\caption{Fusion of pairs of processes}

% Two processes are fused together by starting at the initial label for each process and computing the instruction based on one of the original process' instructions at that label. Instructions are added recursively until all reachable instructions are included.

\label{fig:Fusion:Def:Top}
\end{figure}






\smallskip
% -------------------------------------
\autoref{fig:Fusion:Def:Top} defines function \ti{fusePair} that fuses a pair of processes, constructing a result process that does the job of both. We start with a joint label $l_0$ formed from the initial labels of the two source processes. We then use \ti{tryStepPair} to statically choose which of the two processes to advance, and hence which instruction to execute next. The possible destination labels of that instruction (computed with $outlabels$ from \autoref{fig:Fusion:Utils}) define new joint labels and reachable states. As we discover reachable states we add them to a map $bs$ of joint label to the corresponding instruction, and repeat the process to a fixpoint where no new states can be discovered.

%!TEX root = ../Main.tex

% Settle on a syntax for this later.
% Might be too many nested pairs.
\newcommand\nextStep[5]{\big((#1,~#2),~(#3,~#4),~#5 \big)}


% I tried this colour in a colour blindness simulator and it seems to be OK.
% Should still be readable when converted to grayscale.
\definecolor{notec}{HTML}{C03020}
\newcommand\note[1]{\textcolor{notec}{(#1)}}

\begin{figure}
\begin{tabbing}
M \= M \= M \= M \kill
$\ti{tryStepPair} ~:~ \ChanTypeMap \to \Label_1 \to \Instr \to \Label_1 \to \Instr \to \Maybe~\Instr$ \\
$\ti{tryStepPair} ~\cs~l_p~i_p~l_q~i_q$ \\

\> \note{PreferJump1} \\
\> $~|~i_p'~\in~\ti{tryStep}~\cs~l_p~i_p~l_q ~\wedge~@jump@~(l,u)~\in~i_p'$ 
   ~~~~~ $\to~i_p'$ \\
\> \note{PreferJump2} \\
\> $~|~i_q'~\in~\ti{tryStep}~\cs~l_q~i_q~l_p ~\wedge~@jump@~(l,u)~\in~i_q'$
   ~~~~~ $\to~\ti{swaplabels}~i_q'$ 
\\[0.5em]

\> \note{DeferPull1} \\
\> $~|~i_p'~\in~\ti{tryStep}~\cs~l_p~i_p~l_q ~\wedge~ i_q'~\in~\ti{tryStep}~\cs~l_q~i_q~l_p$ 
   ~~ $\wedge~@pull@~c~x~(l,u)~\not\in~i_p'$ 
   ~~~~~ $\to~i_p'$ \\
\> \note{DeferPull2} \\
\> $~|~i_p'~\in~\ti{tryStep}~\cs~l_p~i_p~l_q ~\wedge~i_q'~\in~\ti{tryStep}~\cs~l_q~i_q~l_p$
   ~~ $\wedge~@pull@~c~x~(l,u)~\not\in~i_q'$ 
   ~~~~~ $\to~\ti{swaplabels}~i_q'$ 
\\[0.5em]

\> \note{Run1} \\
\> $~|~i_p'~\in~\ti{tryStep}~\cs~l_p~i_p~l_q$ ~~~~~ $\to~i_p'$ \\
\> \note{Run2} \\
\> $~|~i_q'~\in~\ti{tryStep}~\cs~l_q~i_q~l_p$ ~~~~~ $\to~\ti{swaplabels}~i_q'$
\end{tabbing}
\caption{Fusion step coordination for a pair of processes.}
% Statically compute the instruction to perform at a particular fused label. Try to execute either process, preferring jumps and other instructions over pulling, as pulling can block while other instructions may perform ``useful work'' without blocking. If neither machine can execute, fusion fails.
\label{fig:Fusion:Def:StepPair}
\end{figure}



% -------------------------------------
\autoref{fig:Fusion:Def:StepPair} defines function \ti{tryStepPair} which decides which process to advance. It starts by calling \ti{tryStep} for both processes. If both can advance, we use heuristics to decide which one to run first.

Clauses (DeferExit1) and (DeferExit2) ensure that the fused process only terminates once both processes are ready to terminate; if either has remaining work, the process with remaining work will execute.
The clauses achieve this by checking if either process is at an @exit@ instruction, and if so, choosing the other process.
The instruction for the second process was computed by calling \ti{tryStep} with the label arguments swapped, so in (DeferExit2) we need to swap the labels back with $\ti{swaplabels}$ (from \autoref{fig:Fusion:Utils}).

Clauses (PreferJump1) and (PreferJump2) prioritise processes that can perform a @jump@.
This helps collect @jump@ instructions together so they are easier for post-fusion optimisation to handle (\autoref{s:Optimisation}).

Similarly, clauses (DeferPull1) and (DeferPull2) defer @pull@ instructions: if one of the instructions is a @pull@, we advance the other one. We do this because @pull@ instructions may block, while other instructions are more likely to produce immediate results.

Clauses (Run1) and (Run2) apply when the above heuristics do not apply, or only one of the processes can advance.
% We try the first process first, and if that can advance then so be it. This priority means that fusion is left-biased, preferring advancement of the left process over the second.

Clause (Deadlock) applies when neither process can advance, in which case the processes cannot be fused together and fusion fails.


%!TEX root = ../Main.tex

\begin{figure*}

\begin{tabbing}
$\ti{tryStep} ~:~ \ChanTypeMap \to \LabelF \to \Instr \to \LabelF \to \Maybe~\Instr$ \\
$\ti{tryStep} ~\cs~(l_p,s_p)~i_p~(l_q,s_q)~=~@match@~i_p~@with@$ 
\end{tabbing}

\vspace*{-\baselineskip}
\vspace*{1ex}

\begin{tabular}{lr}
  $@jump@~(l',u')$  & \note{LocalJump} \\
  $\quad\to~\Just (@jump@~
      \nextStep
        {l'}{s_p}
        {l_q}{s_q}
        {u'})
      $ 
      \\[1ex]
  $@case@~e~(l'_t,u'_t)~(l'_f,u'_f)$ & \note{LocalCase} \\
$\quad\to~\Just (@case@~e~
      \nextStep
        {l'_t}{s_p}
        {l_q}{s_q}
        {u'_t}
      ~
      \nextStep
        {l'_f}{s_p}
        {l_q}{s_q}
        {u'_f})
      $ 
\\[1ex]

$@push@~c~e~(l',u')$ \\
  $\quad~|~\cs[c]=@out1@$ & \note{LocalPush} \\
$\quad\to~\Just (@push@~c~e~
      \nextStep
        {l'}
          {s_p}
        {l_q}
          {s_q}
        {u'})
      $ 
\\
  $\quad~|~\cs[c]=@in1out1@ ~\wedge~ s_q[c]=@none@_F$ & \note{SharedPush} \\
$\quad\to~\Just (@push@~c~e~
      \nextStep
        {l'}
          {s_p}
        {l_q}
          {\HeapUpdateOne{c}{@pending@_F}{s_q}}
        {\HeapUpdateOne{@chan@~c}{e}{u'}})
      $
\\[1ex]

$@drop@~c~(l',u')$ \\
  $\quad~|~\cs[c]=@in1@$ & \note{LocalDrop} \\
  $\quad\to~\Just (@drop@~c~
      \nextStep
        {l'}
          {s_p}
        {l_q}
          {s_q}
        {u'})
      $
      \\

  $\quad~|~\cs[c]=@in1out1@$ & \note{ConnectedDrop} \\
$\quad\to~\Just (@jump@~
      \nextStep
        {l'}
          {\HeapUpdateOne{c}{@none@_F}{s_p}}
        {l_q}
          {s_q}
        {u'})
      $
      \\

$\quad~|~\cs[c]=@in2@ ~\wedge~ (s_q[c]=@have@_F \vee s_q[c]=@pending@_F)$  & \note{SharedDropOne} \\
$\quad\to~\Just (@jump@~
      \nextStep
        {l'}
          {\HeapUpdateOne{c}{@none@_F}{s_p}}
        {l_q}
          {s_q}
        {u'})
      $
      \\


$\quad~|~\cs[c]=@in2@ ~\wedge~ s_q[c]=@none@_F$ & \note{SharedDropBoth} \\
$\quad\to~\Just (@drop@~c~
      \nextStep
        {l'}
          {\HeapUpdateOne{c}{@none@_F}{s_p}}
        {l_q}
          {s_q}
        {u'})
      $
\\[1ex]

$@pull@~c~x~(l'_o,u'_o)~(l'_c,u'_c)$ \\

$\quad~|~\cs[c]=@in1@$ & \note{LocalPull} \\
$\quad\to~\Just (@pull@~c~x~
      \nextStep
        {l'_o}{s_p}
        {l_q}{s_q}
        {u'_o}
      ~
      \nextStep
        {l'_c}{s_p}
        {l_q}{s_q}
        {u'_c})
    $ 
\\[1ex]

$\quad~|~(\cs[c]=@in2@ \vee \cs[c]=@in1out1@) ~\wedge~ s_p[c]=@pending@_F$ & \note{SharedPull} \\
$\quad\to~\Just (@jump@~
      \nextStep
        {l'_o}
          {\HeapUpdateOne{c}{@have@_F}{s_p}}
        {l_q}
          {s_q}
        {\HeapUpdateOne{x}{@chan@~c}{u'_o}})
        $ 
\\[1ex]

$\quad~|~\cs[c]=@in2@ ~\wedge~ s_p[c]=@none@_F ~\wedge~ s_q[c]=@none@_F$ & \note{SharedPullInject} \\
$\quad\to~\Just (@pull@~c~(@chan@~c)$ \\
@              @
      $\nextStep
        {l_p}
          {\HeapUpdateOne{c}{@pending@_F}{s_p}}
        {l_q}
          {\HeapUpdateOne{c}{@pending@_F}{s_q}}
        {[]}$
      \\
@              @
      $\nextStep
        {l_p}
          {\HeapUpdateOne{c}{@closed@_F}{s_p}}
        {l_q}
          {\HeapUpdateOne{c}{@closed@_F}{s_q}}
        {[]})
  $
\\[1ex]

$\quad~|~(\cs[c]=@in2@ \vee \cs[c]=@in1out1@) ~\wedge~ s_p[c]=@closed@_F$ & \note{SharedPullClosed} \\
$\quad\to~\Just (@jump@~
      \nextStep
        {l'_c}{s_p}
        {l_q}{s_q}
        {u'_c})
  $
\\[1ex]

$@close@~c~(l',u')$ \\
$\quad~|~\cs[c]=@out1@$ & \note{LocalClose} \\
$\quad\to~\Just (@close@~c~
      \nextStep
        {l'}{s_p}
        {l_q}{s_q}
        {u'})
    $
    \\

$\quad~|~\cs[c]=@in1out1@ ~\wedge~ s_q[c]=@none@_F$ & \note{SharedClose} \\
$\quad\to~\Just (@close@~c~
      \nextStep
        {l'}{s_p}
        {l_q}{\HeapUpdateOne{c}{@closed@_F}{s_q}}
        {u'})
    $ 
\\[1ex]

$@exit@$ & \note{LocalExit} \\
$\quad\to~\Just @exit@$
\\[1ex]

$\_~|~ @otherwise@ $ & \note{Blocked} \\
$\quad\to ~ \Nothing$
\end{tabular}

\caption{Fusion step for a single process of the pair.} 

% Given the state of both processes, compute the instruction this process can perform. This is analogous to statically evaluating the pair of processes. If this process cannot execute, the other process may still be able to.
\label{fig:Fusion:Def:Step}
\end{figure*}




% -------------------------------------
\smallskip
\autoref{fig:Fusion:Def:Step} defines function \ti{tryStep} which schedules a single instruction. This function takes the map of channel types, along with the current label and associated instruction of the first (left) process, and the current label of the other (right) process.

Clause (LocalJump) applies when the left process wants to jump.
In this case, the result instruction simply performs the corresponding jump, leaving the right process where it is.
This clause corresponds to a static version of the rule (Jump) for advancing processes during execution (\autoref{s:Process:Eval}).

Clause (LocalCase) is similar, except there are two $\Next$ labels.

Clause (LocalPush) applies when the left process wants to push to a non-shared output channel.
In this case the push can be performed directly, with no additional coordination required.

Clause (SharedPush) applies when the left process wants to push to a shared channel.
Pushing to a shared channel requires the downstream process to be ready to accept the value at the same time.
We encode this constraint by requiring the static input state of the downstream channel to be $@none@_F$.
When this constraint is satisfied, the result instruction stores the pushed value in the stream buffer variable $(@chan@~c)$ and sets the static input state to $@pending@_F$, which indicates that the new value is now available.
This clause corresponds to a static version of the rule (Push) for advancing the left process, combined with the rule (InjectPush) for injecting the push action into the right process.

Still in \autoref{fig:Fusion:Def:Step}, clause (LocalPull) applies when the left process wants to pull from a local channel, which requires no coordination.

Clause (SharedPullPending) applies when the left process wants to pull from a shared channel that the other process either pulls from or pushes to.
We know that there is already a value in the stream buffer variable, because the state for that channel is $@pending@_F$.
The result instruction copies the value from the stream buffer variable into a variable specific to the left source process.
The corresponding $@have@_F$ channel state in the result label records that the value has been successfully pulled.

Clause (SharedPullClosed) applies when the left process wants to pull from a shared channel that the other process either pulls from or pushes to, and the channel is closed.
The result instruction jumps to the close output label.

Clause (SharedPullInject) applies when the left process wants to pull from a shared channel that both processes pull from, and neither already has a value.
The result instruction is a @pull@ that loads the stream buffer variable, leaving the labels the same and updating the channel state for both processes.
In the next instruction the left process will try to pull again with the updated channel state, and one of the clauses (SharedPullPending) or (SharedPullClosed) will apply.

Clause (LocalDrop) applies when the left process wants to drop the current value that it read from an unshared input channel, which requires no coordination.

Clause (ConnectedDrop) applies when the left process wants to drop the current value that it received from an upstream process. As the value will have been sent via a heap variable instead of a still extant channel, the result instruction just performs a @jump@ while updating the static channel state.

Clauses (SharedDropOne) and (SharedDropBoth) apply when the left process wants to drop from a channel shared by both processes. In (SharedDropOne) the channel states reveal that the other process is still using the value. In this case the result is a @jump@ updating the channel state to note that the left process has dropped. In (SharedDropBoth) the channel states reveal that the other process no longer needs the value. In this case the result is a real @drop@, because we are sure that neither process requires the value any longer.

Clause (LocalClose) applies when the left process wants to close an unshared output channel, which requires no coordination.

Clause (SharedClose) applies when the left process wants to close a shared output channel.
Closing the channel updates the channel state and requires the downstream process to have dropped any previously read values.

Clause (LocalExit) applies when the left process wants to finish execution, which requires no coordination here, but prioritises the other process in the (DeferExit) clauses in the earlier \ti{tryStepPair} definition.

Clause (Blocked) returns @Nothing@ when no other clauses apply, meaning that this process is waiting for the other process to advance.


\smallskip

% -------------------------------------
%!TEX root = ../Main.tex
\begin{figure}

\begin{tabbing}
$\ChanType_2$   \TABDEF \kill

\ti{channels} \> $:$ \> $\Proc \to \Proc \to \MapType{\Chan}{\ChanType_2}$ \\

  \> $=$    \> $\{ c=@MMMMMMM@~$\= \kill
$\ti{channels}~p~q$
  \> $=$    \> $\{ c=@in2@$
            \> $|~c~\in~(@ins@~p~\cap~@ins@~q) \}$ 
            \\

  \> $\cup$ \> $\{ c=@in1@$
            \> $ |~c~\in~(@ins@~p~\cup~@ins@~q)~\wedge~c~\not\in(@outs@~p~\cup~@outs@~q) \}$ \\

  \> $\cup$ \> $\{ c=@in1out1@$
            \> $|~c~\in~(@ins@~p~\cup~@ins@~q)~\wedge~c~\in(@outs@~p~\cup~@outs@~q) \}$ \\

  \> $\cup$ \> $\{ c=@out1@$
            \> $ |~c~\not\in~(@ins@~p~\cup~@ins@~q)~\wedge~c~\in(@outs@~p~\cup~@outs@~q) \}$ 
\end{tabbing}

\newcommand\funClauseDef[3]
{ $\ti{#1}~(#2)$ \> $=$ \> $#3$
}
\newcommand\outlabelsDef[2]
{ \funClauseDef{outlabels}{#1}{\sgl{#2}} 
}

\vspace{1ex}
\begin{tabbing}
$\ti{outlabels}~(@case@~e~(l_t,u_t)~(l_t,u_f))$ \TABSKIP $=$ \TABSKIP \kill
\ti{outlabels} \> $~:$ \> $\Instr \to \sgl{\Label}$ \\
\outlabelsDef{@pull@~c~x~(l,u)~(l',u')}{l,l'}              \\
\outlabelsDef{@drop@~c~(l,u)}{l}                \\
\outlabelsDef{@push@~c~e~(l,u)}{l}              \\
\outlabelsDef{@close@~c~(l,u)}{l}              \\
\outlabelsDef{@case@~e~(l,u)~(l',u')}{l, l'}    \\
\outlabelsDef{@jump@~(l,u)}{l}                  \\
\outlabelsDef{@exit@}{}
\end{tabbing}

\vspace{1ex}
\begin{tabbing}
$\ti{swaplabels}~(@case@~e~((l_1,l_2),u)~((l'_1,l'_2),u'))$ \TABSKIP $=$ \TABSKIP \kill
\ti{swaplabels} \> $~:$ \> $\Instr \to \Instr$ \\
\funClauseDef{swaplabels}
  {@pull@~c~x~((l_1,l_2),u)~((l'_1,l'_2),u')}
  {@pull@~c~x~((l_2,l_1),u)~((l'_2,l'_1),u')}    \\
\funClauseDef{swaplabels}
  {@drop@~c~((l_1,l_2),u)}
  {@drop@~c~((l_2,l_1),u)}      \\
\funClauseDef{swaplabels}
  {@push@~c~e~((l_1,l_2),u)}
  {@push@~c~e~((l_2,l_1),u)}    \\
\funClauseDef{swaplabels}
  {@close@~c~((l_1,l_2),u)}
  {@close@~c~((l_2,l_1),u)}    \\
\funClauseDef{swaplabels}
  {@case@~e~((l_1,l_2),u)~((l'_1,l'_2),u')}
  {@case@~e~((l_2,l_1),u)~((l'_2,l'_1),u')}     \\
\funClauseDef{swaplabels}
  {@jump@~((l_1,l_2),u)}
  {@jump@~((l_2,l_1),u)}  \\
\funClauseDef{swaplabels}
  {@exit@}
  {@exit@}
\end{tabbing}

\caption{Utility functions for fusion}
\label{fig:Fusion:Utils}
\end{figure}



\autoref{fig:Fusion:Utils} contains definitions of some utility functions which we have already mentioned.
Function \ti{channels} computes the $\ChanType_2$ map for a pair of processes.
Function \ti{outlabels} gets the set of output labels for an instruction, which is used when computing the fixpoint of reachable states.
Function \ti{swaplabels} flips the order of the compound labels in an instruction.



% % -----------------------------------------------------------------------------
\subsection{Fusibility}
\label{s:FusionOrder}
When we fuse a pair of processes, we commit to a particular interleaving of instructions from each process.
When we have at least three processes to fuse, the choice of which two to handle first can determine whether this fused result can then be fused with the third process.
Consider the following example, where @alt2@ pulls two elements from its first input stream, then two from its second, before pushing all four to its output.
\begin{haskell}
alternates :: S Nat -> S Nat -> S Nat -> S (Nat, Nat)
alternates sInA sInB sInC
 = let  s1   = alt2 sInA sInB
        s2   = alt2 sInB sInC
        sOut = zip s1 s2
   in   sOut
\end{haskell}
If we fuse the two @alt2@ processes together first, then try to fuse this result process with the downstream @zip@ process, the final fusion transform fails. This happens because the first fusion transform commits to a sequential instruction interleaving where two output values \emph{must} be pushed to stream @s1@ first, before pushing values to @s2@. On the other hand, @zip@ needs to pull a \emph{single} value from each of its inputs alternately.

Dynamically, if we were to execute the first fused result process, and the downstream @zip@ process concurrently, then the execution would deadlock. Statically, when we try to fuse the result process with the downstream @zip@ process the deadlock is discovered and fusion fails. Deadlock happens when neither process can advance to the next instruction, and in the fusion algorithm this manifests as the failure of the $tryStepPair$ function from \autoref{fig:Fusion:Def:StepPair}. The $tryStepPair$ function determines which instruction from either process to execute next, and when execution is deadlocked there are none. Fusibility is an under-approximation for \emph{deadlock freedom} of the network.

% On the upside, fusion failure is easy to detect. It is also easy to provide a report to the client programmer that describes why two particular processes could not be fused.

% The report is phrased in terms of the process definitions visible to the client programmer, instead of partially fused intermediate code. The joint labels used in the fusion algorithm represent which states each of the original processes would be in during a concurrent execution, and we provide the corresponding instructions as well as the abstract states of all the input channels.

% This reporting ability is \emph{significantly better} than that of prior fusion systems such as Repa~\cite{lippmeier2012:guiding}, as well as the co-recursive stream fusion of \cite{coutts2007stream}, and many other systems based on general purpose program transformations. In such systems it is usually not clear whether the fusion transformation even succeeded, and debugging why it might not have succeeded involves spelunking\footnote{def. spelunking: Exploration of caves, especially as a hobby. Usually not a science.} through many pages (sometimes hundreds of pages) of compiler intermediate representations.

In practice, the likelihood of fusion succeeding depends on the particular dataflow network. For fusion of pipelines of standard combinators such as @map@, @fold@, @filter@, @scan@ and so on, fusion always succeeds. The process implementations of each of these combinators only pull one element at a time from their source streams, before pushing the result to the output stream, so there is no possibility of deadlock. Deadlock can only happen when multiple streams fan-in to a process with multiple inputs, such as with @merge@.

When the dataflow network has a single output stream then we use the method of starting from the process closest to the output stream, walking towards the input streams, and fusing in successive processes as they occur. This allows the interleaving of the intermediate fused process to be dominated by the consumers, rather than producers, as consumers are more likely to have multiple input channels which need to be synchronised. In the worst case the fall back approach is to try all possible orderings of processes to fuse.

% The main fusion algorithm works on pairs of processes.
% When there are more than two processes, there are multiple orders in which the pairs of processes can be fused.
% The order in which pairs of processes are fused does not affect the output values, but it does affect the access pattern: the order in which outputs are produced and inputs read.
% Importantly, the access pattern also affects whether fusion succeeds or fails to produce a process.
% In other words, while evaluating multiple processes is non-deterministic, the act of fusing two processes \emph{commits} to a particular deterministic interleaving of the two processes.
% 
% When we fuse a pair of processes we commit to a particular interleaving of instructions from each process.
% When we have at least three processes to fuse, the choice of which two to handle first can determine whether this fused result can then be fused with the third process.
% 
% Consider the process network @append2zip@ in \autoref{l:bench:append2zip}.
% We have three input streams of integers: @as@, @bs@ and @cs@.
% We append @as@ and @cs@ together, as well as appending @bs@ and @cs@ together.
% Then we zip the two appended streams together.
% This will give us first elements from @as@ paired with those of @bs@, followed by elements from @cs@ paired with themselves.
% 
% \begin{lstlisting}[float,label=l:bench:append2zip,caption=append2zip; S stands for Stream]
% append2zip :: S Int -> S Int -> S Int -> Network m (S (Int, Int))
% append2zip as bs cs = do
%   ac   <- append as cs
%   bc   <- append bs cs
%   acbc <- zip    ac bc
%   return acbc
% \end{lstlisting}
% 
% In our current system, if we fuse the two @append@ processes together first, then try to fuse this result process with the downstream @zip@ process, then this final fusion transform will fail.
% This happens because the first fusion transform commits to a sequential instruction interleaving that first copies all of @as@ to @ac@, then copies all of @bs@ to @bc@, before copying @cs@ to both.
% The @zip@ requires values from @ac@ and @bc@ to be interleaved.
% But by committing to the interleaving that copies \emph{all} @ac@, we can no longer execute the @zip@ without introducing an unbounded buffer.
% 
% Dynamically, if we were to try to execute the result process and the downstream @zip@ process concurrently, then the execution would deadlock.
% Statically, when we try to fuse the result process with the downstream @zip@ process then the deadlock is discovered and fusion fails.
% Deadlock happens when neither process can advance to the next instruction, and in the fusion algorithm this will manifest as the failure of the $tryStepPair$ function from \autoref{fig:Fusion:Def:StepPair}.
% The $tryStepPair$ function determines which of the instructions from either process can be executed next, and when execution is deadlocked there are none.
% 
% On the upside, fusion failure is easy to detect.
% It is also easy to provide a report to the client programmer that describes why two particular processes could not be fused.
% The report is phrased in terms of the process definitions visible to the client programmer, instead of partially fused intermediate code.
% The joint labels used in the fusion algorithm represent which states each of the original processes would be in during a concurrent execution, and we provide the corresponding instructions as well as the abstract states of all the input channels.
% This reporting ability is \emph{significantly better} than that of prior fusion systems such as Repa~\cite{lippmeier2012guiding}, as well as the co-recursive stream fusion of \cite{coutts2007stream}, and many other systems based on general purpose program transformations.
% In such systems it is usually not clear whether the fusion transformation even succeeded, and debugging why it might not have succeeded involves spelunking\footnote{def. spelunking: Exploration of caves, especially as a hobby. Usually not a science.} through many pages (sometimes hundreds of pages) of compiler intermediate representations.
% 
% In practice, the likelihood of fusion suceeding depends on the particular dataflow network being used, as well as the form of the processes in that network.
% For fusion of pipelines of standard combinators such as @map@, @fold@, @filter@, @scan@ and so on, fusion always succeeds.
% The process implementations of each of these combinators only pull one element at a time from their source streams, before pushing the result to the output stream, so there is no possiblity of deadlock.
% Deadlock can only happen when multiple streams fan-in to a process with multiple inputs, such as with @merge@.
% When the dataflow network has a single output stream then we use the method of starting from the process closest to the output stream, walking to towards the input streams, and fusing in successive processes as they occur.
% This allows the interleaving of the intermediate fused process to be dominated by the consumers, rather than producers, as consumers are more likely to have multiple input channels which need to be synchronized.
% In the worst case the fallback approach is to try all possible orderings of processes to fuse, assuming the client programmer is willing to wait for the search to complete. 




\section{Synchronising pulling by dropping}
\label{s:Drop:in:synchrony}

The @drop@ instruction exists to synchronise two consumers of the same input, so that both processes pull the same value at roughly the same time.
When fusing two consumers, the fusion algorithm uses drops when coordinating the processes to ensure that one consumer cannot start processing the next element until the other has finished processing the current element.
Drop instructions are not necessary for correctness, or for ensuring boundedness of buffers, but they improve locality in the fused process.

% The following example, @map2par@, has one input stream which is transformed twice.
% The `@par@' part of the name is because the two @map@ processes are in parallel as opposed to series; both processes consume the same input.
% This example is very similar to the @map2@ from earlier.
% The processes are the same, except for the input to the second @map@, and there is an extra sink.

\begin{figure}
\center
\begin{dot2tex}[dot]
digraph G {
  node [shape="none"];
  stock [texlbl="\Hs/stock/"];

  stock -> pot_tps;
    lblstyle="right";
    pot_tps [texlbl="\Hs/map/"];
    pot_cor [texlbl="\Hs/correlation/"];
    pot_reg [texlbl="\Hs/regression/"];
    pot_tps -> pot_cor;
    pot_tps -> pot_reg;
}
\end{dot2tex}
\caption{Dependency graph for \Hs/priceOverTime/ example}
\label{figs/procs/drop/priceOverTime}
\end{figure}

Recall the @priceOverTime@ example, which computes the correlation and regression of an input stream.
The dependency graph for @priceOverTime@ is shown in \cref{figs/procs/drop/priceOverTime}.

\begin{figure}
\center
\begin{sequencediagram}
\newthreadGAP{map}{@map@}{0.2}
\newthreadGAP{reg}{@regression@}{1.5}
\newthreadGAP{regout}{@reg out@}{1.5}
\newthreadGAP{cor}{@correlation@}{1.5}
\newthreadGAP{corout}{@cor out@}{1.5}

\parallelseq{
  \mess{map}{push $A$}{reg}
}{
  \mess{map}{}{cor}
}

\mess{reg}{pull $A$}{reg}
\mess{reg}{drop [update $A$]}{reg}
\mess{cor}{pull $A$}{cor}
\mess{cor}{drop [update $A$]}{cor}


\parallelseq{
  \mess{map}{push $B$}{reg}
}{
  \mess{map}{}{cor}
}

\mess{reg}{pull $B$}{reg}
\mess{reg}{drop [update $B$]}{reg}
\mess{cor}{pull $B$}{cor}
\mess{cor}{drop [update $B$]}{cor}

\parallelseq{
  \mess{map}{close}{reg}
}{
  \mess{map}{}{cor}
}

\mess{reg}{push $R$}{regout}
\mess{reg}{close}{regout}
\mess{cor}{push $C$}{corout}
\mess{cor}{close}{corout}


\end{sequencediagram}
\caption{sequence diagram for a possible linearised execution of the correlation and regression of priceOverTime, showing drop synchronisation. }
\label{figs/swim/drop/priceOverTime}
\end{figure}

\Cref{figs/swim/drop/priceOverTime} shows an example execution of the @correlation@ and @regression@ processes from @priceOverTime@, with an input stream containing two elements.
Execution is displayed as a sequence diagram.
Each process and output stream is represented as a vertical line which communicates with other processes by messages, represented by arrows.
The names of each stream and process are written above the line, and time flows downwards.
To highlight the synchronisation between @regression@ and @correlation@ processes, we use placeholder values such as $A$ and $B$ instead of actual stream values, and show only a subset of the whole execution, omitting the @stock@ input stream and the internal messages of the @map@ process.

In the definition of the @fold@ process template, the @drop@ instruction also updates the fold state with the most recently pulled value.
We use the shorthand (drop [update $A$]) to signify that the process updates its fold state with the pulled value $A$ after dropping the element.

Execution starts with the @map@ process pushing the value $A$ to both @regression@ and @correlation@ processes.
In the execution semantics from \cref{s:Process:Eval}, this push changes the input state of each recipient process from @none@ to @pending@, to signify that there is a value available to pull.
At this point, the @map@ process cannot push again until both consumers have pulled and dropped the $A$ value.
Next, the @regression@ process pulls the value $A$, temporarily changing its input state to @have@, before using and dropping the value, changing the input state back to @none@.
The @correlation@ process now performs the same.
After both consumers have dropped the input value, the @map@ process is able to push the next value, $B$, which the consumers operate on similarly.
Finally, the @map@ process sends close messages to both consumers, which both push the results of the folds to their corresponding output streams before closing them.
In this execution, both consumer processes transform the same element at roughly the same time, because the next element is only available once both have dropped, thereby agreeing to accept the next element.


\begin{figure}
\center
\begin{sequencediagram}
\newthreadGAP{map}{@map@}{0.2}
\newthreadGAP{reg}{@regression@}{1.5}
\newthreadGAP{regout}{@reg out@}{1.5}
\newthreadGAP{cor}{@correlation@}{1.5}
\newthreadGAP{corout}{@cor out@}{1.5}

\parallelseq{
  \mess{map}{push $A$}{reg}
}{
  \mess{map}{}{cor}
}

\mess{reg}{pull $A$}{reg}
\mess{cor}{pull $A$}{cor}

\parallelseq{
  \mess{map}{push $B$}{reg}
}{
  \mess{map}{}{cor}
}

\mess{reg}{jump [update $A$]}{reg}
\mess{reg}{pull $B$}{reg}
\mess{reg}{jump [update $B$]}{reg}

\mess{cor}{jump [update $A$]}{cor}
\mess{cor}{pull $B$}{cor}

\parallelseq{
  \mess{map}{close}{reg}
}{
  \mess{map}{}{cor}
}

\mess{reg}{push $R$}{regout}
\mess{reg}{close}{regout}

\mess{cor}{jump [update $B$]}{cor}
\mess{cor}{push $C$}{corout}
\mess{cor}{close}{corout}

\end{sequencediagram}
\caption{sequence diagram for a possible linearised execution of the correlation and regression of priceOverTime, using a hypothetical semantics without drop synchronisation. }
\label{figs/swim/drop/priceOverTime-nosync}
\end{figure}



\Cref{figs/swim/drop/priceOverTime-nosync} shows a hypothetical execution which may occur if our process network semantics did not use drop instructions to synchronise between all consumers.
In this execution, we replace the drop instructions with a jump instruction, using the shorthand (jump [update $A$]) to signify updating the fold state.
As before, execution starts with the @map@ process pushing the $A$ value, which both consumers pull.
Without drop synchronisation, the single-element buffer is cleared as soon as the process pulls, allowing the @map@ process to push another element to both consumers.
Now, the @regression@ process executes and updates the fold state with both values $A$ and $B$.
The @regression@ process has consumed both elements before the second consumer, @correlation@, has even looked at the first.
This is not a problem for concurrent execution: the execution results in the same value, and the buffer is still bounded, containing at most one element.
However, when we fuse this network into a single process, we commit to a particular interleaving of execution of the processes in the network.
When performing fusion, we would prefer to use the previous interleaving with drop synchronisation to this unsynchronised interleaving, because the process with the unsynchronised interleaving would need to keep track of two consecutive elements at the same time.
Keeping both elements means the process requires more live variables, which makes it less likely that both elements will fit in the available registers or cache when we eventually convert the fused process to machine code.

% \TODO{Other options:} what if the producer could only push when all consumers are trying to pull?
% This is more restrictive and causes deadlock in @zip x y / zip y x@ case.
% 
% \TODO{Size:} show fused process with and without drop synchronisation. Drop synchronisation makes it smaller.
% 
% \TODO{SIMD?:} the bit above says we don't want to keep multiple consecutive elements at the same time.
% Maybe we do want to, in order to use SIMD instructions.
% In fact, for \cref{figs/swim/map2par-no-sync} if we had vector instructions to compute @f@ and @g@ two elements at a time, that interleaving is exactly what we want.
% There are many other interleavings though, and we are not guaranteed to get this one.
% We do not want to rely on chance to find a SIMD interleaving.
% Future work may involve looking for the right interleavings to exploit SIMD instructions.

% By synchronising the two processes together, when we fuse we will only have one copy of the code that pulls each element.
% Because @P@ can only start pulling again by the time @Q@ has dropped, this means @P@ and @Q@ must both be trying to pull at the same time, which means we can reuse the same instructions generated from the previous time they both pulled.
% The example @PQ_drop@ shows the fused process with drop instructions.
%                                     ~~~~~[BL: highlight?]~~~
% Note that there is only one copy of each input process' code.
% On the other hand, @PQ_no_drop@ shows the fused process without drop instructions.
%                           ~~~~~~[BL: highlight?]~~~~~~~
% In @PQ_no_drop@ there are two copies of pushing to @CP@, though the main loop only executes one per iteration: pushing the current element to @CP@, and the previous element to @CQ@.
%                                                                                                     ~~~~[BL: examples]~~~~
% As the processes get larger, and more processes are fused together, the issue of duplicating code becomes more serious.
% There are two parts to this: first, we need to hold the entire process in memory to generate its code.
% Secondly, as the generated assembly code gets larger, it is less likely to fit into the processor's cache.
% Smaller code is generally better for performance.
% Having two copies of the push to @CP@ means that any consumers of @CP@ must in turn have their code duplicated, with the pull instructions from @CP@ copied into both sites of the pushes.
% 
% \TODO{diagrams}
% \begin{lstlisting}
% PQ_drop = process
%   P1Q1: c_buf <- pull C
%         c_p    = c_buf
%         c_q    = c_buf
%         push CP c_p
%         push CQ c_q
%         drop C
%         jump P1Q1
% 
% PQ_no_drop = process
%   P1Q1:  c_buf <- pull C
%          c_p    = c_buf
%          c_q    = c_buf
%          push CP c_p
%   P2Q2:  c_buf <- pull c
%          c_p    = c_buf
%          push CP c_p
%          push CQ c_q
%          c_q    = c_buf
%          jump P2Q2
% \end{lstlisting}
% 
% Without drops, the result program is still correct, but one process can overtake another by one element.
% Overtaking leads to larger processes because there end up being two copies of the processing code.
% 
% 
% 



\section{Transforming process networks}
\label{s:Optimisation}

The fusion algorithm described in \autoref{s:Fusion} operates on a pair of input processes.
For process networks which contain more than two processes, we repeatedly fuse pairs of processes in the network together until only one process remains.

When fusing a pair of processes, the fused process tends to have more states than each input process individually, because the fused process has to do the work of both input processes.
In general, the larger the input processes, the larger the fused process will be, and when we have many processes to fuse, the result will get progressively larger as we fuse more processes in.
If the fused process becomes too large such that the process does not fit in memory, then fusing in the next process will take longer, and code generation will take longer.
When repeatedly fusing the pairs of processes in a network, we perform some simplifications between each fusion step, to remove unnecessary states and simplify the input for the next fusion step.

% The fusion algorithm also introduces some spurious states.
% It is designed for simplicity of the fusion algorithm, at the expense of simplicity of the output.
% If the input process has a couple more states than necessary, this can turn into several unnecessary states in the fused process.
% When this fused process is used as the input to another fusion step, the unnecessary states compound.
% What started as a couple can become dozens, then hundreds, then thousands.
% As with compound interest on a loan, it is best to pay back early and often.


\subsection{Fusing a network}
\label{ss:Fusing:a:network}

\begin{figure}
\center
\begin{dot2tex}[dot]
digraph G {
  node [shape="none"];
  stock; index;

  stock -> pom_join;
  index -> pom_join;
  stock -> pot_tps;

  graph [style="rounded corners,filled"];

  subgraph cluster_z {
    graph [bgcolor="0.0 0.0 0.9"];
    subgraph cluster_priceAgainstMarket {
      lblstyle="right";
      label="priceOverMarket";
      graph [bgcolor="0.0 0.0 0.8"];
      pom_join [label="join"];
      subgraph cluster_x {
        label="";
        graph [bgcolor="0.0 0.0 0.7"];
        subgraph cluster_x {
          graph [bgcolor="0.0 0.0 0.6"];
          pom_price [label="map"];
          pom_cor [label="correlation"];
        };
      pom_reg [label="regression"];
      }
      pom_join -> pom_price;
      pom_price -> pom_cor;
      pom_price -> pom_reg;
    };

    subgraph cluster_priceOverTime  {
      lblstyle="left";
      label="priceOverTime";
      graph [bgcolor="0.0 0.0 0.8"];
      subgraph cluster_x1 {
        label="";
        graph [bgcolor="0.0 0.0 0.7"];
        pot_tps [label="map"];
        pot_cor [label="correlation"];
      }
      pot_reg [label="regression"];
      pot_tps -> pot_cor;
      pot_tps -> pot_reg;
    };

  }
}
\end{dot2tex}
\caption{Pairwise fusion ordering of the priceAnalyses network.}
\label{figs/procs/priceAnalyses-fusing-whole}
\end{figure}

As we shall see, when we fuse pairs of processes in a network, the order in which we fuse pairs can determine whether fusion succeeds.
Rather than trying all possible orders, of which there are many, we use a bottom-up heuristic to choose a fusion order.
% This heuristic is not guaranteed to choose the, but it .
\autoref{figs/procs/priceAnalyses-fusing-whole} shows the heuristically chosen fusion order for the @priceAnalyses@ example.
The processes are nested inside boxes; each box denotes the result of fusing a pair of processes, and inner-most boxes are to be fused first.
Each box is shaded to denote its nesting, and the more deeply nested a box is, the darker its shade.
In @priceOverTime@, we start by fusing the @correlation@ process with its producer, @map@; we then fuse the resulting process with the @regression@ process.
In @priceOverMarket@, we also start by fusing the @correlation@ process with @map@, then adding @regression@, and fusing in the @join@ process.
Finally, we fuse the result process for @priceOverTime@ with the result process for @priceOverMarket@.

% There are many orders in which we could fuse the pairs.
% For $n$ processes, there are $n!$ different permutations and $(n-1)!$ different ways to nest the parentheses.
% For $3$ processes there are $12$ orders; for $4$ processes there are $144$; for $5$ processes there are $2,880$.
% It does not take many processes for there to become too many orders to try.
% For $10$ processes, there are more than a trillion possibilities.
% Even if we could fuse one process in a single instruction, at 3GHz it would take forty minutes to try all orders.
% If the process network is fundamentally unfusable, it is unacceptable to force the user to wait forty minutes before telling them we cannot fuse it.

% We cannot try all the orders; we need some way to choose the order.
% As we shall see in \autoref{s:extraction:future}, it is impossible to choose the right order by looking at the dependency graph alone; the correct order depends upon the implementation of each process.
% As future work we shall propose a fusion algorithm which is commutative and associative, which means we can fuse processes in any order.
% We shall start by explaining a heuristic which does not always choose the right order, but works for the benchmarks in \REFTODO{benchmarks}.
% \TODO{Maybe the explanation of why it has to be a heuristic should come first, but it makes it unappealing to start with an example where the heuristic doesn't work. If I explain the future work, it seems like there's no reason to explain the heuristic. Reorganise later.}

To demonstrate how fusion order can affect whether fusion succeeds, consider the following list program, which takes three input lists, appends them, and zips the appended lists together.

\begin{lstlisting}
append2zip :: [a] -> [a] -> [a] -> [(a,a)]
append2zip a b c =
  let ba = b ++ a
      bc = b ++ c
      z  = zip ba bc
  in  z
\end{lstlisting}

We use the more convenient syntax for list programs rather than the process network syntax introduced earlier, but in the discussion we interpret this program as a process network.
In the process network interpretation, each list combinator corresponds to a process, and each list corresponds to a stream.
The dependency graph for the corresponding process network is shown in \autoref{figs/specconstr/append2zip}.


\FigurePdf{figs/specconstr/append2zip}{Dependency graph for append2zip}{Dependency graph for append2zip}

This example appends the input streams, then pairs together the elements in both appended streams.
The result of the two append processes, @ba@ and @bc@, both contain the elements from @b@ stream, followed by the elements of the second append argument; stream @a@ or stream @c@ respectively.
These two streams, @ba@ and @bc@, when paired together, will result in each element of the @b@ stream paired with itself, followed by elements of the two other streams paired together.


\autoref{figs/swim/append2zip} shows an example execution of @append2zip@, displayed as a sequence diagram.
In this diagram, we omit the @drop@ and @pull@ internal messages for all processes, and focus instead on the communication between processes.
Each input stream pushes its elements to its consumers; input stream @a@ has elements $[1, 2]$, input stream @b@ has elements $[3, 4]$, and input stream @c@ has elements $[5, 6]$.
Input stream @b@ has multiple consumers, so when it pushes elements, it pushes to both its consumers at the same time.


\begin{figure}
\center
\begin{sequencediagram}
\newthreadGAP{a}{@a@}{0.0}
\newthreadGAP{b}{@b@}{0.7}
\newthreadGAP{c}{@c@}{0.7}
\newthreadGAP{appba}{@b++a@}{0.7}
\newthreadGAP{appbc}{@b++c@}{0.7}
\newthreadGAP{zip}{@zip@}{0.7}
\newthreadGAP{z}{@z@}{0.7}

\messmessx{b}{push 3}{appba}{appbc}

\mess{appba}{push 3}{zip}
\mess{appbc}{push 3}{zip}
\mess{zip}{push (3,3)}{z}

\messmessx{b}{push 4}{appba}{appbc}

\mess{appba}{push 4}{zip}
\mess{appbc}{push 4}{zip}
\mess{zip}{push (4,4)}{z}

\messmessx{b}{close}{appba}{appbc}

\addtocounter{seqlevel}{3}

\mess{a}{push 1}{appba}
\mess{appba}{push 1}{zip}

\mess{c}{push 5}{appbc}
\mess{appbc}{push 5}{zip}

\mess{zip}{push (1,5)}{z}

\mess{a}{push 2}{appba}
\mess{appba}{push 2}{zip}

\mess{c}{push 6}{appbc}
\mess{appbc}{push 6}{zip}

\mess{zip}{push (2,6)}{z}

\addtocounter{seqlevel}{3}

\mess{a}{close}{appba}
\mess{appba}{close}{zip}

\mess{c}{close}{appbc}
\mess{appbc}{close}{zip}

\mess{zip}{close}{z}

\end{sequencediagram}
\caption[Concurrent sequence diagram for two `append2zip']{sequence diagram for execution of @append2zip@. }
\label{figs/swim/append2zip}
\end{figure}


The execution has three sections.
In the first section, all the values from the @b@ stream are pushed to both append processes, then paired together.
In the second section, execution alternates between the other streams, @a@ and @c@, with one value from each.
In the third section, the streams are closed, which propagates down to the consumers.
The bottom-most consumer process, @zip@, executes by alternately pulling from each of the append processes.
The order in which a process pulls from its inputs is called its \emph{access pattern}.
Each append process can only push when the @zip@ process' buffer for that channel is empty: append must wait for @zip@ to read the most recent element before pushing a new element.
When each append process is waiting, its producer---the input stream---must also wait before pushing the next element.
This waiting propagates the @zip@ process' access pattern upwards through the append processes and to the input streams.

This example contains three processes.
We could perform fusion in twelve different orders.
Of these twelve orders, there are two main categories, distinguished by whether we start by fusing the append processes with each other, or start by fusing the @zip@ process with one of the append processes.
% The first category, we fuse the two append processes together, then fuse with the @zip@ process.
% In the second category, we fuse the @zip@ process with one of the append processes, then fuse with the other append process.

If we fuse the two append processes together first, we interleave their instructions without considering the access pattern of the @zip@ process.
There are many ways to interleave the two processes; one possibility is that the fused process reads all of the shared prefix from stream @b@, then all of stream @a@, then all of stream @c@.
For the shared prefix, this interleaving alternates between pushing to streams @ba@ and @bc@.
After the shared prefix, this interleaving pushes the rest of the stream @ba@, then pushes the rest of the stream @bc@.
When we try to fuse the @zip@ process with the fused append processes with this interleaving, we get stuck.
The @zip@ process needs to alternate between its inputs, which works for the shared prefix, but not for the remainder.
By fusing the two append processes together first, we risk choosing an interleaving that works for the two append processes on their own, but does not take into account the access pattern of the @zip@ process.

Fusion does succeed if we fuse the @zip@ process with one of the append processes first, then fuse with the other append process.
The consumer, @zip@, must dictate the order in which the append processes push; fusing the @zip@ process first gives it this control.
We start from the consumer and fuse them upwards with their producers, because this allows the consumer to impose its access pattern on the producers.

To fuse an arbitrary process network, we consider a restricted view of the dependency graph, ignoring the overall inputs and outputs of the network.
We start at the bottom of the dependency graph, finding the \emph{sink} processes, or those with no output edges.
These sink processes are the bottom-most consumers which, like @zip@ in our @append2zip@ example, dictate the access pattern on their inputs.
For each sink process, we find its parents and fuse the sink process with its parents.
When the sink process has multiple parents, we need to choose which parent to fuse with first.
In the @append2zip@ example, we can fuse the @zip@ process with its append parents in any order.
In general, one parent may consume the other parent's output; in which case we first fuse with the consumer parent.
This order allows the consuming parent to impose its access pattern upon the producing parent.
We repeatedly fuse each sink process with its closest parent until there are no more parents.

After fusing each sink process with all its ancestors, there may remain multiple processes.
This only occurs if the remaining processes do not share ancestors.
The remaining processes also cannot share descendents, since if they had descendents they would not be sink.
This means the processes are completely separate and could be executed separately, in any order or even in parallel.
Having unconnected processes in the same process network is a degenerate case, as it could be represented as multiple process networks.
We err on the side of caution, telling the programmer about anything even slightly unexpected.
Rather than making the decision of which order to execute them in, we display a compile-time error and make the programmer separate the network.

The fusion algorithm for pairs of processes fails and does not produce a result process when two processes have conflicting access patterns on their shared inputs.
As the access patterns are determined statically, apparent conflicts may in fact never occur at runtime; we must approximate.
If at any point we encounter a pair of processes which we cannot fuse together, we display a compile-time error telling the programmer that the network cannot be fused.

\subsubsection{Future work}
\label{s:extraction:future}
\TODO{This likely belongs elsewhere}

Regardless of the order the processes are fused in, if fusion succeeds, the result process has the same meaning.
This makes the heuristic described above \emph{sound}, in that it will never produce a wrong program.

The corrolary to soundness is \emph{completeness}, which is that it will always produce a right program.
That is, if there exists any order in which fusion succeeds, then the order described above also succeeds.
This heuristic is not complete, but it does works for all the examples in \REFTODO{benchmarks}.
% For large process networks, there are many orders we could fuse the processes.
% There are factorially many permutations of the set of processes, and on top of that, the fusion operation can be nested arbitrarily.
% This factorial number of possibilities is because the fusion algorithm is non-associative and non-commutative.
The order described above is a heuristic to avoid trying all possible orders, but it does not always work.

The following example, @append3@, is very similar to @append2zip@ except it returns three output streams constructed by appending the three input streams in different orders.

\begin{lstlisting}
append3 a b c =
  ab <- append a b
  ac <- append a c
  bc <- append b c
  return (ab, ac, bc)
\end{lstlisting}


\begin{figure}
\begin{minipage}[t]{0.5\textwidth}
\center
\begin{dot2tex}[dot]
digraph G {
  node [shape="none"];
  b; a; c;
  app1 [label="append"];
  app2 [label="append"];
  app3 [label="append"];
  a -> app1; a -> app2;
  b -> app1; b -> app3;
  c -> app2; c -> app3;
  app1 -> ab;
  app2 -> ac;
  app3 -> bc;
}
\end{dot2tex}
\end{minipage}
\begin{minipage}[t]{0.5\textwidth}
\center
\begin{dot2tex}[dot]
digraph G {
  node [shape="none"];
  b; a; c;
  app1 [label="zip"];
  app2 [label="zip"];
  app3 [label="zip"];
  a -> app1; a -> app2;
  b -> app1; b -> app3;
  c -> app2; c -> app3;
  app1 -> ab;
  app2 -> ac;
  app3 -> bc;
}
\end{dot2tex}
\end{minipage}
\caption[Process network for `append3']{process networks for @append3@ and @zip3@.}
\label{figs/procs/append3-zip3}
\end{figure}


This process network can be executed with no buffering.
First, read all of the @a@ input stream, then read the @b@ stream, then read the @c@ stream.
There is no single consumer in this example which imposes its access pattern on its producers, so our heuristic fails.
This process network can only be fused if the process that produces @ab@ and the process that produces @bc@ are first fused together.
If we fused @ab@ and @ac@ together first, the fusion algorithm would make an arbitrary decision of whether to read the @b@ stream before, after, or interleaved with the @c@ stream.
The heuristic described will not necessarily choose the right order.

Looking at the dependency graph alone, it is impossible to tell which is the right order for fusion.
If we take the @append3@ example and replace the append processes with @zip@ processes, the dependency graph remains the same, but either fusion order would work.
\autoref{figs/procs/append3-zip3} shows the process networks of @append3@ and @append3@ replaced with @zip@ processes.



We propose to solve this in future work \REFTODO{future of fusion} by modifying the fusion algorithm to be commutative and associative.
These properties would allow us to apply fusion in any order, knowing that all orders produce the same result.

The fusion algorithm is not commutative because when two processes are trying to execute instructions which could occur in either order, the algorithm must choose only one instruction.
Fusion commits too early to a particular interleaving, when there are multiple interleavings that would work.
By explicitly introducing non-determinism in the fused process, we can represent all possible interleavings, and do not have to commit to one too early.
We are moving the non-determinism from the order in which fusion occurs, and reifying it in the process itself.

Reifying the non-determinism in the processes will mean that all fusion orders produce the same process at the end.
Different orders will not affect the result, or whether things fuse.
Different orders do affect the size of the intermediate process, before all processes are fused together.
Fusing two unrelated processes which read from different streams introduces a lot of non-determinism: at each step of the fused process, either of the original processes can take a step.
The two processes do not constrain each other and the result process will have a lot of states.
Fusing related processes, for example a producer and a consumer, introduce less non-determinism because there are points when only one of the processes can run.
When the consumer is waiting for a value, only the producer can run.
Generally, fusing related processes will produce a smaller process than fusing unrelated processes.
The size of the overall result for the entire network is not any different, but the intermediate process will be smaller.
Larger intermediate programs generally take longer to compile, so some heuristic order which fuses related processes is likely to be useful, even if the order does not affect the result.


% \subsection{Fusing a pair of processes}
% Fusing a pair of processes is slightly different because variables are passed as function arguments, instead of a global heap.
% It also needs to take into account variable bindings per label, because of the locally-bound variables.
% For a pair of labels, the variables is the union of variables in the original processes, as well as any \emph{new} channel buffers which need to be bound.
% These are the ones which have an input state of \emph{have}.




% -----------------------------------------------------------------------------


\section{Proofs}
\label{s:Proofs}

Our fusion system is formalised in Coq\footnote{\url{https://github.com/amosr/papers/tree/master/2017mergingmerges/proof}}, and we have proved soundness of \ti{fusePair}: if the fused result process produces a particular sequence of values on its output channels then the two source processes may also produce that same sequence. Note that due to non-determinism of process execution the converse is not true in practice: just because the two concurrent processes can produce a particular output sequence does not mean the fused result process will as well --- the fused result process uses only one of the many possible orders.

% Note that the converse is not necessarily true: just because two processes can evaluate to a particular output does not mean the fused program will evaluate to that. This is because, as explained in~\autoref{s:EvaluationOrder}, evaluation of a process network is non-deterministic, and fusion commits to a particular evaluation order.

\begin{coq}
Theorem Soundness (P1 : Program L1 C V1) (P2  : Program L2 C V2)
                  (ss : Streams)         (h   : Heap)
                  (l1 : L1)              (is1 : InputStates)
                  (l2 : L2)              (is2 : InputStates)
  :  EvalBs (fuse P1 P2) ss h (LX l1 l2 is1 is2)
  -> EvalOriginal Var1 P1 P2 is1 ss h l1
  /\ EvalOriginal Var2 P2 P1 is2 ss h l2.
\end{coq}

The @Soundness@ theorem uses @EvalBs@ to evaluate the fused program, and @EvalOriginal@ ensures that the original program evaluates with that program's subset of the result heap, using @Var1@ and @Var2@ to extract the variables.
The @Streams@ type corresponds to the channel value map used to accumulate stream elements while feeding a process network (\autoref{fig:Process:Eval:Feed}), and the @Heap@ type corresponds to the value store used while advancing a single process (\autoref{fig:Process:Eval:Shake}).

% Care must be taken to remove stream values that the other process has pulled but this one has not yet.
%%% AR: Really would like to say something about this but no room to explain properly
% For shared inputs when one program has pulled, the other program must be evaluated with the other value removed from the end of the stream.

To aid mechanisation, the Coq formalisation has some small differences from the system presented earlier in this thesis.
Firstly, the Coq formalisation uses a separate @update@ instruction to modify variables in the local heap, rather than attaching heap updates to the \Next~ label of every instruction.
Performing this desugaring makes the low level lemmas easier to prove, but we find attaching the updates to each instruction makes for an easier exposition.
This causes the fusion definition to be slightly more complicated, as two output instructions must be emitted when performing a push or pull followed by an update.
This is a fairly minor difference.

Secondly, the formalisation only implements sequential evaluation for a single process, rather than non-deterministic evaluation for whole process networks.
Instead, we sequentially evaluate each source processes independently, and compare the output values to the ones produced by sequential evaluation of the fused result process.
This is sufficient for our purposes because we are mainly interested in the value correctness of the fused program, rather than making a statement about the possible execution orders of the source processes when run concurrently.

Like the earlier presentation, each program has a mapping from labels to instructions.
We also associate each label with a precondition, which is expressed as a predicate of the evaluation state.
The precondition for the initial label must be true for the initial evaluation state.
Whenever the program takes an evaluation step, assuming the precondition for the original label was satisfied at the start of the step, the precondition for the result label must be satisfied by the updated evaluation state.
With these two conditions satisfied, we can show that all evaluations respect the preconditions by performing induction over the evaluation relation.

% To prove soundness of fusion, we assert the invariant that defines how evaluation of the fused program relates to evaluation of the two original programs. 
% For the fused process, the precondition is that, when the fused process evaluates to a particular evaluation state, then each original process must
% \TODO{example evaluation, show how evaluation relation works with @pending@ values}
% 
% \TODO{also mention that formalisation is infinite streams, rather than finite}
