\chapter{Processes and networks}
\label{chapter:process:processes}

\section{Gold panning with processes}
In \autoref{taxonomy} we introduced the @priceAnalyses@ example, which performs statistical analyses over the daily prices of a particular corporate stock and market index.
\autoref{figs/procs/priceAnalyses-again} shows the dependency graph for @priceAnalyses@ with the two input streams, @index@ and @stock@, at the top of the graph.

\begin{figure}
\center
\begin{dot2tex}[dot]
digraph G {
  node [shape="none"];
  stock; index;

  stock -> pom_join;
  index -> pom_join;
  stock -> pot_tps;

  graph [style="rounded corners"];

  subgraph cluster_priceOverTime  {
    lblstyle="right";
    label="priceOverTime";
    pot_tps [label="map"];
    pot_cor [label="correlation"];
    pot_reg [label="regression"];
    pot_tps -> pot_cor;
    pot_tps -> pot_reg;
  };

  subgraph cluster_priceAgainstMarket {
    lblstyle="left";
    label="priceOverMarket";
    pom_join [label="join"];
    pom_price [label="map"];
    pom_cor [label="correlation"];
    pom_reg [label="regression"];
    pom_join -> pom_price;
    pom_price -> pom_cor;
    pom_price -> pom_reg;
  };
}
\end{dot2tex}
\caption{Dependency graph for priceAnalyses example.}
\label{figs/procs/priceAnalyses-again}
\end{figure}

Recall that we cannot execute this example using the pull streams from \autoref{taxonomy/pull}, because the @stock@ input stream is used twice, and pull streams only support a single consumer.
Similarly, we cannot execute this example using the push streams from \autoref{taxonomy/push}, because the @join@ combinator has two inputs, and push streams only support a single producer except for non-deterministic merge.
Rather than just using pull streams, or just using push streams, we wish to be able to perform both pulling and pushing at the same time, in a way that supports multiple consumers and multiple producers.
Kahn process networks \citep{kahn1976coroutines} are a flexible, expressive way of writing streaming computations, where a network is composed of communicating processes.
Executing communicating processes introduces runtime overhead, as stream elements must be passed between processes.
Instead, we wish to take this concurrent process network and convert it back to sequential code, without any runtime scheduling or message passing overhead.

% This additional communication means that a stream element which may have been in cache, has likely been swapped out by the time the consumer receives it.
% Or if the consumer is running on a different processor, it is unlikely to be in the lower level of cache in the first place.
% Furthermore, communication primitives likely require some kind of locking, which will add even more overhead.
%
% A lot of the time, the ideal execution model is actually just a simple imperative loop.
% By using concurrent processes we have gained expressivity, but at the cost of speed.

A \emph{process} in our system is a simple imperative program with a local heap.
A process pulls source values from an arbitrary number of input streams and pushes result values to at least one output stream.
The process language is an intermediate representation we use when fusing the overall dataflow network.
When describing the fusion transform we describe the control flow of the process as a state machine.

A \emph{combinator} is a template for a process which parameterises it over the particular input and output streams, as well as values of configuration parameters such as the worker function used in a @map@ process.
Each process implements a logical \emph{operator} --- so we use ``operator'' when describing the values being computed, but ``process'' and ``machine'' when referring to the implementation.


\subsection{Fold combinator}
\begin{figure}
\begin{process}
foldl 
  = \ (k  : b -> a -> b) (z   : b) (j : b -> c)
      (sIn: Stream a)    (sOut: Stream b). 
    / (s  : b) (v : a)   (F0..F4: Label).
    process
     { ins:    { sIn  }
     , outs:   { sOut }
     , heap:   { s = z, v }
     , label:    F0
     , instrs: { F0 = pull  sIn     v  F1[] else F2[]
               , F1 = drop  sIn        F0[s = k s v]

               -- sIn closed
               , F2 = push  sOut (j s) F3[]
               , F3 = close sOut       F4[]
               , F4 = exit } }
\end{process}
\caption{Process implementation of foldl}
\label{figs/procs/impl/foldl}
\end{figure}

The definition of the @foldl@ combinator, used to implement @correlation@ and @regression@ in our @priceAnalyses@ process network, is given in \autoref{figs/procs/impl/foldl}.
The combinator is parameterised by the fold state update function (@k@) and the fold state initialisation (@z@).
In @correlation@ and @regression@, the result must be extracted from the fold state; we extend the standard presentation of @foldl@ with an eject function (@j@) to perform this extraction.
The process reads from an input stream and, at the end of the input stream, produces a single-element output stream containing the fold result.
The \emph{nu-binders} ($\nu@ (s : a) (v : b)@\ldots$) indicate that each time the @foldl@ combinator is instantiated, fresh names must be given to @s@, @v@ and so on, that do not conflict with other insantiations.
The @s@ and @v@ bindings refer to variables in the mutable heap of the process.
The @s@ variable stores the current fold state and is initialised to the initial fold value (@z@); the @v@ variable stores the most recent value from the input stream, and is left uninitialised.

The body of the combinator is a record that defines the process.
The @ins@ field defines the set of input streams, and the @outs@ field defines the set of output streams.
The @heap@ field gives the initial values of each of the local variables; variables without an explicit initial value are given some arbitrary value.
The @instrs@ field contains a set of labelled instructions that define the program, while the @label@ field gives the label of the initial instruction.
In this form, the output stream (@sOut@) is defined via a parameter, rather than being the result of the combinator.

The initial instruction (\lstiproc!pull sIn v F1[] else F2[]!) pulls the next element from the stream @sIn@, writes it into the heap variable @v@, then proceeds to the instruction at label @F1@.
The empty list @[]@ after the target label @F1@ can be used to update heap variables, but as we do not need to update anything yet we leave it empty. 
If the input stream is finished, trying to pull the next element fails, and execution proceeds to the instruction at label @F2@ instead.

After successfully pulling a new element from the input stream, the instruction at label @F1@ (\lstiproc!drop sIn F0[s = k s v]!) signals that the current element that was pulled from stream @sIn@ is no longer required, before updating the fold state (@s@) by applying the fold update function (@k@).
Execution then proceeds back to the pull instruction at label @F0@.
In \REF{kpn/drop-synchronisation} we shall see how this @drop@ instruction is used to synchronise processes reading from the same shared stream, ensuring that all processes operate on the same element together without overtaking one another.

When pulling fails because the input stream is finished, the instruction (\lstiproc!push sOut (j s) F3[]!) pushes the eject function applied to the final fold state to the output stream @sOut@.
Execution then proceeds to the instruction at label @F3@.

Next, the instruction (\lstiproc!close sOut F4[]!) signals that the output stream @sOut@ is finished, and then proceeds to the instruction at label @F4@.

Finally, the instruction (\lstiproc!exit!) signals that the process is finished, and has no further work to do.
The process terminates.

\subsection{Map combinator}

\begin{figure}
\begin{process}
map 
  = \ (f  : a -> b)
      (sIn: Stream a) (sOut: Stream b). 
    / (v  : a)        (M0..M4: Label).
    process
     { ins:    { sIn  }
     , outs:   { sOut }
     , heap:   { v }
     , label:  M0
     , instrs: { M0 = pull  sIn     v  M1[] else M3[]
               , M1 = push  sOut (f v) M2[]
               , M2 = drop  sIn        M0[]

               -- sIn closed
               , M3 = close sOut       M4[]
               , M4 = exit } }
\end{process}
\caption{Process implementation of map}
\label{figs/procs/impl/map}
\end{figure}

The definition of the @map@ combinator, which applies a worker function to every element in the input stream, is given in \autoref{figs/procs/impl/map}.
The combinator is parameterised by the worker function (@f@), and takes one input stream (@sIn@) and produces one output stream (@sOut@).
The heap variable (@v@) is used to store the last value read from the input stream.
The process starts by pulling from the input stream, storing the element in the heap variable (@v@).
It then pushes the transformed element (@f v@) into the output stream, drops the element from the input stream, and pulls again.
When the input stream finishes, the process closes the output stream and terminates.

\subsection{Join combinator}

\begin{figure}
\begin{process}
join 
  = \ (cmp : a -> b -> Ordering)
      (sA  : Stream a) (sB : Stream b)
      (sOut: Stream (a,b)). 
    / (va : a) (vb : b) (c : Ordering) (...: Label).
    process
     { ins:    { sA, sB }
     , outs:   { sOut }
     , heap:   { va, vb, c }
     , label:  IN0
     , instrs: { IN0 = pull  sA va      IN1[] else DB0[]
               , IN1 = pull  sB vb      IN2[] else DA1[]
               , IN2 = jump             IN3[ c = cmp va vb ]
               , IN3 = case  (c == EQ)  EQ0[] else NE0[]

               -- cmp va vb $=$ EQ
               , EQ0 = push  sOut (a,b) EQ1[]
               , EQ1 = drop  sA         EQ2[]
               , EQ2 = drop  sB         IN0[]

               -- cmp va vb $\not=$ EQ
               , NE0 = case  (c == LT)  LT0[] else GT0[]

               -- cmp va vb $=$ LT
               , LT0 = drop  sA         LT1[]
               , LT1 = pull  sA va      IN2[] else DB1[]

               -- cmp va vb $=$ GT
               , GT0 = drop  sB         GT1[]
               , GT1 = pull  sB vb      IN2[] else DA1[]

               -- sB closed; drain sA
               , DA0 = pull  sA         DA1[] else EX0[]
               , DA1 = drop  sA         DA0[]

               -- sA closed; drain sB
               , DB0 = pull  sB         DB1[] else EX0[]
               , DB1 = drop  sB         DB0[]

               -- sA and sB closed
               , EX0 = close sOut       EX1[]
               , EX1 = exit } }
\end{process}
\caption{Process implementation of join}
\label{figs/procs/impl/join}
\end{figure}

The last combinator we need for the @priceAnalyses@ process network is the @join@ combinator, which pairs together the elements of two sorted input streams.
The combinator is parameterised by the key comparison function, which returns an @Ordering@ describing whether the key of the first argument is equal to the key of the second argument (@EQ@), lesser (@LT@), or greater (@GT@).
The process reads from two input streams (@sA@ and @sB@), and produces one output stream (@sOut@).
Two heap variables are used to store the most recent input elements (@va@ and @vb@), and another is used to store the key comparison (@c@).

In the first group of instructions, the instructions at labels @IN0@ and @IN1@ pull an element from each input stream.
If both pulls are successful, the instruction at label @IN2@ compares the input values using the key comparison function (@cmp va vb@).
Next, the instruction at label @IN3@ checks whether the keys are equal: if so, execution proceeds to the instruction at label @EQ0@; otherwise, instruction proceeds to the instruction at label @NE0@.

The group of instructions at label @EQ0@ execute when the element keys are equal, and pushes the pair of elements to the output stream, before dropping both input streams.
Execution then proceeds back to the instruction at label @IN0@ to pull from the inputs.

The instruction at label @NE0@ executes when the element keys are not equal, and proceeds to the instruction at label @LT0@ if the first element is lesser, or to the instruction at label @GT0@ if the first element is greater.
These two groups of instructions drop the input stream with the lesser key and pull a new value from the same input stream, before returning back to the instruction at label @IN2@ to compare the elements.

The group of instructions at label @DA0@ executes when the @sB@ input stream is finished, while the @sA@ input stream may still have elements.
As with the polarised stream implementation of @join_iii@ in \autoref{taxonomy/polarised}, we must \emph{drain} the leftover elements from the unfinished stream by repeatedly pulling and dropping until there are no more elements.
Without draining, the producer of the unfinished input stream would continue pushing, potentially requiring an unbounded buffer; we shall look at draining in more detail in \REF{kpn/draining}.

The group of instructions at label @DB0@ executes when the @sA@ input stream is finished, and drain the unfinished @sB@ input stream.

Finally, the group of instructions at @EX0@ close the output stream and terminate the process.

\section{Process definition}

%!TEX root = ../Main.tex

\begin{figure}
\begin{minipage}[t]{0.4\textwidth}
\begin{tabbing}
MMMMMx \TABDEF @MMMMM@  \TABSKIP $\Exp$ \TABSKIP $\Exp$ \TABSKIP $\Exp$ \kill

\Exp,~$e$       \> ::= \> ~~~ $x~|~v~|~e~e $ \\
                \> $\enskip|~$ \> ~~ $ (e~||~e) ~|~ e+e ~|~ e~@/=@~e ~|~ e < e$ \\
\Value,~$v$     \> ::= \> ~~~ $\mathbb{N}~|~\mathbb{B}~|~(\lambda{}x.~e)$ \\
\Heap,~$bs$     \> ::= \> ~~~ $\cdot~|~bs,~x~=~v$ \\
\Updates,~$us$  \> ::= \> ~~~ $\cdot~|~us,~x~=~e$
\\[0.5em]

\Proc,~$p$      \> ::=\> @process@ \\
MMMMMM \= M \= \kill
\> \> @ins:   @  $(\Chan ~\mapsto~ \InputState)$ \\
\> \> @outs:  @  $\sgl{\Chan}$ \\
\> \> @heap:  @  \Heap \\
\> \> @label: @  \Label \\
\> \> @instrs:@  $(\Label ~\mapsto~ \Instr)$ 
\\[0.5em]
\InputState \> ::= \> ~~ @none@~$|$~@pending@~\Value~$|$~@have@
\end{tabbing}

\begin{tabbing}
MMMMMM \TABDEF MMMM \TABSKIP $\Chan$ \TABSKIP $\Chan$ \TABSKIP $\Exp$ \kill

\Var,~$x$       \> $\to$ \> ~~~ (value variable) \\
\Chan,~$c$      \> $\to$ \> ~~~ (channel name) \\
\Label,~$l$     \> $\to$ \> ~~~ (label name) \\ 
\ChannelStates  \> ~ =   \> ~~~ $(\Chan \mapsto \InputState)$ \\
\Action,~$a$   \> ::=    \> ~~~ $\cdot ~~|~~ \Push~\Chan~\Value$ \\[0.5em]

\Instr
    \> ::=\> @pull@  \> \Chan  \> \Var  \> \Next \\
    \TABALT  @push@  \> \Chan  \> \Exp  \> \Next \\
    \TABALT  @drop@  \> \Chan  \>       \> \Next \\
    \TABALT  @case@  \> \Exp   \> \Next \> \Next \\
    \TABALT  @jump@  \>        \>       \> \Next \\
\\[0.5em]

\Next \> = \> $\Label~\times~\Updates$ 
\end{tabbing}
\end{minipage}
\caption{Process definitions}
\label{fig:Process:Def}
\end{figure}



The formal grammar for process definitions is given in \autoref{fig:Process:Def}.
Variables, Channels and Labels are specified by unique names.
We refer to the \emph{endpoint} of a stream as a channel.
A particular stream may flow into the input channels of several different processes, but can only be produced by a single output channel.
For values and expressions we use an untyped lambda calculus with a few primitives chosen to facilitate the examples.
The `$||$' operator is boolean-or, `+' addition, `/=' not-equal, and `$<$' less-than.

A $\Proc$ is a record with five fields: the @ins@ field specifies the input channels; the @outs@ field the output channels; the @heap@ field the process-local heap; the @label@ field the label of the instruction currently being executed, and the @instrs@ a map of labels to instructions.
We use the same record when specifying both the definition of a particular process, as well as when giving the evaluation semantics.
When specifying a process the @label@ field gives the entry-point to the process code, though during evaluation it is the label of the instruction currently being executed.
Likewise, when specifying a process we usually only list channel names in the @ins@ field, though during evaluation they are also paired with their current $\InputState$.
If an $\InputState$ is not specified we assume it is `none'.
A network is a set of processes that are able to communicate with each other.

In the grammar of \autoref{fig:Process:Def} the $\InputState$ has four options: @none@, which means no value is currently stored in the associated stream buffer variable; $(@pending@~\Value)$, which gives the current value in the stream buffer variable and indicates that it has not yet been copied into a process-local variable; @have@, which means the pending value has been copied into a process-local variable; and @closed@, which means the producer has signalled that the channel is finished and will not receive any more values.
The $\Value$ attached to the @pending@ state is used when specifying the evaluation semantics of processes.
When performing the fusion transform the $\Value$ itself will not be known, but we can still reason statically that a process must be in the @pending@ state.
When defining the fusion transform in \autoref{s:Fusion} we will use a version of $\InputState$ with only this statically known information.

The @instrs@ field of the $\Proc$ maps labels to instructions.
The possible instructions are: @pull@, which tries to pull the next value from a channel into a given heap variable, blocking until the producer pushes a value or closes the channel; @push@, which pushes the value of an expression to an output channel; @close@, which signals the end of an output channel; @drop@, which indicates that the current value pulled from a channel is no longer needed; @case@, which branches based on the result of a boolean expression; @jump@, which causes control to move to a new instruction; and @exit@, which signals that the process is finished.

Instructions include a $\Next$ field containing the label of the next instruction to execute, as well as a list of $\Var \times \Exp$ bindings used to update the heap.
The list of update bindings is attached directly to instructions to make the fusion algorithm easier to specify, in contrast to a presentation with a separate @update@ instruction.

When lowering process code to a target language, such as C, LLVM, or some sort of assembly code, we can safely convert @drop@ to plain @jump@ instructions.
The @drop@ instructions are used to control how processes should be synchronised, but do not affect the execution of a single process.
We discuss @drop@s further in \autoref{s:Drop:in:synchrony}.

% This allows us to \emph{deliberately} introduce artificial deadlocks when a process network would require more than one element of buffering.
%%% AR: added to highlight that this rules out networks that require unbounded buffers
%%% BL: We don't have any examples of explicitly introducing deadlocks. The process networks just happen to have them when viewed abstractly.

%%% AR: feels a bit disjointed because drops were only mentioned once a few paragraphs ago. Maybe reword to talk about lowering in general is obvious for most instructions, and drops are just treated as jumps. Or move up.


% -----------------------------------------------------------------------------
\subsection{Execution}
\label{s:Process:Eval}

The dynamic execution of a process network consists of:

\begin{enumerate}
\item \emph{Injection} of a single value from a stream into a process, or a network.
  Each individual process only needs to accept an injected value when it is ready for it, and injection into a network succeeds only when they \emph{all} processes accept it.

\item \emph{Advancing} a single process from one state to another.
  Advancing a network succeeds when \emph{any} of the processes in the network can advance.

\item \emph{Feeding} outputs of some processes to the inputs of others.
  Feeding alternates between Injecting and Advancing.
    When a process pushes a value to an output channel we attempt to inject this value into all processes that have that same channel as an input.
    If they all accept it, we then advance their programs as far as they will go, which may cause more values to be pushed to output channels, and so on.
\end{enumerate}

Execution of a network is non-deterministic.
At any moment several processes may be able to take a step, while others are blocked.
As with Kahn processes~\cite{kahn1976coroutines}, pulling from a channel is blocking, which enables the overall sequence of values on each output channel to be deterministic.
Unlike Kahn processes, pushing to a channel can also block.
Each consumer has a single element buffer, and pushing only succeeds when that buffer is empty.

%%% AR: what is the distinction between 'execution' and 'evaluation'?  I only have a vague feeling that execution is something a computer does, while evaluation is the mathematical rules. Either way, these should probably be consistent.
%%% BL: "Evaluation" is pure.  E-"value"-ation. Execution has visible actions, like pushing to streams.

% TODO BL: Mention what happens if we choose a different ordering,
% and how the particular ordering chosen is decided upon.
Importantly, it is the order in which values are \emph{pushed to each particular output channel} which is deterministic, whereas the order in which different processes execute their instructions is not.
When we fuse two processes we choose one particular instruction ordering that enables the network to advance without requiring unbounded buffering.
The single ordering is chosen by heuristically deciding which pair of states to merge during fusion, and is discussed in \autoref{s:EvaluationOrder}.

Each channel may be pushed to by a single process only, so in a sense each output channel is owned by a single process.
The only intra-process communication is via channels and streams.
Our model is ``pure data flow'' as there are no side-channels between processes --- in contrast to ``impure data flow'' systems such as StreamIt~\cite{thies2002streamit}.


%!TEX root = ../Main.tex

\begin{figure}

$$
\arrLR{
  \boxed{\ProcInject{\Proc}{\Value}{\Chan}{\Proc}}
}{
  \boxed{\ProcsInject{\sgl{\Proc}}{\Value}{\Chan}{\sgl{\Proc}}}
}
$$

$$
\ruleIN{
  p[@ins@][c] = @none@
}{
  \ProcInject{p}{v}{c}{p~[@ins@ \mapsto (p[@ins@][c \mapsto @pending@~v]) ] }
}{InjectValue}
$$

$$
\ruleIN{
  c \not\in p[@ins@]
}{
  \ProcInject{p}{v}{c}{p}
}{InjectIgnore}
%
\quad
%
\ruleIN{
  \{~ \ProcInject{p_i}{v}{c}{p'_i} ~\}^i
}{
  \ProcsInject{\sgl{p_i}^i}{v}{c}{\sgl{p'_i}^i}
}{InjectMany}
$$

\caption{Injection of values into input channels}
\label{fig:Process:Eval:Inject}
\end{figure}



% -----------------------------------------------------------------------------
\subsubsection{Injection}
\autoref{fig:Process:Eval:Inject} gives the rules for injecting values into processes.
Injection is a meta-level operation, in contrast to @pull@ and @push@ which are instructions in the object language.
The statement $(\ProcInject{p}{a}{p'})$ reads ``given process $p$, injecting action $a$ yields an updated process $p'$''.
An action $a$ is a message describing the state change that can occur to a channel, with three options: $(\cdot)$, the empty action, used when a process simply updates internal state; $(\Push~\Chan~\Value)$, which encodes the value a process pushes to one of its output channels; and $(\Close~\Chan)$, which denotes the end of the stream.
The @injects@ form is similar to the @inject@ form, operating on a process network.

Rule (InjectPush) injects a single value into a single process. The value is stored as a (@pending@~ v) binding in the $\InputState$ of the associated channel of the process. The $\InputState$ acts as a single element buffer, and must be empty (@none@) for injection to succeed.
Rule (InjectClose) injects a close message and correspondingly updates the input state.

Rules (InjectNopPush) and (RuleNopClose) allow processes that do not use a particular named channel to ignore messages injected into that channel.
Rule (InjectNopInternal) allows processes to ignore empty messages.

Rule (InjectMany) attempts to inject a single value into a network.
We use the single process judgment form to inject the value into all processes, which must succeed for all of them.
To inject a message into a process network, all the processes which do not ignore the message must be ready to accept the message by having the corresponding $\InputState$ set to @none@; otherwise, the process would require more than a single-element buffer to store multiple messages.
% Once a value has been injected into all consuming processes that require it, the producing process no longer needs to retain it.


%!TEX root = ../Main.tex

% -----------------------------------------------------------------------------
\begin{figure}
\begin{tabbing}
MM \= MMMMMMM \= MM \= MMMMMMMMM\kill
\end{tabbing}


% -----------------------------
$$
  \boxed{
    \ProcBlockShake
      {\Instr}
      {\ChannelStates}
      {\Heap}
      {\Action}
      {\Label}
      {\ChannelStates}
      {\Updates}
  }
$$

$$
\ruleIN{
  is[c] = @pending@~v
}{
  \ProcBlockShake
        {@pull@~c~x~(l,us)~(l',us')}
        {is}
        {\Sigma}
        {\cdot}
        {l}
        {is[c \mapsto @have@]}
        {(us, x = v)}
}{PullPending}
$$

$$
\ruleIN{
  is[c] = @closed@
}{
  \ProcBlockShake
        {@pull@~c~x~(l,us)~(l',us')}
        {is}
        {\Sigma}
        {\cdot}
        {l'}
        {is}
        {us'}
}{PullClosed}
$$

$$
\ruleIN{
  \ExpEval{\Sigma}{e}{v}
}{
  \ProcBlockShake
        {@push@~c~e~(l,us)}
        {is}
        {\Sigma}
        {\Push~c~v}
        {l}
        {is}
        {us}
}{Push}
$$

$$
\ruleAx{
  \ProcBlockShake
        {@close@~c~(l,us)}
        {is}
        {\Sigma}
        {\Close~c}
        {l}
        {is}
        {us}
}{Close}
$$


$$
\ruleIN{
  is[c] = @have@
}{
  \ProcBlockShake
        {@drop@~c~(l,us)}
        {is}
        {\Sigma}
        {\cdot}
        {l}
        {is[c \mapsto @none@]}
        {us}
}{Drop}
\ruleIN{
}{
  \ProcBlockShake
        {@jump@~(l,us)}
        {is}
        {\Sigma}
        {\cdot}
        {l}
        {is}
        {us}
}{Jump}
$$

$$
\ruleIN{
  \ExpEval{\Sigma}{e}{@True@}
}{
  \ProcBlockShake
        {@case@~e~(l_t,us_t)~(l_f,us_f)}
        {is}
        {\Sigma}
        {\cdot}
        {l_t}
        {is}
        {us_t}
}{CaseT}
$$
$$
\ruleIN{
  \ExpEval{\Sigma}{e}{@False@}
}{
  \ProcBlockShake
        {@case@~e~(l_t,us_t)~(l_f,us_f)}
        {is}
        {\Sigma}
        {\cdot}
        {l_f}
        {is}
        {us_f}
}{CaseF}
$$

\vspace{2em}

% ----------------
$$
  \boxed{\ProcShake{\Proc}{\Action}{\Proc}}
$$
$$
\ruleIN{
  \ProcBlockShake
    {p[@instrs@][p[@label@]]} 
    {p[@ins@]}
    {p[@heap@]}
    {a}
    {l}
    {is}
    {us}
  \qquad
    \ExpEval{p[@heap@]}{us}{bs}
}{
  \ProcShake
        {p}
        {a}
        {p~[    @label@~ \mapsto ~l
           , ~~ @heap@~  \mapsto (p[@heap@] \lhd bs)
           , ~~ @ins@~   \mapsto ~is]}
}{Advance}
$$

% ---------------------------------------------------------
\vspace{2em}

$$
  \boxed{\ProcsShake{\sgl{\Proc}}{\Action}{\sgl{\Proc}}}
$$

$$
\ruleIN{
  \ProcShake{p_i}{a}{p'_i}
  \qquad
  \forall j~|~j \neq i.~
  \ProcInject{p_j}{a}{p'_j}
}{
  \ProcsShake{
    \sgl{p_0 \ldots p_i \ldots p_n}
  }{a}{
    \sgl{p'_0 \ldots p'_i \ldots p'_n}
  }
}{AdvanceMany}
$$



\caption{Advancing processes}

% Evaluation: shaking allows proceses to take a step from one label to another as well as produce an output message. If the message is a push, the value is injected to all other processes in the network; otherwise it is an internal step.}
\label{fig:Process:Eval:Shake}
\end{figure}




% -----------------------------------------------------------------------------
\subsubsection{Advancing}
\autoref{fig:Process:Eval:Shake} gives the rules for advancing a single process and process networks.
The statement $(\ProcBlockShake{i}{is}{bs}{a}{l}{is'}{us'})$ reads ``instruction $i$, given channel states $is$ and the heap bindings $bs$, passes control to instruction at label $l$ and yields new channel states $is'$, heap update expressions $us'$, and performs an output action $a$.''

Rule (PullPending) takes the @pending@ value $v$ from the channel state and produces a heap update to copy this value into the variable $x$ in the @pull@ instruction.
Control is passed to the first output label, $l$.
We use the syntax $us,x=v$ to mean that the list of updates $us$ is extended with the new binding $x=v$.
In the result channel states, the state of the channel $c$ that was pulled from is set to @have@, to indicate the value has been copied into the local variable.

Rule (PullClosed) applies when the channel state is @closed@, passing control to the second output label, $l'$.
As the channel remains closed, there is no need to update the channel state as in the (PullPending) rule.

Rule (Push) evaluates the expression $e$ under heap bindings $bs$ to a value $v$, and produces a corresponding action which carries this value.
The judgment $(bs \vdash e \Downarrow v)$ expresses standard untyped lambda calculus reduction using the heap $bs$ for the values of free variables. As this evaluation is completely standard we omit it to save space.

Rule (Close) emits a @close@ action; once injected, this action will transition the recipients' channel states to @closed@.
Once a channel is closed, it can no longer be pushed to as the recipients' channel states cannot transition back to @none@.

Rule (Drop) changes the input channel state from @have@ to @none@. A @drop@ instruction can only be executed after @pull@ has set the input channel state to @have@.

Rule (Jump) produces a new label and associated update expressions. Rules (CaseT) and (CaseF) evaluate the scrutinee $e$ and emit the appropriate label.

There is no corresponding rule for the @exit@ instruction, which denotes a finished process.

The statement $\ProcShake{p}{a}{p'}$ reads ``process $p$ advances to new process $p'$, yielding action $a$''. Rule (Advance) advances a single process. We look up the current instruction for the process' @label@ and pass it, along with the channel states and heap, to the above single instruction judgment. The update expressions $us$ from the single instruction judgment are reduced to values before updating the heap. We use $(us \lhd bs)$ to replace bindings in $us$ with new ones from $bs$. As the update expressions are pure, the evaluation can be done in any order.

The statement $\ProcShake{ps}{a}{ps'}$ reads ``the network $ps$ advances to the network $ps'$ yielding output action $a$''.

Rule (AdvanceMany) allows an arbitrary, non-deterministically chosen process in the network to advance to a new state while yielding an output action $a$.
For this to succeed it must be possible to inject the action into all other processes.
As all consuming processes must accept the output action at the time it is created, there is no need to buffer it further in the producing process.
When any process in the network produces an output action we take that as the action of the whole network.

%!TEX root = ../Main.tex

\begin{figure}

\newcommand\vs {\ti{vs}}
\newcommand\ps {\ti{ps}}

$$
  \boxed{
    \ProcsFeed
      {(\Chan \mapsto \overline{Value})~}
      {\sgl{\Proc}}
      {(\Chan \mapsto \overline{Value})~}
      {\sgl{\Proc}}
  }
$$

$$
\ruleIN{
  \ProcsShake
    {ps}
    {\cdot}
    {ps'}
\qquad
  \ProcsFeed{i}{ps'}{o}{ps''}
}{
  \ProcsFeed
    {i}
    {ps}
    {o}
    {ps''}
}{FeedInternal}
$$

$$
\ruleIN{
  \ProcsShake
    {ps}
    {\Push~c~v}
    {ps'}
\qquad
  \ProcsFeed
    {i}
    {ps'}
    {o}
    {ps''}
\qquad
  (c \mapsto \vs) \in o
}{
  \ProcsFeed
    {i}
    {ps}
    {o[c \mapsto ([v] \listappend \vs)]}
    {ps''}
}{FeedPush}
$$

$$
\ruleIN{
  \ProcsShake
    {ps}
    {\Close~c}
    {ps'}
\qquad
  \ProcsFeed
    {i}
    {ps'}
    {o}
    {ps''}
}{
  \ProcsFeed
    {i}
    {ps}
    {o[c \mapsto []]}
    {ps''}
}{FeedClose}
$$





$$
\ruleIN{
%   (\forall p \in \ps.~c \not\in p[@outs@])
% \qquad
  \ProcsInject
    {ps}
    {(\Push~c~v)}
    {ps'}
\qquad
  \ProcsFeed
    {i[c \mapsto \vs]}
    {ps'}
    {o}
    {ps''}
}{
  \ProcsFeed
    {i[c \mapsto ([v] \listappend vs)]}
    {ps}
    {o}
    {ps''}
}{FeedExternalPush}
$$

$$
\ruleIN{
%  (\forall p \in \ps.~c \not\in p[@outs@])
%\qquad
  \ProcsInject
    {ps}
    {(\Close~c)}
    {ps'}
\qquad
  \ProcsFeed
    {i}
    {ps'}
    {o}
    {ps''}
}{
  \ProcsFeed
    {i[c \mapsto [] ]}
    {ps}
    {o}
    {ps''}
}{FeedExternalClose}
$$



\caption{Feeding Process Networks}
\label{fig:Process:Eval:Feed}
\end{figure}



% -----------------------------------------------------------------------------
\subsubsection{Feeding}
\autoref{fig:Process:Eval:Feed} gives the rules for collecting output actions and feeding the contained values to other processes.
These rules exchange input and output values with the environment in which the network runs.
% The first set of rules concerns feeding values to other processes within the same network, while the second exchanges input and output values with the environment the network is running in.

The statement $\ProcsFeed{cvs}{ps}{cvs'}{ps'}$ reads ``with channel values $cvs$, network $ps$ takes a step and produces new channel values $cvs'$ and network $ps'$''.
The channel values $cvs$ map channel names to a list of values.
For input channels of the overall network, we initialise the map to contain a list of input values for each channel.
For output channels of the overall network, values pushed to those channels are also collected in the same channel map.
In a concrete implementation the input and output values would be transported over some IO device, but for the semantics we describe the abstract behavior only.

Rule (FeedInternal) allows the network to perform local computation in the context of the channel values.

Rule (FeedPush) collects an output action containing a pushed value (@push@ $c$ $v$) produced by a network and appends the value $v$ to the list corresponding to the output channel $c$.

Rule (FeedClose) discards a close output action (@close@ $c$) produced by a network; this signals that the corresponding list of values in the channel value map are complete, containing all pushed values.

Rule (FeedExternalPush) injects values from the external environment as push messages.
This rule also has the side condition that values cannot be injected from the environment into output channels that are already owned by some process.
This constraint is required for correctness proofs, but can be ensured by construction in a concrete implementation.
The list of values in the channel value map is updated to remove the pushed value.

Rule (FeedExternalClose) injects a close message for an external environment stream when the corresponding list is empty.
Channels are only closed once; injecting a close message transitions the recipient's channel state from (@none@) to (@closed@).

% The topology of the dataflow network does not change at runtime, so it only needs to be checked once, before execution.


% -----------------------------------------------------------------------------
\subsection{Non-deterministic Execution Order}
\label{s:EvaluationOrder}

The execution rules of \autoref{fig:Process:Eval:Feed} are non-deterministic in several ways. Rule (ProcessInternal) allows any process to perform internal computation at any time, without synchronising with other processes in the network; (ProcessPush) allows any process to perform a push action at any time, provided all other processes in the network are ready to accept the pushed value; (FeedExternal) also allows new values to be injected from the environment, provided all processes that use the channel are ready to accept the value.

In the semantics, allowing the execution order of processes to be non-deterministic is critical, as it defines a search space where we might find an order that does not require unbounded buffering. For a direct implementation of concurrent processes using message passing and operating system threads, an actual, working, execution order would be discovered dynamically at runtime. In contrast, the role of our fusion system is to construct one of these working orders statically. In the fused result process, the instructions will be scheduled so that they run in one of the orders that would have arisen if the network were executed dynamically. Fusion also eliminates the need to pass messages between processes --- once they are fused we can just copy values between heap locations.

% In our system, allowing the execution order of processes to be non-deterministic is critical, as it provides freedom to search for a valid ordering that does not require excessive buffering. Consider the following example, where the @alt2@ operator pulls two elements from its first input stream, then two from the second, before pushing all four to its output stream.
% \begin{code}
%   alternates : S Nat -> S Nat -> S Nat -> S (Nat, Nat)
%   alternates sInA sInB sInC
%    = let  s1   = alt2 sInA sInB
%           s2   = alt2 sInB sInC
%           sOut = zip s1 s2
%      in   sOut
% \end{code}
%
% Note that the middle stream @sInB@ is shared, and the result streams from both @alt2@ operators are zipped into tuples. Given the inputs @sInA@ = @[a1,a2]@, @sInB@ = @[b1,b2]@ and @sInC@ = @[c1,c2]@ the output of @zip@ will be @[(a1,b1),(a2,b2),(b1,c1),(b2,c2)]@, assuming @a1,a2,b1,b2@ and so on are values of type @Nat@.
%
% Now, note that the first @alt2@ process pushes values to its output stream @s1@ two at a time, and the second @alt2@ process also pushes values to its own output stream @s2@ two at a time. However, the downstream @zip@ process needs to pull one value from @s1@ then one from @s2@, then another from @s1@, then another from @s2@, alternating between the @s1@ and @s2@ streams. This will work, provided we can arrange for the two \emph{separate} @alt2@ processes to push to their separate output streams alternatively. They can still push two values at a time to their own outputs, but the downstream @zip@ process needs receive one from each process alternately. Here is a table of intermediate values to help make the explanation clearer:
%
% \begin{code}
%     sInA = [a1, a2, a3, a4, a5 ...]
%     sInB = [b1, b2, b3, b4, b5 ...]
%     sInC = [c1, c2, c3, c4, c5 ...]
%
%     s1   = alt2 sInA sInB
%          = [a1, a2, b1, b2, a3, a4, b3, b4 ...]
%
%     s2   = alt2 sInB sInC
%          = [b1, b2, c1, c2, b3, b4, c3, c4 ...]
%
%     sOut = zip s1 s2
%          = [(a1,b1), (a2,b2), (b1,c1), (b2,c2) ...]
% \end{code}
%
% Considering the last line in the above table, note that @zip@ needs to output a tuple of @a1@ and @b1@ together, then @a2@ and @b2@ together, and so on. The implementation of the @zip@ process will attempt to pull the first value @a1@ from stream @s1@, blocking until it gets it, then pull the next value @b1@ from stream @s2@, blocking until it gets it. While @zip@ is blocked waiting for @b1@, the first @alt2@ process cannot yet push @a2@. The execution order of the overall network is constrained by communication patterns of processes in that network.

% As we cannot encode all possible orderings into the definition of the processes themselves, we have defined the execution rules to admit many possible orderings. In a direct implementation of concurrent processes using message passing and operating system threads, an actual, working, execution order would be discovered dynamically at runtime. In contrast, the role of our fusion transform is to construct one of these working orders statically. In the fused result process, the instructions will be scheduled so that they run in one of the orders that would have arisen if the network was executed dynamically. In doing so, we also eliminate the need to pass messages between processes --- once they are fused we can just copy values between heap locations.

% Although alt2 produces output elems two at a time, the consumer zip need its input elements to arrive alternately. At evaluation time we need the results pushed to sA1 and sA2 in the sA1 sA2 sA1 sA2 order, not sA1 sA1 sA2 sA2. Writing the rules nondeterministically allows the elaborator to discover a usable order, if there is one. This also affects fusion, we don't want to commit to the wrong order too early. We shall see that if we fuse the two alt processes first fusion will not work. We need to start with zip so that the order in which input elems arrive is constrained.


\section{Fusion}
\label{s:Fusion}

Our core fusion algorithm constructs a static execution schedule for a single pair of processes. To fuse a whole process network we fuse successive pairs of processes until only one remains.

\autoref{fig:Fusion:Types} defines some auxiliary grammar used during fusion. We extend the $\Label$ grammar with a new alternative, $\LabelF \times \LabelF$ for the labels in a fused result process. Each $\LabelF$ consists of a $\Label$ from a source process, paired with a map from $\Chan$ to the statically known part of that channel's current $\InputState$. When fusing a whole network, as we fuse pairs of individual processes the labels in the result collect more and more information. Each label of the final, completely fused process encodes the joint state that all the original source processes would be in at that point.

% The definition of $\Label$ is now recursive.
% These new labels $LabelF$ consist of a pair of source labels, as well as the static part of the $\InputState$ of each input channel.

% If the static $\InputStateF$ is $@pending@_F$, there is a value waiting to be pulled, do not know the actual value.

We also extend the existing $\Var$ grammar with a (@chan@ $c$) form which represents the buffer variable associated with \mbox{channel @c@}. We only need one buffer variable for each channel, and naming them like this saves us from inventing fresh names in the definition of the fusion rules.
We used a fresh name back in \autoref{s:Fusion:FusingPulls} to avoid introducing a new mechanism at that point in the discussion.

Still in \autoref{fig:Fusion:Types}, $\ChanTypeTwo$ classifies how channels are used, and possibly shared, between two processes. Type @in2@ indicates that the two processes @pull@ from the same channel, so these actions must be coordinated. Type @in1@ indicates that only a single process pulls from the channel. Type @in1out1@ indicates that one process pushes to the channel and the other pulls. Type @out1@ indicates that the channel is pushed to by a single process. Each output channel is uniquely owned and cannot be pushed to by more than one process.

%!TEX root = ../Main.tex

\begin{figure}

\begin{tabbing}
@MMMMMMMMMMMM@   \TABDEF \kill

$\Label$        \> ::=  \> ~~~ $\ldots ~|~\LabelF~\times~\LabelF ~|~ \ldots$ \\
$\LabelF$      \> =    \> ~~~ $\Label~\times~\MapType{\Chan}{\InputStateF}$  \\
$\InputStateF$ \> ::=  \> ~~~ $@none@_F ~|~ @pending@_F ~|~ @have@_F ~|~ @closed@_F$    \\
$\Var$          \> ::=  \> ~~~ $\ldots ~|~@chan@~\Chan ~|~ \ldots$ \\
\\

$\ChanType_2$   \> ::=  \> ~~~ $@in2@~|~@in1@~|~@in1out1@~|~@out1@$
\end{tabbing}

\caption{Fusion type definitions.}
% The labels for a fused program consists of both of the original program labels, as well as the statically known part of the input state for each channel. The channels of both processes are classified into inputs and outputs, this describes what coordination is required between the two.
\label{fig:Fusion:Types}
\end{figure}



%!TEX root = ../Main.tex

\begin{figure}

\begin{tabbing}
\ti{fusePair}~@M M@   \TABDEF \kill

\ti{fusePair} \> $~:$ \> $\Proc \to \Proc \to  \Maybe~\Proc$ \\
$\ti{fusePair}~p~q$ \> $=$ \\
@    process@ \\
@        ins: @ $\sgl{c~|~c=t \in \cs,~t \in \sgl{@in1@,@in2@}} $ \\
@       outs: @ $\sgl{c~|~c=t \in \cs,~t \in \sgl{@in1out1@,@out1@}} $ \\
@       heap: @ $p[@heap@]~\cup~q[@heap@]$ \\
@      label: @ $l_0$ \\
@     instrs: @ $\ti{go}~\sgl{}~l_0$ \\
@ where@ \\
MM\=MM\=~=~\=\kill
 \> \cs \> $=$ \> $\ti{channels}~p~q$ \\
 \> $l_0$   \> $=$ \> $
      \big( 
      (p[@label@],~\sgl{c=none_S~|~c~\in~p[@ins@]}),~
      (q[@label@],~\sgl{c=none_S~|~c~\in~q[@ins@]})
      \big)$ \\
 \\
 \> $\ti{go}~\ti{bs}~(l_p,l_q)$ \\
 \> \> $~|$ \> $(l_p,l_q)~\in~\ti{bs}$ \\
 \> \> $=$  \> $\ti{bs}$ \\
 \> \> $~|$ \>
        $b~\in~\ti{tryStepPair}~\cs~l_p~p[@instrs@][l_p]~l_q~q[@instrs@][l_q]$ \\ 
 \> \> $=$ \> $\ti{fold}~\ti{go}~(\ti{bs}~\cup~\sgl{(l_p,l_q)=b})~(\ti{outlabels}~b)$
\end{tabbing}

\caption{Fusion of pairs of processes}

% Two processes are fused together by starting at the initial label for each process and computing the instruction based on one of the original process' instructions at that label. Instructions are added recursively until all reachable instructions are included.

\label{fig:Fusion:Def:Top}
\end{figure}






\smallskip
% -------------------------------------
\autoref{fig:Fusion:Def:Top} defines function \ti{fusePair} that fuses a pair of processes, constructing a result process that does the job of both. We start with a joint label $l_0$ formed from the initial labels of the two source processes. We then use \ti{tryStepPair} to statically choose which of the two processes to advance, and hence which instruction to execute next. The possible destination labels of that instruction (computed with $outlabels$ from \autoref{fig:Fusion:Utils}) define new joint labels and reachable states. As we discover reachable states we add them to a map $bs$ of joint label to the corresponding instruction, and repeat the process to a fixpoint where no new states can be discovered.

%!TEX root = ../Main.tex

% Settle on a syntax for this later.
% Might be too many nested pairs.
\newcommand\nextStep[5]{\big((#1,~#2),~(#3,~#4),~#5 \big)}


% I tried this colour in a colour blindness simulator and it seems to be OK.
% Should still be readable when converted to grayscale.
\definecolor{notec}{HTML}{C03020}
\newcommand\note[1]{\textcolor{notec}{(#1)}}

\begin{figure}
\begin{tabbing}
M \= M \= M \= M \kill
$\ti{tryStepPair} ~:~ \ChanTypeMap \to \Label_1 \to \Instr \to \Label_1 \to \Instr \to \Maybe~\Instr$ \\
$\ti{tryStepPair} ~\cs~l_p~i_p~l_q~i_q$ \\

\> \note{PreferJump1} \\
\> $~|~i_p'~\in~\ti{tryStep}~\cs~l_p~i_p~l_q ~\wedge~@jump@~(l,u)~\in~i_p'$ 
   ~~~~~ $\to~i_p'$ \\
\> \note{PreferJump2} \\
\> $~|~i_q'~\in~\ti{tryStep}~\cs~l_q~i_q~l_p ~\wedge~@jump@~(l,u)~\in~i_q'$
   ~~~~~ $\to~\ti{swaplabels}~i_q'$ 
\\[0.5em]

\> \note{DeferPull1} \\
\> $~|~i_p'~\in~\ti{tryStep}~\cs~l_p~i_p~l_q ~\wedge~ i_q'~\in~\ti{tryStep}~\cs~l_q~i_q~l_p$ 
   ~~ $\wedge~@pull@~c~x~(l,u)~\not\in~i_p'$ 
   ~~~~~ $\to~i_p'$ \\
\> \note{DeferPull2} \\
\> $~|~i_p'~\in~\ti{tryStep}~\cs~l_p~i_p~l_q ~\wedge~i_q'~\in~\ti{tryStep}~\cs~l_q~i_q~l_p$
   ~~ $\wedge~@pull@~c~x~(l,u)~\not\in~i_q'$ 
   ~~~~~ $\to~\ti{swaplabels}~i_q'$ 
\\[0.5em]

\> \note{Run1} \\
\> $~|~i_p'~\in~\ti{tryStep}~\cs~l_p~i_p~l_q$ ~~~~~ $\to~i_p'$ \\
\> \note{Run2} \\
\> $~|~i_q'~\in~\ti{tryStep}~\cs~l_q~i_q~l_p$ ~~~~~ $\to~\ti{swaplabels}~i_q'$
\end{tabbing}
\caption{Fusion step coordination for a pair of processes.}
% Statically compute the instruction to perform at a particular fused label. Try to execute either process, preferring jumps and other instructions over pulling, as pulling can block while other instructions may perform ``useful work'' without blocking. If neither machine can execute, fusion fails.
\label{fig:Fusion:Def:StepPair}
\end{figure}



% -------------------------------------
\autoref{fig:Fusion:Def:StepPair} defines function \ti{tryStepPair} which decides which process to advance. It starts by calling \ti{tryStep} for both processes. If both can advance, we use heuristics to decide which one to run first.

Clauses (DeferExit1) and (DeferExit2) prioritise processes that can execute an instruction over finished processes that are ready to @exit@.
This ensures that the fused process will only finish once both processes are finished; if either has remaining work, the fused process will continue.
The instruction for the second process was computed by calling \ti{tryStep} with the label arguments swapped, so in (DeferExit2) we need to swap the labels back with $\ti{swaplabels}$ (from \autoref{fig:Fusion:Utils}).

Clauses (PreferJump1) and (PreferJump2) prioritise processes that can perform a @jump@.
This helps collect @jump@ instructions together so they are easier for post-fusion optimisation to handle (\autoref{s:Optimisation}).

Similarly, clauses (DeferPull1) and (DeferPull2) defer @pull@ instructions: if one of the instructions is a @pull@, we advance the other one. We do this because @pull@ instructions may block, while other instructions are more likely to produce immediate results.

Clauses (Run1) and (Run2) apply when the above heuristics do not apply, or only one of the processes can advance.
% We try the first process first, and if that can advance then so be it. This priority means that fusion is left-biased, preferring advancement of the left process over the second.

Clause (Deadlock) applies when neither process can advance, in which case the processes cannot be fused together and fusion fails.


%!TEX root = ../Main.tex

\begin{figure*}

\begin{tabbing}
$\ti{tryStep} ~:~ \ChanTypeMap \to \LabelF \to \Instr \to \LabelF \to \Maybe~\Instr$ \\
$\ti{tryStep} ~\cs~(l_p,s_p)~i_p~(l_q,s_q)~=~@match@~i_p~@with@$ 
\end{tabbing}

\vspace*{-\baselineskip}
\vspace*{1ex}

\begin{tabular}{lr}
  $@jump@~(l',u')$  & \note{LocalJump} \\
  $\quad\to~\Just (@jump@~
      \nextStep
        {l'}{s_p}
        {l_q}{s_q}
        {u'})
      $ 
      \\[1ex]
  $@case@~e~(l'_t,u'_t)~(l'_f,u'_f)$ & \note{LocalCase} \\
$\quad\to~\Just (@case@~e~
      \nextStep
        {l'_t}{s_p}
        {l_q}{s_q}
        {u'_t}
      ~
      \nextStep
        {l'_f}{s_p}
        {l_q}{s_q}
        {u'_f})
      $ 
\\[1ex]

$@push@~c~e~(l',u')$ \\
  $\quad~|~\cs[c]=@out1@$ & \note{LocalPush} \\
$\quad\to~\Just (@push@~c~e~
      \nextStep
        {l'}
          {s_p}
        {l_q}
          {s_q}
        {u'})
      $ 
\\
  $\quad~|~\cs[c]=@in1out1@ ~\wedge~ s_q[c]=@none@_F$ & \note{SharedPush} \\
$\quad\to~\Just (@push@~c~e~
      \nextStep
        {l'}
          {s_p}
        {l_q}
          {\HeapUpdateOne{c}{@pending@_F}{s_q}}
        {\HeapUpdateOne{@chan@~c}{e}{u'}})
      $
\\[1ex]

$@drop@~c~(l',u')$ \\
  $\quad~|~\cs[c]=@in1@$ & \note{LocalDrop} \\
  $\quad\to~\Just (@drop@~c~
      \nextStep
        {l'}
          {s_p}
        {l_q}
          {s_q}
        {u'})
      $
      \\

  $\quad~|~\cs[c]=@in1out1@$ & \note{ConnectedDrop} \\
$\quad\to~\Just (@jump@~
      \nextStep
        {l'}
          {\HeapUpdateOne{c}{@none@_F}{s_p}}
        {l_q}
          {s_q}
        {u'})
      $
      \\

$\quad~|~\cs[c]=@in2@ ~\wedge~ (s_q[c]=@have@_F \vee s_q[c]=@pending@_F)$  & \note{SharedDropOne} \\
$\quad\to~\Just (@jump@~
      \nextStep
        {l'}
          {\HeapUpdateOne{c}{@none@_F}{s_p}}
        {l_q}
          {s_q}
        {u'})
      $
      \\


$\quad~|~\cs[c]=@in2@ ~\wedge~ s_q[c]=@none@_F$ & \note{SharedDropBoth} \\
$\quad\to~\Just (@drop@~c~
      \nextStep
        {l'}
          {\HeapUpdateOne{c}{@none@_F}{s_p}}
        {l_q}
          {s_q}
        {u'})
      $
\\[1ex]

$@pull@~c~x~(l'_o,u'_o)~(l'_c,u'_c)$ \\

$\quad~|~\cs[c]=@in1@$ & \note{LocalPull} \\
$\quad\to~\Just (@pull@~c~x~
      \nextStep
        {l'_o}{s_p}
        {l_q}{s_q}
        {u'_o}
      ~
      \nextStep
        {l'_c}{s_p}
        {l_q}{s_q}
        {u'_c})
    $ 
\\[1ex]

$\quad~|~(\cs[c]=@in2@ \vee \cs[c]=@in1out1@) ~\wedge~ s_p[c]=@pending@_F$ & \note{SharedPull} \\
$\quad\to~\Just (@jump@~
      \nextStep
        {l'_o}
          {\HeapUpdateOne{c}{@have@_F}{s_p}}
        {l_q}
          {s_q}
        {\HeapUpdateOne{x}{@chan@~c}{u'_o}})
        $ 
\\[1ex]

$\quad~|~\cs[c]=@in2@ ~\wedge~ s_p[c]=@none@_F ~\wedge~ s_q[c]=@none@_F$ & \note{SharedPullInject} \\
$\quad\to~\Just (@pull@~c~(@chan@~c)$ \\
@              @
      $\nextStep
        {l_p}
          {\HeapUpdateOne{c}{@pending@_F}{s_p}}
        {l_q}
          {\HeapUpdateOne{c}{@pending@_F}{s_q}}
        {[]}$
      \\
@              @
      $\nextStep
        {l_p}
          {\HeapUpdateOne{c}{@closed@_F}{s_p}}
        {l_q}
          {\HeapUpdateOne{c}{@closed@_F}{s_q}}
        {[]})
  $
\\[1ex]

$\quad~|~(\cs[c]=@in2@ \vee \cs[c]=@in1out1@) ~\wedge~ s_p[c]=@closed@_F$ & \note{SharedPullClosed} \\
$\quad\to~\Just (@jump@~
      \nextStep
        {l'_c}{s_p}
        {l_q}{s_q}
        {u'_c})
  $
\\[1ex]

$@close@~c~(l',u')$ \\
$\quad~|~\cs[c]=@out1@$ & \note{LocalClose} \\
$\quad\to~\Just (@close@~c~
      \nextStep
        {l'}{s_p}
        {l_q}{s_q}
        {u'})
    $
    \\

$\quad~|~\cs[c]=@in1out1@ ~\wedge~ s_q[c]=@none@_F$ & \note{SharedClose} \\
$\quad\to~\Just (@close@~c~
      \nextStep
        {l'}{s_p}
        {l_q}{\HeapUpdateOne{c}{@closed@_F}{s_q}}
        {u'})
    $ 
\\[1ex]

$@exit@$ & \note{LocalExit} \\
$\quad\to~\Just @exit@$
\\[1ex]

$\_~|~ @otherwise@ $ & \note{Blocked} \\
$\quad\to ~ \Nothing$
\end{tabular}

\caption{Fusion step for a single process of the pair.} 

% Given the state of both processes, compute the instruction this process can perform. This is analogous to statically evaluating the pair of processes. If this process cannot execute, the other process may still be able to.
\label{fig:Fusion:Def:Step}
\end{figure*}




% -------------------------------------
\smallskip
\autoref{fig:Fusion:Def:Step} defines function \ti{tryStep} which schedules a single instruction. This function takes the map of channel types, along with the current label and associated instruction of the first (left) process, and the current label of the other (right) process.

Clause (LocalJump) applies when the left process wants to jump.
In this case, the result instruction simply performs the corresponding jump, leaving the right process where it is.
This clause corresponds to a static version of the rule (Jump) for advancing processes during execution (\autoref{s:Process:Eval}).

Clause (LocalCase) is similar, except there are two $\Next$ labels.

Clause (LocalPush) applies when the left process wants to push to a non-shared output channel.
In this case the push can be performed directly, with no additional coordination required.

Clause (SharedPush) applies when the left process wants to push to a shared channel.
Pushing to a shared channel requires the downstream process to be ready to accept the value at the same time.
We encode this constraint by requiring the static input state of the downstream channel to be $@none@_F$.
When this constraint is satisfied, the result instruction stores the pushed value in the stream buffer variable $(@chan@~c)$ and sets the static input state to $@pending@_F$, which indicates that the new value is now available.
This clause corresponds to a static version of the rule (Push) for advancing the left process, combined with the rule (InjectPush) for injecting the push action into the right process.

Still in \autoref{fig:Fusion:Def:Step}, clause (LocalPull) applies when the left process wants to pull from a local channel, which requires no coordination.

Clause (SharedPullPending) applies when the left process wants to pull from a shared channel that the other process either pulls from or pushes to.
We know that there is already a value in the stream buffer variable, because the state for that channel is $@pending@_F$.
The result instruction copies the value from the stream buffer variable into a variable specific to the left source process, and the corresponding $@have@_F$ channel state in the result label records that it has done so.

Clause (SharedPullClosed) applies when the left process wants to pull from a shared channel that the other process either pulls from or pushes to, and the channel is closed.
The result instruction jumps to the close output label.

Clause (SharedPullInject) applies when the left process wants to pull from a shared channel that both processes pull from, and neither already has a value.
The result instruction is a @pull@ that loads the stream buffer variable, leaving the labels the same and updating the channel state for both processes.
In the next instruction the left process will try to pull again with the updated channel state, and one of the clauses (SharedPullPending) or (SharedPullClosed) will apply.

Clause (LocalDrop) applies when the left process wants to drop the current value that it read from an unshared input channel, which requires no coordination.

Clause (ConnectedDrop) applies when the left process wants to drop the current value that it received from an upstream process. As the value will have been sent via a heap variable instead of a still extant channel, the result instruction just performs a @jump@ while updating the static channel state.

Clauses (SharedDropOne) and (SharedDropBoth) apply when the left process wants to drop from a channel shared by both processes. In (SharedDropOne) the channel states reveal that the other process is still using the value. In this case the result is a @jump@ updating the channel state to note that the left process has dropped. In (SharedDropBoth) the channel states reveal that the other process no longer needs the value. In this case the result is a real @drop@, because we are sure that neither process requires the value any longer.

Clause (LocalClose) applies when the left process wants to close an unshared output channel, which requires no coordination.

Clause (SharedClose) applies when the left process wants to close a shared output channel.
Closing the channel updates the channel state and requires the downstream process to have dropped any previously read values.

Clause (LocalExit) applies when the left process wants to finish execution, which requires no coordination here, but prioritises the other process in the (DeferExit) clauses in the earlier \ti{tryStepPair} definition.

Clause (Blocked) returns @Nothing@ when no other clauses apply, meaning that this process is waiting for the other process to advance.


% -------------------------------------
%!TEX root = ../Main.tex
\begin{figure}

\begin{tabbing}
$\ChanType_2$   \TABDEF \kill

\ti{channels} \> $:$ \> $\Proc \to \Proc \to \MapType{\Chan}{\ChanType_2}$ \\

  \> $=$    \> $\{ c=@MMMMMMM@~$\= \kill
$\ti{channels}~p~q$
  \> $=$    \> $\{ c=@in2@$
            \> $|~c~\in~(@ins@~p~\cap~@ins@~q) \}$ 
            \\

  \> $\cup$ \> $\{ c=@in1@$
            \> $ |~c~\in~(@ins@~p~\cup~@ins@~q)~\wedge~c~\not\in(@outs@~p~\cup~@outs@~q) \}$ \\

  \> $\cup$ \> $\{ c=@in1out1@$
            \> $|~c~\in~(@ins@~p~\cup~@ins@~q)~\wedge~c~\in(@outs@~p~\cup~@outs@~q) \}$ \\

  \> $\cup$ \> $\{ c=@out1@$
            \> $ |~c~\not\in~(@ins@~p~\cup~@ins@~q)~\wedge~c~\in(@outs@~p~\cup~@outs@~q) \}$ 
\end{tabbing}

\newcommand\funClauseDef[3]
{ $\ti{#1}~(#2)$ \> $=$ \> $#3$
}
\newcommand\outlabelsDef[2]
{ \funClauseDef{outlabels}{#1}{\sgl{#2}} 
}

\vspace{1ex}
\begin{tabbing}
$\ti{outlabels}~(@case@~e~(l_t,u_t)~(l_t,u_f))$ \TABSKIP $=$ \TABSKIP \kill
\ti{outlabels} \> $~:$ \> $\Instr \to \sgl{\Label}$ \\
\outlabelsDef{@pull@~c~x~(l,u)~(l',u')}{l,l'}              \\
\outlabelsDef{@drop@~c~(l,u)}{l}                \\
\outlabelsDef{@push@~c~e~(l,u)}{l}              \\
\outlabelsDef{@close@~c~(l,u)}{l}              \\
\outlabelsDef{@case@~e~(l,u)~(l',u')}{l, l'}    \\
\outlabelsDef{@jump@~(l,u)}{l}                  \\
\outlabelsDef{@exit@}{}
\end{tabbing}

\vspace{1ex}
\begin{tabbing}
$\ti{swaplabels}~(@case@~e~((l_1,l_2),u)~((l'_1,l'_2),u'))$ \TABSKIP $=$ \TABSKIP \kill
\ti{swaplabels} \> $~:$ \> $\Instr \to \Instr$ \\
\funClauseDef{swaplabels}
  {@pull@~c~x~((l_1,l_2),u)~((l'_1,l'_2),u')}
  {@pull@~c~x~((l_2,l_1),u)~((l'_2,l'_1),u')}    \\
\funClauseDef{swaplabels}
  {@drop@~c~((l_1,l_2),u)}
  {@drop@~c~((l_2,l_1),u)}      \\
\funClauseDef{swaplabels}
  {@push@~c~e~((l_1,l_2),u)}
  {@push@~c~e~((l_2,l_1),u)}    \\
\funClauseDef{swaplabels}
  {@close@~c~((l_1,l_2),u)}
  {@close@~c~((l_2,l_1),u)}    \\
\funClauseDef{swaplabels}
  {@case@~e~((l_1,l_2),u)~((l'_1,l'_2),u')}
  {@case@~e~((l_2,l_1),u)~((l'_2,l'_1),u')}     \\
\funClauseDef{swaplabels}
  {@jump@~((l_1,l_2),u)}
  {@jump@~((l_2,l_1),u)}  \\
\funClauseDef{swaplabels}
  {@exit@}
  {@exit@}
\end{tabbing}

\caption{Utility functions for fusion}
\label{fig:Fusion:Utils}
\end{figure}



\smallskip
\autoref{fig:Fusion:Utils} contains definitions of some utility functions which we have already mentioned.
Function \ti{channels} computes the $\ChanType_2$ map for a pair of processes.
Function \ti{outlabels} gets the set of output labels for an instruction, which is used when computing the fixpoint of reachable states.
Function \ti{swaplabels} flips the order of the compound labels in an instruction.


% -----------------------------------------------------------------------------
\subsection{Fusibility}
\label{s:FusionOrder}
When we fuse a pair of processes we commit to a particular interleaving of instructions from each process. When we have at least three processes to fuse, the choice of which two to handle first can determine whether this fused result can then be fused with the third process. Consider the following example, where @alt2@ pulls two elements from its first input stream, then two from its second, before pushing all four to its output.
\begin{code}
 alternates : S Nat -> S Nat -> S Nat -> S (Nat, Nat)
 alternates sInA sInB sInC
  = let  s1   = alt2 sInA sInB
         s2   = alt2 sInB sInC
         sOut = zip s1 s2
    in   sOut
\end{code}
If we fuse the two @alt2@ processes together first, then try to fuse this result process with the downstream @zip@ process, the final fusion transform fails. This happens because the first fusion transform commits to a sequential instruction interleaving where two output values \emph{must} be pushed to stream @s1@ first, before pushing values to @s2@. On the other hand, @zip@ needs to pull a \emph{single} value from each of its inputs alternately.

Dynamically, if we were to execute the first fused result process, and the downstream @zip@ process concurrently, then the execution would deadlock. Statically, when we try to fuse the result process with the downstream @zip@ process the deadlock is discovered and fusion fails. Deadlock happens when neither process can advance to the next instruction, and in the fusion algorithm this manifests as the failure of the $tryStepPair$ function from \autoref{fig:Fusion:Def:StepPair}. The $tryStepPair$ function determines which instruction from either process to execute next, and when execution is deadlocked there are none. Fusibility is an under-approximation for \emph{deadlock freedom} of the network.

% On the upside, fusion failure is easy to detect. It is also easy to provide a report to the client programmer that describes why two particular processes could not be fused.

% The report is phrased in terms of the process definitions visible to the client programmer, instead of partially fused intermediate code. The joint labels used in the fusion algorithm represent which states each of the original processes would be in during a concurrent execution, and we provide the corresponding instructions as well as the abstract states of all the input channels.

% This reporting ability is \emph{significantly better} than that of prior fusion systems such as Repa~\cite{lippmeier2012:guiding}, as well as the co-recursive stream fusion of \cite{coutts2007stream}, and many other systems based on general purpose program transformations. In such systems it is usually not clear whether the fusion transformation even succeeded, and debugging why it might not have succeeded involves spelunking\footnote{def. spelunking: Exploration of caves, especially as a hobby. Usually not a science.} through many pages (sometimes hundreds of pages) of compiler intermediate representations.

In practice, the likelihood of fusion succeeding depends on the particular dataflow network. For fusion of pipelines of standard combinators such as @map@, @fold@, @filter@, @scan@ and so on, fusion always succeeds. The process implementations of each of these combinators only pull one element at a time from their source streams, before pushing the result to the output stream, so there is no possibility of deadlock. Deadlock can only happen when multiple streams fan-in to a process with multiple inputs, such as with @merge@.

When the dataflow network has a single output stream then we use the method of starting from the process closest to the output stream, walking towards the input streams, and fusing in successive processes as they occur. This allows the interleaving of the intermediate fused process to be dominated by the consumers, rather than producers, as consumers are more likely to have multiple input channels which need to be synchronised. In the worst case the fall back approach is to try all possible orderings of processes to fuse.



% -----------------------------------------------------------------------------


\section{Proofs}
\label{s:Proofs}

Our fusion system is formalised in Coq, and we have proved soundness of \ti{fusePair}: if the fused result process produces a particular sequence of values on its output channels then the two source processes may also produce that same sequence. Note that due to non-determinism of process execution the converse is not true in practice: just because the two concurrent processes can produce a particular output sequence does not mean the fused result process will as well --- the fused result process uses only one of the many possible orders.

% Note that the converse is not necessarily true: just because two processes can evaluate to a particular output does not mean the fused program will evaluate to that. This is because, as explained in~\autoref{s:EvaluationOrder}, evaluation of a process network is non-deterministic, and fusion commits to a particular evaluation order.

\begin{coq}
Theorem Soundness (P1 : Program L1 C V1) (P2  : Program L2 C V2)
                  (ss : Streams)         (h   : Heap)
                  (l1 : L1)              (is1 : InputStates)
                  (l2 : L2)              (is2 : InputStates)
  :  EvalBs (fuse P1 P2) ss h (LX l1 l2 is1 is2)
  -> EvalOriginal Var1 P1 P2 is1 ss h l1
  /\ EvalOriginal Var2 P2 P1 is2 ss h l2.
\end{coq}

The @Soundness@ theorem uses @EvalBs@ to evaluate the fused program, and @EvalOriginal@ ensures that the original program evaluates with that program's subset of the result heap, using @Var1@ and @Var2@ to extract the variables.
The @Streams@ type corresponds to the channel value map used to accumulate stream elements while feeding a process network (\autoref{fig:Process:Eval:Feed}), and the @Heap@ type corresponds to the value store used while advancing a single process (\autoref{fig:Process:Eval:Shake}).

% Care must be taken to remove stream values that the other process has pulled but this one has not yet.
%%% AR: Really would like to say something about this but no room to explain properly
% For shared inputs when one program has pulled, the other program must be evaluated with the other value removed from the end of the stream.

To aid mechanisation, the Coq formalisation has some small differences from the system presented earlier in this thesis.
Firstly, the Coq formalisation uses a separate @update@ instruction to modify variables in the local heap, rather than attaching heap updates to the \Next~ label of every instruction.
Performing this desugaring makes the low level lemmas easier to prove, but we find attaching the updates to each instruction makes for an easier exposition.
This causes the fusion definition to be slightly more complicated, as two output instructions must be emitted when performing a push or pull followed by an update.
This is a fairly minor difference.

Secondly, the formalisation only implements sequential evaluation for a single process, rather than non-deterministic evaluation for whole process groups.
Instead, we sequentially evaluate each source processes independently, and compare the output values to the ones produced by sequential evaluation of the fused result process.
This is sufficient for our purposes because we are mainly interested in the value correctness of the fused program, rather than making a statement about the possible execution orders of the source processes when run concurrently.

Like the earlier presentation, each program has a mapping from labels to instructions.
We also associate each label with a precondition, which is expressed as a predicate of the evaluation state.
The precondition for the initial label must be true for the initial evaluation state.
Whenever the program takes an evaluation step, assuming the precondition for the original label was satisfied at the start of the step, the precondition for the result label must be satisfied by the updated evaluation state.
With these two conditions satisfied, we can show that all evaluations respect the preconditions by performing induction over the evaluation relation.

% To prove soundness of fusion, we assert the invariant that defines how evaluation of the fused program relates to evaluation of the two original programs. 
% For the fused process, the precondition is that, when the fused process evaluates to a particular evaluation state, then each original process must
\TODO{example evaluation, show how evaluation relation works with @pending@ values}

\TODO{also mention that formalisation is infinite streams, rather than finite}
