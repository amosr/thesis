\chapter{Implementation and code generation}
\label{chapter:process:implementation}


We have implemented this fusion system \REFTODO{the one described in chapter} in a library called @folderol@\footnote{\url{https://github.com/amosr/folderol}}.
Rather than modifying the compiler to support our particular fusion system, we use Template Haskell, which allows us to perform code-generation at compile-time: a form of macro metaprogramming.

This chapter explains how to generate code for processes, while taking advantage of the existing optimisations in the Glasgow Haskell Compiler (GHC).
Generating code for a single process is fairly straightforward in itself; the process language is a simple imperative language, and once the process network has been fused into a single process there is no longer any need for threading or inter-process communication.
Haskell supports imperative language constructs like mutable references, but being a functional language, the optimisations in GHC are more geared towards functional programs.
If we wish to get good performance, we must generate code closer to what the compiler expects.

\section{Template Haskell}
Template Haskell is a metaprogramming extension for Haskell \cite{sheard2002template}.
Template Haskell can be thought of as an untyped, limited form of staged computation, where the only `stages' are compile-time and run-time.
There is also a typed variant, which we will use for defining a typesafe interface, while using the untyped variant for code generation.
Let us look at the untyped variant first.

Template Haskell extends regular Haskell with two syntactic constructs: quasiquoting and splicing.
Quasiquoting converts the syntactic form of an expression to a value-level abstract syntax tree representation.
For example, we can quasiquote some simple arithmetic using the syntax \lstinline/[|1 + 2|]/.
Splicing is in some ways the opposite, as it takes a value-level abstract syntax tree, evaluates it at compile-time and converts it back to an expression which can be run at runtime.
For example, we can splicing back the quasiquoted arithmetic using the syntax \lstinline/$([|1 + 2|])/, which will evaluate (@1 + 2@) at runtime.

Splicing and quasiquoting operate in the @Q@ monad.
The @Q@ monad gives a fresh name supply, so that when names are bound inside expressions they can be given unique names.
This way newly bound names will not interfere with other bindings.

Let us look at a simple example of how to use Template Haskell.
Suppose we wish to define a power function where the exponent is known at compile time, but the mantissa is not known until runtime.

We define @power@ as a function with the type \lstinline/Int ->  Q Exp/.
This means it takes an integer at compile-time, and produces a quoted expression in the quote monad.
This expression will, once spliced in, have type \lstinline/Int ->  Int/.

\begin{lstlisting}
power :: Int -> Q Exp
power 0 = [|\i -> 1                      |]
power n = [|\i -> @$(power (n - 1))@ i * i |]
\end{lstlisting}

The power function pattern matches on the exponent.
When it is zero, we enter quasiquoting mode and construct a function that always returns one.
When the exponent is non-zero, we again enter quasiquoting mode, and construct a function.
Inside the quasiquote, we need to handle the recursive case, so we enter into splicing mode with \lstinline/@$(power (n - 1))@/ to compute the one-smaller power, which returns the function expression, to which we apply @i@. Finally, we multiply the smaller power of @i@ with @i@ itself.
This function does not handle negative exponents: it is just to show the use of staging.

We can then, in another module, define a specialised power function that computes the square.
We define this as a top-level binding, by performing a splice, and inside that splice we call @power@ with the statically known at compile-time argument @2@.
The type inside the splice is \lstinline/Q Exp/, and once splicing is complete the result is an expression of type \lstinline/Int ->  Int/.
% After this is spliced in, it unwraps the quote computation and expression and we end up reifying it to a real function of \lstinline/Int -> Int/.

\begin{lstlisting}
power2 :: Int -> Int
power2 = @$(power 2)@
\end{lstlisting}

This is a ``top-level splice'', because it is not inside a quasiquotation.
Top-level splices can only refer to bindings imported from other modules, not ones defined locally.
This is why it needs to be in a different module.
This is just a silly restriction to be aware of. Is it even worth mentioning?

If we turn on the compiler option @-ddump-splices@ we can view the resulting code.
The names of variables have changed slightly for for readability.

\begin{lstlisting}
Splicing expression
    power 2 => \i0 -> (\i1 -> (\i2 -> 1) i1 * i1) i0 * i0
\end{lstlisting}

It's a pretty roundabout way to multiply a number by itself. There are a lot of opportunities for simplifying that code. And while GHC should be able to remove these, it would be better to not introduce them in the first place.
So let us fix it.
The function @powerS'@ (@S@ for \emph{simplified}) takes the argument as a `real' argument, rather than returning an expression of function type: its type is \lstinline/Int ->  Q Exp ->  Q Exp/.
We have moved the function from later (expression) to now (value), but the argument is still a quoted expression.

\begin{lstlisting}
powerS' :: Int -> Q Exp -> Q Exp
powerS' 0 i = [|1                        |]
powerS' n i = [|@$(powerS' (n-1) i)@ * @$(i)@|]
\end{lstlisting}

The definition of @powerS@ introduces the lambda binding as before, but passes the expression of this binding to the worker function.
We enter quoting mode, then introduce a lambda. Then we go back into splicing mode, in order to call the helper function @powerS'@.
Then, in order to pass the mantissa @i@ to the helper function, we need to go back into quoting mode.
Note that the quoted expresssion \lstinline/[|i|]/ is \emph{open}: it refers to bindings outside the environment.
This means if you somehow kept a hold of that expression and used it in a different context, outside the lambda binding, it would be incorrect.
So you can construct bad, ill-typed expressions with Template Haskell. This is certainly not ideal, but is just something to be aware of.

Is it worth noting that MetaOCaml has similar problems, including BER MetaOCaml?
You don't need to rant about let-insertion in MetaOCaml: how it is allegedly safe, but it is not \emph{type} safe.

\begin{lstlisting}
powerS :: Int -> Q Exp
powerS n = [|\i -> @$(powerS' n [|i|])@|]
\end{lstlisting}

The output is a lot simpler now.
Whereas before there was a lambda introduced and applied at each recursive step, now there is only a single lambda.

\begin{lstlisting}
Splicing expression
    powerS 2 => \i -> ((1 * i) * i)
\end{lstlisting}

There is still more we could do to improve the function, for example removing the multiplication by one, but this is sufficient to show the core splicing and quoting ideas behind Template Haskell.
For more information on staging in general, \citet{rompf2010lightweight} takes this example further.

Meta-Repa similar idea but for flat data parallel computations, not really for streaming computations \cite{ankner2013edsl}.
But it also uses Template Haskell, so that's worth mentioning.

\subsection{Typed expressions}

One of the problems with Template Haskell shown above is that there are no types attached to expressions.
Quasiquoting a string \lstinline/[|"one"|]/ and quasiquoting an integer \lstinline/[|1|]/ produce a value of the same type: @Q Exp@.

The original Template Haskell paper  doesn't include \emph{typed} Template Haskell, which is what we really want to talk about.
In the original paper, generated expressions don't have types.
Typechecking does still occur on the expressions, though, but only after they have been spliced together.
This is less of a problem for generation alone --- if you trust the generation code.
But when you want higher-order templates that take expressions as arguments, you want to be able to make sure your caller gives you an expression of the right type.
Otherwise the type error will be very well hidden, somewhere inside the generated code.
For the user, having to figure out where their input expression ended up inside the generated code, is a real hassle.
So the typed Template Haskell attaches a type argument to each expression: \lstinline|TExp t| is an expression that, when evaluated \emph{probably} returns a value of type \lstinline|t|.

We can use typed quasiquoting and typed splicing.
Typed quasiquoting uses the same syntax except with two pipes, for example \lstinline/[||"one"||]  :: Q (TExp String)/.
Similarly, typed splicing uses two dollar signs: \lstinline/$$(f) :: String/, if \lstinline/f :: Q (TExp String)/.

We can also go between the typed and untyped representation.
One can convert from a typed expression to an untyped expression.
This does not affect the expression itself, and once it is spliced in it will have the same type.
It just throws away some type-level information.

\begin{lstlisting}
unTypeQ :: Q (TExp a) -> Q Exp
\end{lstlisting}

Conversely, if one is very careful, one can convert an untyped expression to a typed one.
This is an unsafe operation, because one can choose any type at all for the expression.
The expression is not checked against the chosen type until it is spliced in.
\begin{lstlisting}
unsafeTExpCoerce :: Q Exp -> Q (TExp a)
\end{lstlisting}


\section{All this boxing and unboxing}
In Haskell, most values are boxed by default \citep{jones1991unboxed}.
Boxed values are stored as pointers to heap objects, which can in turn reference other boxed or unboxed values.
Boxed values are useful for implementing parametric polymorphism because they give a uniform representation to all the different types.
A list which is polymorphic in its element type can use a pointer to refer to its values regardless of the actual element type.

The problem with boxed values is that they require an extra allocation per object, as well as a pointer indirection for each access.
This can cause performance issues in tight loops, particularly because most objects are immutable.

Consider the following function, which loops over an array to compute its sum.
It starts by calling the local function @loop@ with the initial loop index, and the initial sum.
The definition of @loop@ checks if it has reached the end of the array, and if so returns the sum; otherwise it increments the running sum and proceeds to the next index.

\begin{lstlisting}
sum :: Vector Int -> Int
sum vector = loop 0 0
 where
  loop index running_sum
   | index == length vector
   = running_sum
   | otherwise
   = let value = vector ! index
     in loop (index + 1) (running_sum + value)
\end{lstlisting}

It is not clear from the program source alone, but the loop index and the running sum are both boxed values, because their type (@Int@) is boxed.
If this were compiled naively it would be rather disastrous for performance, as in order to process each element of the array, it must allocate two new boxed values.
Hilariously, all of these new boxed values except the very last iteration are used once by the next iteration and then thrown away.
While the garbage collector is tuned for small, short-lived objects, it is better to not introduce the garbage in the first place.

Removing boxing is not a novel thing, and there are many ways to do this.
The point to make is not that this is an interesting thing, just that we must know how it works in order to generate good code that fits it.
Boxed machine-word integers are represented by the following type, which defines @Int@ with a single constructor @I#@, taking an unboxed integer @Int#@. By convention, unboxed values and constructors that use them are named with the @#@ suffix.

\begin{lstlisting}
data Int = I# Int#
\end{lstlisting}

Now we know how machine-word integers are represented, we can look at an explicitly boxed version of @sum@.
This is still using boxed integers, but all arithmetic is explicitly unboxing and reboxing.
Unboxed literals are written as @0#@ or @1#@.
Unboxed arithmetic are written as @+#@ or @==#@, and @!#@ for unboxed indexing.
With explicit boxing, it should now be visible that the recursive call to @loop@ constructs new boxed integers.

\begin{lstlisting}
sum :: Vector Int -> Int
sum vector = loop (I# 0#) (I# 0#)
 where
  loop (I# index) (I# running_sum)
   | index ==# length vector
   = I# running_sum
   | otherwise
   = let value = vector !# index
     in loop (I# (index +# 1#)) (I# (running_sum +# value))
\end{lstlisting}

Constructor specialisation \cite{peyton2007call} is a loop optimisation that can remove these boxed arguments to recursive calls.
It looks at the constructors to recursive calls, and counts which ones are scrutinised or unwrapped at the start of the function definition.
In this case, @loop@ is first (and later, as well) called with the constructors @I#@ for both arguments, and both arguments are scrutinised.
So it creates a specialised version of @loop@ where both arguments are @I#@ constructors.
This specialised version is the same as the original, except the arguments are known to be @I#@ constructors, which means the pattern matching on the arguments can be simplified away.
We will call this specialised version @loop'I#'I#@.
Then everywhere that @loop@ is called with @I#@ constructors, it will be replaced with a call to @loop'I#'I#@.
So any function call that looks like (@loop (I# x) (I# y)@) is replaced by a call to our new function (@loop'I#'I# x y@).

\begin{lstlisting}
sum :: Vector Int -> Int
sum vector = loop'I#'I# 0# 0#
 where
  loop'I#'I# index running_sum
   | index ==# length vector
   = I# running_sum
   | otherwise
   = let value = vector !# index
     in loop'I#'I# (index +# 1#) (running_sum +# value)
\end{lstlisting}

Constructor specialisation has removed all the boxing except for the final return value, which is only constructed once anyway.
In this example, the original @loop@ function was no longer called, so it was able to be removed entirely.
It is not always the case that the original function can be removed, and constructor specialisation can duplicate the code many times: once for each combination of constructors.
This can cause quite a lot of copies of the original function, which can cause large intermediate programs that do not fit in memory.
To alleviate this, GHC implements some heuristics to limit the number of duplicates created, as well as only creating specialisations if the original function is not too large.
This makes sense for general purpose code, but for tight loops where we expect most of our runtime to be, we really want to be sure that all specialisations are created.
For tight loops, we want to \emph{force} constructor specialisation to occur as much as possible.
This is achieved by annotating the function to be specialised with the special constructor @SPEC@.
Going back to the original @sum@ function, if we want to force constructor specialisation on @loop@, we can do this by adding the @SPEC@ to the function binding as well as all calls to it:

\begin{lstlisting}
sum :: Vector Int -> Int
sum vector = loop SPEC 0 0
 where
  loop SPEC index running_sum
   | index == length vector
   = running_sum
   | otherwise
   = let value = vector ! index
     in loop SPEC (index + 1) (running_sum + value)
\end{lstlisting}

\subsection{Mutable references}
\label{ss:extraction:mutablerefs}

We have seen that GHC is able to eliminate boxing from function arguments, and we will take advantage of this during code generation.
We will make use of @SPEC@ to force constructor specialisation, to ensure as much can be unboxed as possible.
Sadly, mutable references are stored boxed, and an analogous constructor specialisation transform does not exist for mutable references.
This means that in order to get unboxed values, we must structure our generated code to pass values via function arguments instead of mutable references.

Unboxed mutable references do exist, but are unsuitable because they can \emph{only} store unboxed values.
Recursive types such as linked lists cannot be stored in unboxed references.
We desire an unboxed representation when possible, and boxed representation when necessary.

It may be surprising to users of other languages that we should move away from using mutable references in favour of function arguments.
Indeed, \citet{biboudis2017expressive} describes the \emph{opposite} transform when implementing stream fusion in MetaOCaml.
So this will not necessarily map to other languages, but it is true in the particular case of GHC.

In Data Flow Fusion \cite{lippmeier2013data} there is a transform called \emph{loop winding}, which converts mutable references to function arguments.
The motivation here is that GHC does not track aliasing information of arrays stored in mutable references, but does track it for arrays as function arguments.

\subsection{Extended constructor specialisation}

Constructor specialisation is not limited to boxing and unboxing, but works for arbitrary constructors, including types with multiple constructors such as (@Maybe a@) or (@Either a b@).
It even works for recursive types such as lists, which could produce an an infinite number of specialisations.
Constructor specialisation must be careful to limit the specialisations to a finite number of \emph{useful} ones.

Information about the initial state can be very helpful in finding the most specific call patterns.
In the following example, @go@ is first called with the call pattern (@go (Just _) (Just _)@).
Using this as the `seed' from which we start exploring, we can see that the initial call pattern proceeds to the next call pattern (@go Nothing (Just _)@), followed by a call to (@go Nothing Nothing@).
If we were to look at the body of @go@ without this initial seed, however, we would find the call patterns (@go Nothing _@) and (@go _ Nothing@).
These call patterns from the unseeded body are less specific than the call patterns for the seed, which means using them would not allow the second argument to be specialised away.
By starting from the initial seed, the extra information about the initial state can be propagated to the other states.

\begin{lstlisting}
initial = go (Just 1) (Just 2)
 where
  go (Just _) b       = go Nothing b
  go a       (Just _) = go a       Nothing
  go Nothing Nothing  = 0
\end{lstlisting}

Not all specialisations are useful.
To limit compilation time, memory usage and code blowup, it is important to limit the specialisations to those which will be used.
That is, \emph{only} those which are reachable from the initial state.
In the following example, the initial state is the call pattern (@go (Left _) (Right _)@).
At each step, the arguments are flipped, so from the initial state the next reachable call pattern is (@go (Right _) (Left _)@).
From here, we can get back to the original state.
This means in total there are only two reachable call patterns.

However, if we look at the body alone without the seed, the first two call patterns are (@go _ (Left _)@) and (@go _ (Right _)@).
From here, more call patterns can be found: (@go _ (Left _)@) calls (@go (Left _) (Left a)@) and (@go (Left _) (Right _)@).
Similarly, there are two call patterns reachable from @(go _ (Right _)@), and these are distinct from the two already seen.
In this way, starting from the initial state means we do not have to generate all the possible specialisations.

\begin{lstlisting}
reachable = go (Left 1) (Right 2)
 where
  go (Left  a) b = go b (Left  a)
  go (Right a) b = go b (Right a)
\end{lstlisting}

TODO: diagrams.

Using the initial calls as the seed is important.
This \emph{was} implemented, but it only occurred for locally bound functions, not for top-level bindings.
The problem is that even though our examples were locally bound functions, other transforms such as let-floating occur before constructor specialisation, which means locally bound functions can be `floated' up to top-level bindings, where seeding does not work.
The other issue is that top-level bound functions can be exported; if functions are exported, we cannot know their initial call pattern, as they may be called from other modules.
So for exported top-level functions, we must seed the call-patterns using all initial calls in the current module, as well as those in the body.
For non-exported top-level functions, we can be sure that the initial state is in the current module, and so use any initial calls outside of the body as the seed.

When arguments are of recursive types, there can be an infinite number of reachable call patterns.
Suppose we wish to reverse a linked list.
We can write this using a helper function, which takes the list that is reversed so far, as well as the list to reverse.

\begin{lstlisting}
reverse :: [Int] -> [Int]
reverse xs0 = go [] xs0
 where
  go zs []     = zs
  go zs (x:xs) = go (x:zs) xs
\end{lstlisting}

The helper function @go@ could be specialised an infinite number of times, but this would lead to non-terminating compilation.
First, the call to @go@ is seeded with the call pattern (@go [] _@).
Then, at every step in the evaluation, a list constructor is moved from the second argument to the first, resulting in the infinite chain of call patterns, (@go [_] _@), (@go [_, _] _@), and so on.
Usually, these specialisations would not be produced because they do not reduce allocation.
However, in the original implementation, when @SPEC@ is used to force constructor specialisation, an infinite number of specialisations were produced, and the compiler did not terminate.
I implemented Roman Leshchinskiy's suggestion to fix this by setting a limit on how many times recursive types can be specialised, even when forcing constructor specialisation.

\section{Sources and Sinks}
In order to write meaningful streaming computations, we need to interact with the outside world.
The processes in our process networks are pure and have no way of interacting with the world: they simply shuffle data along channels.
At the start of the process network, for the inputs, we use \emph{sources} to pull from the outside, such as reading from a file.
At the end of the process network, for the outputs, we use \emph{sinks} to push to the outside, such as writing to a file.

These sources and sinks are really the pull and push streams we have seen before \REFTODO{pull and push}, but in this case we do not need to implement combinators over them; such plumbing will be expressed as processes in the process network.
Sources and sinks are in many ways opposites of each other, but they also share many similarities, so let us refer to them collectively as \emph{endpoints}.

Endpoints need to encapsulate some internal state: for example writing to a file requires a filehandle, and perhaps a buffer to fill before writing, to amortise the cost of the system call.
As explained previously (\autoref{ss:extraction:mutablerefs}), using mutable references for this internal state would lead to poor performance due to boxing.
We need to use the same approach of passing this state as function arguments so they can be unboxed by constructor specialisation.
This is a tad more complicated than it sounds, because the each endpoint requires a different type of state: reading from a file requires a filehandle, while reading from an in-memory array requires the array and the current index.
On the other hand, this state type is an internal thing and should not be exposed to the user, which rules out adding it as a type parameter on the endpoint.
Just because we need to pass the state around as function arguments should not change the external interface.

In order to `wrap up' the internal state type, so only the endpoint itself can inspect the internal state, while the user can only hold on to the abstract state and pass it to the endpoint, we use existentially quantified types.
This is similar to how existential types are used in Stream Fusion \cite{coutts2007stream}, to hide the internal state of a pull stream.

We say that each endpoint has an internal state type, and only it knows what the type is.
We provide some operations with the state: a way to construct an initial state, for example opening the file and returning the handle; a pull or push function which takes the state and returns a new state; and a close function for when we have finished reading from or writing to the endpoint.

\subsection{Sources}

We define sources in Haskell with the following datatype (@Source a@), where the type parameter @a@ is the type of values to be pulled.
The internal state type is bound to @s@, and we define a record with three fields.
The first field, @sourceInit@, contains an effectful computation which returns the initial state.
The second field, @sourcePull@, is a function which takes the current state and returns a pair containing the pulled value, and the updated state.
The pulled value is wrapped in a @Maybe@, because streams are finite: @Nothing@ means the end of the stream, and (@Just v@) means the value @v@.
Streams end only once, and after pulling a @Nothing@, the source should not be pulled on again.
The third and final field, @sourceDone@, is a function which takes the current state and closes the stream.
The source should not be pulled again after it is closed, but these invariants are not checked.

\begin{lstlisting}[mathescape=true]
data Source a
 = $\exists$s. Source
 { sourceInit ::      IO s
 , sourcePull :: s -> IO (Maybe a, s)
 , sourceDone :: s -> IO ()
 }
\end{lstlisting}

We can define a @Source@ that reads lines of text from a file.
Here the internal state is simply a filehandle.
To initialise the source, we open the file in reading mode, with the @openFile@ function.
When the source is done, we close the file handle with @hClose@.
To pull from the source, we define a helper function @pull@, which takes the filehandle as an argument.
The @pull@ function checks whether the end of the file has been reached (@hIsEof@).
If the end of the file, it returns @Nothing@.
Otherwise, it reads a line from the handle (@hGetLine@) and wraps the line in a @Just@ constructor.

\begin{lstlisting}
sourceReadFile :: FilePath -> Source String
sourceReadFile filepath
  = Source
  { sourceInit = openFile ReadMode filepath
  , sourcePull = pull
  , sourceDone = hClose }
 where
  pull handle = do
    eof <- hIsEof
    case eof of
     True  -> return Nothing
     False -> Just <$> hGetLine handle
\end{lstlisting}

For the sake of example, this is a simplified version.
Certainly, this could be improved in terms of error handling: what if the file does not exist; and performance: reading a single line at a time will not give the best performance.

\subsection{Sinks}

\begin{lstlisting}[mathescape=true]
data Sink a
 = $\exists$s. Sink
 { sinkInit ::           IO s
 , sinkPush :: s -> a -> IO s
 , sinkDone :: s ->      IO ()
 }
\end{lstlisting}


\section{Example}

Look at a simple example first: we just want to read some file and do some things.
It doesn't matter what.
\begin{lstlisting}
mapFilter :: FilePath -> FilePath -> IO ()
mapFilter fileIn fileOut = do
  @$$(fuse $ do
     ins    <- source @[||sourceOfFile fileIn||]@
     above  <- filter @[||\i -> i > 0        ||]@ ins
     double <- map    @[||\i -> i * 2        ||]@ above
     sink double      @[||sinkToFile fileOut ||]@)@
\end{lstlisting}

We start with the Template Haskell splice \verb/$$(fuse ...)/. In the code it is blue. 
It has the following type.
\begin{lstlisting}
fuse :: Monad m => Network m () -> Q (TExp (m ()))
\end{lstlisting}
That is, it takes a process network with underlying monad @m@ and returns the expression for the underlying computation in the monad @m@.
The process network @Network m ()@ is a monad as well.

The process network first constructs a source that reads from a file.
\begin{lstlisting}
source :: Q (TExp (Source m a))              -> Network m (Channel a)
filter :: Q (TExp (a -> Bool))  -> Channel a -> Network m (Channel a)
map    :: Q (TExp (a -> b))     -> Channel a -> Network m (Channel b)
sink   :: Q (TExp (Sink m a))   -> Channel a -> Network m (Channel a)
\end{lstlisting}
The @source@ function takes a quasiquoted expression of how to construct the source at runtime.

Now show the generated code.

\section{Size hints}
\label{s:implementation:sizehints}
Talk about @vectorSizeIO@ and why it's useful.
Reference \autoref{s:Future:SizeInference} for how to infer this.

% -----------------------------------------------------------------------------
\subsection{Optimisation}
\label{s:Optimisation}
After we have fused two processes together, it may be possible to simplify the result before fusing in a third. Consider the result of fusing @group@ and @merge@ which we saw back in Figure~\ref{fig:Process:Fused}. At labels @F1@ and @F2@ are two consecutive @jump@ instructions.
The update expressions attached to these instructions are also non-interfering, which means we can safely combine these instructions into a single @jump@.
In general, we prefer to have @jump@ instructions from separate processes scheduled into consecutive groups, rather than spread out through the result code.
The (PreferJump) clauses of Figure~\ref{fig:Fusion:Def:StepPair} implement a heuristic that causes jump instructions to be scheduled before all others, so they tend to end up in these groups.

Other @jump@ instructions like the one at @F5@ have no associated update expressions, and thus can be eliminated completely. Another simple optimization is to perform constant propagation, which in this case would allow us to eliminate the first @case@ instruction. 

Minimising the number of states in an intermediate process has the follow-on effect that the final fused result also has fewer states. Provided we do not change the order of instructions that require synchronization with other processes (@pull@, @push@ or @drop@), the fusibility of the overall process network will not be affected.

Another optimization is to notice that in some cases, when a heap variable is updated it is always assigned the value of another variable. In Fig.\ref{fig:Process:Fused}, the @v@ and @x1@ variables are only ever assigned the value of @b1@, and @b1@ itself is only ever loaded via a @pull@ instruction. Remember from \S\ref{s:Fusion:FusingPulls} that the variable @b1@ is the stream buffer variable. Values pulled from stream @sIn1@ are first stored in @b1@ before being copied to @v@ and @x1@. When the two processes to be fused share a common input stream, use of stream buffer variable allows one process to continue using the value that was last pulled from the stream, while the other moves onto the next one. 


% When the two processes are able to accept the next variable from the stream at the same time, there is no need for the separate stream buffer variable. This is the case in Figure~\ref{fig:Process:Fused}, and we can perform a copy-propagation optimisation, replacing all occurrences of @v@ and @x1@ with the single variable @b1@. To increase the chance that we can perform copy-propagation, we need both processess to want to pull from the same stream at the same time. Moving the @drop@ instruction for a particular stream as late as possible prevents a @pull@ instruction from a second process being scheduled in too early.
% To increase the chance that we can perform this above copy-propagation, we need both processess to want to pull from the same stream at the same time. In the definition of a particular process, moving the @drop@ instruction for a particular stream as late as possible prevents a @pull@ instruction from a second process being scheduled in too early. In general, the @drop@ for a particlar stream should be placed just before a @pull@ from the same stream. 

\section{What's the deal with drop anyway}
\TODO{elsewhere}

The purpose of the @drop@ instructions is to keep two consumers more closely synchronised.
One consumer cannot start processing the next element until the other has finished processing the current element.
Drop is not necessary for correctness, or even for ensuring boundedness of buffers: without it, the result program would still be correct, and one process could not `overtake' another by processing more than one element before the other one.

If we have two consumers @P@ and @Q@, both pulling from the same channel @C@, they both need to agree about when to pull from the channel.
Suppose that @P@ and @Q@ both pull from @C@, then push to their own output channel, @CP@ and @CQ@ respectively, drop their input, then loop back to pull again.
If we are executing @P@, and it pulls from @C@, it can keep executing and push to its output channel @CP@, then drop the input.
Now, @P@ has dropped its input, but cannot pull again yet, because there is no new value available to pull.
There is no new value available to pull because the producer cannot \emph{push} to @C@ yet, because @Q@ has not consumed its input.
Now, @Q@ can run, and pulls from its input.
This transitions its input from \emph{pending} to \emph{have}, which means the producer still cannot push yet, until @Q@ drops its input.
In this way, the drop allows the the producer to push only once all consumers have dealt with their input.
Without drops, @P@ would be able to process the next element before @Q@ had finished the previous one.

\begin{lstlisting}
P = process
  P1: c <- pull C
  P2: push CP c
  P3: drop C
  P4: jump P1

Q = process
  Q1: c <- pull C
  Q2: push CQ c
  Q3: drop C
  Q4: jump Q1
\end{lstlisting}

By synchronising the two processes together, when we fuse we will only have one copy of the code that pulls each element.
Because @P@ can only start pulling again by the time @Q@ has dropped, this means @P@ and @Q@ must both be trying to pull at the same time, which means we can reuse the same instructions generated from the previous time they both pulled.
The example @PQ_drop@ shows the fused process with drop instructions.
Note that there is only one copy of each input process' code.
On the other hand, @PQ_no_drop@ shows the fused process without drop instructions.
Here, there are two copies of pushing to @CP@, though the main loop only executes one per iteration: pushing the current element to @CP@, and the previous element to @CQ@.
As the processes get larger, and more processes are fused together, the issue of duplicating code becomes more serious.
There are two parts to this: first, we need to hold the entire process in memory in order to generate its code.
Secondly, as the generated assembly code gets larger, it is less likely to fit into the processor's cache.
Smaller code is generally better for performance.
Having two copies of the push to @CP@ means that any consumers of @CP@ must in turn have their code duplicated, with the pull instructions from @CP@ copied into both sites of the pushes.

\TODO{diagrams}
\begin{lstlisting}
PQ_drop = process
  P1Q1: c_buf <- pull C
        c_p    = c_buf
        c_q    = c_buf
        push CP c_p
        push CQ c_q
        drop C
        jump P1Q1

PQ_no_drop = process
  P1Q1:  c_buf <- pull C
         c_p    = c_buf
         c_q    = c_buf
         push CP c_p
  P2Q2:  c_buf <- pull c
         c_p    = c_buf
         push CP c_p
         push CQ c_q
         c_q    = c_buf
         jump P2Q2
\end{lstlisting}




\section{Untyped interface}
Talk about how convert from @TExp a@ to @Exp@, and why.
Our processes can have arbitrarily many input channels and output channels, which makes it hard to construct a type-safe process, since each channel can be a different type.
We would need a list of types --- a type-level list --- for the type of each input channel, and another list for the output channels.
Each process also has a store (heap) for local variables, and these would need a different type for each variable as well.

These problems are not insurmountable: one can use type-level lists and fake dependent types in Haskell \CITE, but this is not really the point of this thesis.
We take a simpler tack.
Instead, we provide an untyped core for the processes and networks, then build on top of this to provide a typed interface for constructing networks.

Type safety.
This is a similar idea as our untyped core providing a ``trusted base''.
One must be very careful that it is indeed safe, but this is not exposed to the user, and is small enough to reason about by hand.
Also, the code generation will still end up being typechecked by the Haskell compiler.
We are not losing any actual typesafety, then, because the code will still eventually be typechecked.
What we are losing is good error locations. If there are bugs in the untyped core, the errors will show up somewhere in the generated code.
This may make it hard to track down the original error location.
But if the core is correct, this should not happen.
So this is the idea at least.

\section{Instructions}

\begin{lstlisting}
data Info
 = Info
 { infoBindings     :: Set Var
 , infoInstruction  :: Instruction
 }

data Instruction
 = I'Pull Channel Var Next Next
 | I'Push Channel Haskell.Exp Next
 | I'Jump Next
 | I'Bool Haskell.Exp Next Next
 | I'Drop Channel Next
 | I'Done

data Next
 = Next
 { nextLabel    :: Label
 , nextUpdates  :: Map Var Haskell.Exp
 }
\end{lstlisting}

\begin{lstlisting}
data Process
 = Process
 { pName          :: [Char]
 , pInputs        :: Set Channel
 , pOutputs       :: Set Channel
 , pInitial       :: Next
 , pInstructions  :: Map Label Info
 }
\end{lstlisting}
