%!TEX root = ../Main.tex

% -----------------------------------------------------------------------------
\begin{figure}
\begin{tabbing}
MM \= MMMMMMM \= MM \= MMMMMMMMM\kill
\end{tabbing}


% -----------------------------
$$
  \boxed{
    \ProcBlockShake
      {\Instr}
      {\ChannelStates}
      {\Heap}
      {\Action}
      {\Label}
      {\ChannelStates}
      {\Updates}
  }
$$

$$
\ruleIN{
  is[c] = \lstiproc@pending@~v
}{
  \ProcBlockShake
        {\lstiproc@pull@~c~x~(l,us)~(l',us')}
        {is}
        {\Sigma}
        {\cdot}
        {l}
        {is[c \mapsto \lstiproc@have@]}
        {(us, x = v)}
}{PullPending}
$$

$$
\ruleIN{
  is[c] = \lstiproc@closed@
}{
  \ProcBlockShake
        {\lstiproc@pull@~c~x~(l,us)~(l',us')}
        {is}
        {\Sigma}
        {\cdot}
        {l'}
        {is}
        {us'}
}{PullClosed}
$$

$$
\ruleIN{
  \ExpEval{\Sigma}{e}{v}
}{
  \ProcBlockShake
        {\lstiproc@push@~c~e~(l,us)}
        {is}
        {\Sigma}
        {\Push~c~v}
        {l}
        {is}
        {us}
}{Push}
$$

$$
\ruleAx{
  \ProcBlockShake
        {\lstiproc@close@~c~(l,us)}
        {is}
        {\Sigma}
        {\Close~c}
        {l}
        {is}
        {us}
}{Close}
$$


$$
\ruleIN{
  is[c] = \lstiproc@have@
}{
  \ProcBlockShake
        {\lstiproc@drop@~c~(l,us)}
        {is}
        {\Sigma}
        {\cdot}
        {l}
        {is[c \mapsto \lstiproc@none@]}
        {us}
}{Drop}
\ruleIN{
}{
  \ProcBlockShake
        {\lstiproc@jump@~(l,us)}
        {is}
        {\Sigma}
        {\cdot}
        {l}
        {is}
        {us}
}{Jump}
$$

$$
\ruleIN{
  \ExpEval{\Sigma}{e}{\lstiproc@True@}
}{
  \ProcBlockShake
        {\lstiproc@case@~e~(l_t,us_t)~(l_f,us_f)}
        {is}
        {\Sigma}
        {\cdot}
        {l_t}
        {is}
        {us_t}
}{CaseT}
$$
$$
\ruleIN{
  \ExpEval{\Sigma}{e}{\lstiproc@False@}
}{
  \ProcBlockShake
        {\lstiproc@case@~e~(l_t,us_t)~(l_f,us_f)}
        {is}
        {\Sigma}
        {\cdot}
        {l_f}
        {is}
        {us_f}
}{CaseF}
$$

\vspace{2em}

% ----------------
$$
  \boxed{\ProcShake{\Proc}{\Action}{\Proc}}
$$
$$
\ruleIN{
  \ProcBlockShake
    {p[\lstiproc@instrs@][p[\lstiproc@label@]]} 
    {p[\lstiproc@ins@]}
    {p[\lstiproc@heap@]}
    {a}
    {l}
    {is}
    {us}
  \qquad
    \ExpEval{p[\lstiproc@heap@]}{us}{bs}
}{
  \ProcShake
        {p}
        {a}
        {p~[    \lstiproc@label@~ \mapsto ~l
           , ~~ \lstiproc@heap@~  \mapsto (p[\lstiproc@heap@] \lhd bs)
           , ~~ \lstiproc@ins@~   \mapsto ~is]}
}{Advance}
$$

% ---------------------------------------------------------
\vspace{2em}

$$
  \boxed{\ProcsShake{\sgl{\Proc}}{\Action}{\sgl{\Proc}}}
$$

$$
\ruleIN{
  \ProcShake{p_i}{a}{p'_i}
  \qquad
  \forall j~|~j \neq i.~
  \ProcInject{p_j}{a}{p'_j}
}{
  \ProcsShake{
    \sgl{p_0 \ldots p_i \ldots p_n}
  }{a}{
    \sgl{p'_0 \ldots p'_i \ldots p'_n}
  }
}{AdvanceMany}
$$



\caption{Advancing processes}

% Evaluation: shaking allows proceses to take a step from one label to another as well as produce an output message. If the message is a push, the value is injected to all other processes in the network; otherwise it is an internal step.}
\label{fig:Process:Eval:Shake}
\end{figure}

