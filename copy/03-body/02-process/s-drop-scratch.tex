\begin{lstlisting}
map2par = $$(fuse $ do
  a <- source [||sourceOfFile "in.txt"||]
  b <- map    [|| f ||] a
  c <- map    [|| g ||] a
  sink b      [||sinkToFile   "b.txt"||]
  sink c      [||sinkToFile   "c.txt"||])
\end{lstlisting}

This function is represented by the following process network.

\begin{lstlisting}[linebackgroundcolor={
  \hilineFst{9}
  \hilineFst{10}
  \hilineFst{11}
  \hilineFst{12}
  \hilineFst{13}
  \hilineSnd{18}
  \hilineSnd{19}
  \hilineSnd{20}
  \hilineSnd{21}
  \hilineSnd{22}
  }]
Process network:
  Sources: a = sourceOfFile "in.txt"
  Sinks:   b = sinkToFile   "b.txt"
           c = sinkToFile   "c.txt"
  Processes:
    Process "map f a":
      Inputs:  a; Outputs: b; Initial: b0
      Instructions:
        b0   = Pull  a b1 b3
        b1 e = Push  b (f e) b2
        b2   = Drop  a b0
        b3   = Close b b4        
        b4   = Done

    Process "map g a":
      Inputs:  a; Outputs: c; Initial: c0
      Instructions:
        c0   = Pull  a c1 c3      
        c1 e = Push  c (g e) c2     
        c2   = Drop  a c0
        c3   = Close c c4         
        c4   = Done
\end{lstlisting}


\begin{figure}
\center
\begin{sequencediagram}
\newthreadGAP{a}{@a@}{0.2}
\newthreadGAP{mapf}{@map f@}{1.5}
\newthreadGAP{b}{@b@}{1.5}
\newthreadGAP{mapg}{@map g@}{1.5}
\newthreadGAP{c}{@c@}{1.5}

\parallelseq{
  \mess{a}{}{mapf}
}{
  \mess{a}{Push 1}{mapg}
}

\parallelseq{
  \mess{mapf}{Pull 1}{mapf}
  \mess{mapf}{Push (f 1)}{b}
  \mess{mapf}{Drop}{mapf}
}{
  \mess{mapg}{Pull 1}{mapg}
  \mess{mapg}{Push (g 1)}{c}
  \mess{mapg}{Drop}{mapg}
}

\parallelseq{
  \mess{a}{}{mapf}
}{
  \mess{a}{Push 2}{mapg}
}

\parallelseq{
  \mess{mapf}{Pull 2}{mapf}
  \mess{mapf}{Push (f 2)}{b}
  \mess{mapf}{Drop}{mapf}
}{
  \mess{mapg}{Pull 2}{mapg}
  \mess{mapg}{Push (g 2)}{c}
  \mess{mapg}{Drop}{mapg}
}
\end{sequencediagram}
\caption[Concurrent sequence diagram for parallel maps]{sequence diagram for concurrent execution of @map2par@. }
\label{figs/swim/map2par}
\end{figure}

\begin{figure}
\center
\begin{sequencediagram}
\newthreadGAP{a}{@a@}{0.2}
\newthreadGAP{mapf}{@map f@}{1.5}
\newthreadGAP{b}{@b@}{1.5}
\newthreadGAP{mapg}{@map g@}{1.5}
\newthreadGAP{c}{@c@}{1.5}

\parallelseq{
  \mess{a}{}{mapf}
}{
  \mess{a}{Push 1}{mapg}
}

\mess{mapf}{Pull 1}{mapf}

\mess{mapg}{Pull 1}{mapg}

\parallelseq{
  \mess{a}{}{mapf}
}{
  \mess{a}{Push 2}{mapg}
}

\mess{mapf}{Push (f 1)}{b}
\mess{mapf}{Pull 2}{mapf}
\mess{mapf}{Push (f 2)}{b}

\mess{mapg}{Push (g 1)}{c}
\mess{mapg}{Pull 2}{mapg}
\mess{mapg}{Push (g 2)}{c}
\end{sequencediagram}
\caption[Hypothetical sequence diagram for parallel maps without drop synchronisation]{sequence diagram for parallel execution of two map processes, using a hypothetical execution semantics without drop synchronisation.}
\label{figs/swim/map2par-no-sync}
\end{figure}

