% -----------------------------------------------------------------------------
\subsection{Fusibility}
\label{s:FusionOrder}
When we fuse a pair of processes, we commit to a particular interleaving of instructions from each process.
When we have at least three processes to fuse, the choice of which two to handle first can determine whether this fused result can then be fused with the third process.
Consider the following example, where @alt2@ pulls two elements from its first input stream, then two from its second, before pushing all four to its output.
\begin{haskell}
alternates :: S Nat -> S Nat -> S Nat -> S (Nat, Nat)
alternates sInA sInB sInC
 = let  s1   = alt2 sInA sInB
        s2   = alt2 sInB sInC
        sOut = zip s1 s2
   in   sOut
\end{haskell}
If we fuse the two @alt2@ processes together first, then try to fuse this result process with the downstream @zip@ process, the final fusion transform fails. This happens because the first fusion transform commits to a sequential instruction interleaving where two output values \emph{must} be pushed to stream @s1@ first, before pushing values to @s2@. On the other hand, @zip@ needs to pull a \emph{single} value from each of its inputs alternately.

Dynamically, if we were to execute the first fused result process, and the downstream @zip@ process concurrently, then the execution would deadlock. Statically, when we try to fuse the result process with the downstream @zip@ process the deadlock is discovered and fusion fails. Deadlock happens when neither process can advance to the next instruction, and in the fusion algorithm this manifests as the failure of the $tryStepPair$ function from \cref{fig:Fusion:Def:StepPair}. The $tryStepPair$ function determines which instruction from either process to execute next, and when execution is deadlocked there are none. Fusibility is an under-approximation for \emph{deadlock freedom} of the network.

% On the upside, fusion failure is easy to detect. It is also easy to provide a report to the client programmer that describes why two particular processes could not be fused.

% The report is phrased in terms of the process definitions visible to the client programmer, instead of partially fused intermediate code. The joint labels used in the fusion algorithm represent which states each of the original processes would be in during a concurrent execution, and we provide the corresponding instructions as well as the abstract states of all the input channels.

% This reporting ability is \emph{significantly better} than that of prior fusion systems such as Repa~\cite{lippmeier2012:guiding}, as well as the co-recursive stream fusion of \cite{coutts2007stream}, and many other systems based on general purpose program transformations. In such systems it is usually not clear whether the fusion transformation even succeeded, and debugging why it might not have succeeded involves spelunking\footnote{def. spelunking: Exploration of caves, especially as a hobby. Usually not a science.} through many pages (sometimes hundreds of pages) of compiler intermediate representations.

In practice, the likelihood of fusion succeeding depends on the particular dataflow network. For fusion of pipelines of standard combinators such as @map@, @fold@, @filter@, @scan@ and so on, fusion always succeeds. The process implementations of each of these combinators only pull one element at a time from their source streams, before pushing the result to the output stream, so there is no possibility of deadlock. Deadlock can only happen when multiple streams fan-in to a process with multiple inputs, such as with @merge@.

When the dataflow network has a single output stream then we use the method of starting from the process closest to the output stream, walking towards the input streams, and fusing in successive processes as they occur. This allows the interleaving of the intermediate fused process to be dominated by the consumers, rather than producers, as consumers are more likely to have multiple input channels which need to be synchronised. In the worst case the fall back approach is to try all possible orderings of processes to fuse.

% The main fusion algorithm works on pairs of processes.
% When there are more than two processes, there are multiple orders in which the pairs of processes can be fused.
% The order in which pairs of processes are fused does not affect the output values, but it does affect the access pattern: the order in which outputs are produced and inputs read.
% Importantly, the access pattern also affects whether fusion succeeds or fails to produce a process.
% In other words, while evaluating multiple processes is non-deterministic, the act of fusing two processes \emph{commits} to a particular deterministic interleaving of the two processes.
% 
% When we fuse a pair of processes we commit to a particular interleaving of instructions from each process.
% When we have at least three processes to fuse, the choice of which two to handle first can determine whether this fused result can then be fused with the third process.
% 
% Consider the process network @append2zip@ in \cref{l:bench:append2zip}.
% We have three input streams of integers: @as@, @bs@ and @cs@.
% We append @as@ and @cs@ together, as well as appending @bs@ and @cs@ together.
% Then we zip the two appended streams together.
% This will give us first elements from @as@ paired with those of @bs@, followed by elements from @cs@ paired with themselves.
% 
% \begin{lstlisting}[float,label=l:bench:append2zip,caption=append2zip; S stands for Stream]
% append2zip :: S Int -> S Int -> S Int -> Network m (S (Int, Int))
% append2zip as bs cs = do
%   ac   <- append as cs
%   bc   <- append bs cs
%   acbc <- zip    ac bc
%   return acbc
% \end{lstlisting}
% 
% In our current system, if we fuse the two @append@ processes together first, then try to fuse this result process with the downstream @zip@ process, then this final fusion transform will fail.
% This happens because the first fusion transform commits to a sequential instruction interleaving that first copies all of @as@ to @ac@, then copies all of @bs@ to @bc@, before copying @cs@ to both.
% The @zip@ requires values from @ac@ and @bc@ to be interleaved.
% But by committing to the interleaving that copies \emph{all} @ac@, we can no longer execute the @zip@ without introducing an unbounded buffer.
% 
% Dynamically, if we were to try to execute the result process and the downstream @zip@ process concurrently, then the execution would deadlock.
% Statically, when we try to fuse the result process with the downstream @zip@ process then the deadlock is discovered and fusion fails.
% Deadlock happens when neither process can advance to the next instruction, and in the fusion algorithm this will manifest as the failure of the $tryStepPair$ function from \cref{fig:Fusion:Def:StepPair}.
% The $tryStepPair$ function determines which of the instructions from either process can be executed next, and when execution is deadlocked there are none.
% 
% On the upside, fusion failure is easy to detect.
% It is also easy to provide a report to the client programmer that describes why two particular processes could not be fused.
% The report is phrased in terms of the process definitions visible to the client programmer, instead of partially fused intermediate code.
% The joint labels used in the fusion algorithm represent which states each of the original processes would be in during a concurrent execution, and we provide the corresponding instructions as well as the abstract states of all the input channels.
% This reporting ability is \emph{significantly better} than that of prior fusion systems such as Repa~\cite{lippmeier2012guiding}, as well as the co-recursive stream fusion of \cite{coutts2007stream}, and many other systems based on general purpose program transformations.
% In such systems it is usually not clear whether the fusion transformation even succeeded, and debugging why it might not have succeeded involves spelunking\footnote{def. spelunking: Exploration of caves, especially as a hobby. Usually not a science.} through many pages (sometimes hundreds of pages) of compiler intermediate representations.
% 
% In practice, the likelihood of fusion suceeding depends on the particular dataflow network being used, as well as the form of the processes in that network.
% For fusion of pipelines of standard combinators such as @map@, @fold@, @filter@, @scan@ and so on, fusion always succeeds.
% The process implementations of each of these combinators only pull one element at a time from their source streams, before pushing the result to the output stream, so there is no possiblity of deadlock.
% Deadlock can only happen when multiple streams fan-in to a process with multiple inputs, such as with @merge@.
% When the dataflow network has a single output stream then we use the method of starting from the process closest to the output stream, walking to towards the input streams, and fusing in successive processes as they occur.
% This allows the interleaving of the intermediate fused process to be dominated by the consumers, rather than producers, as consumers are more likely to have multiple input channels which need to be synchronized.
% In the worst case the fallback approach is to try all possible orderings of processes to fuse, assuming the client programmer is willing to wait for the search to complete. 


