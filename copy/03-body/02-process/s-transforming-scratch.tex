\subsection{Jump contraction}
\label{section:implementation:jump-contraction}
Between fusing each pair of processes in a process network, we want to simplify the fused process a bit.
The fusion algorithm introduces some extra jump instructions in the fused process, and we can often make the process smaller by removing the jump instructions.
Because the fusion algorithm is already hard to prove correct, we do not wish to make it more complex; instead we remove the jump instructions in a separate step after fusion.
It is easier to write two simple transforms than one complex transform.

We call this transform that removes jump instructions \emph{jump contraction}.
We will look at a fused process in order to understand jump contraction.

In the following example, @map2@, we take an input stream and transform it twice in a pipeline.

\begin{lstlisting}
map2 = $$(fuse $ do
  a <- source [||sourceOfFile "in.txt"||]
  b <- map    [|| f ||] a
  c <- map    [|| g ||] b
  sink c      [||sinkToFile  "out.txt"||])
\end{lstlisting}

This function is represented by the following process network, which has two processes; one for each @map@.

\begin{lstlisting}[linebackgroundcolor={
  \hilineFst{8}
  \hilineFst{9}
  \hilineFst{10}
  \hilineFst{11}
  \hilineFst{12}
  \hilineSnd{17}
  \hilineSnd{18}
  \hilineSnd{19}
  \hilineSnd{20}
  \hilineSnd{21}
  }]
Process network:
  Sources: a = sourceOfFile "in.txt"
  Sinks:   c = sinkToFile  "out.txt"
  Processes:
    Process "map f a":
      Inputs:  a; Outputs: b; Initial: b0
      Instructions:
        b0   = Pull  a b1 b3
        b1 e = Push  b (f e) b2
        b2   = Drop  a b0
        b3   = Close b b4        
        b4   = Done

    Process "map g b":
      Inputs:  b; Outputs: c; Initial: c0
      Instructions:
        c0   = Pull  b c1 c3      
        c1 e = Push  c (g e) c2     
        c2   = Drop  b c0
        c3   = Close c c4         
        c4   = Done
\end{lstlisting}

We fuse process (@map f a@) with process (@map f b@), to create a new process which pushes to outputs streams @b@ and @c@.
The instructions of each input process are highlighted differently; the same highlighting is used in the fused process to illustrate which input process each instruction comes from.
Each instruction in the fused process has a comment to the right describing the original process labels and the channel states.

\begin{lstlisting}[linebackgroundcolor={
  \hilineFst{8}
  \hilineCom{9}
  \hilineFst{10}
  \hilineSnd{11}
  \hilineSnd{12}
  \hilineFst{13}
  \hilineSnd{14}
  \hilineFst{16}
  \hilineSnd{17}
  \hilineSnd{18}
  \hilineCom{19}
}]
Process network:
  Sources: a = sourceOfFile "in.txt"
  Sinks:   c = sinkToFile  "out.txt"
  Processes:
    Process "map f a / map g b"
      Inputs:  a                  Outputs: b c                Initial: l0
      Instructions:
        l0   = Pull  a       l1 l7            -- b0, \{\},       c0, \{\}
        l1 w = Jump         (l2 (f w))        -- b1, \{\},       c0, \{\}
        l2 x = Push  b x    (l3 x)            -- b1, \{have b\}, c0, \{have b\}
        l3 y = Jump         (l4 y)            -- b2, \{\},       c0, \{have b\}
        l4 z = Push  c (g z) l5               -- b2, \{\},       c1, \{\}
        l5   = Drop  a       l6               -- b2, \{\},       c2, \{\}
        l6   = Jump          l0               -- b0, \{\},       c2, \{\}

        l7   = Close b       l8               -- b3, \{\},       c0, \{closed b\}
        l8   = Jump          l9               -- b4, \{\},       c3, \{closed b\}
        l9   = Close c       l10              -- b4, \{\},       c3, \{closed b\}
        l10  = Done                           -- b4, \{\},       c4, \{closed b\}
\end{lstlisting}

The fused process (@map f a / map g b@) produces to both streams @b@ and @c@.
The fused process keeps pushing to stream @b@, even though we have fused a producer with a consumer.
In this case there are no other consumers of stream @b@, but in general a stream may have multiple consumers.
When a stream has multiple consumers and we fuse the producer with one of the consumers, the fused process must keep producing for the sake of the other consumers.
In \autoref{section:implementation:cull-outputs} we shall see how to remove streams that have become redundant.

For the most part the fused process alternates between the two input processes, but there is some finer coordination between labels @l1@ to @l3@, where the (@map f a@) process is trying to push to the stream @b@ while the (@map g b@) process is trying to pull from the same.
When pushing, the (@map f a@) process applies the function @f@ to the current element and pushes this value to the stream.
The fused process also pushes the value to the stream, while keeping a copy to pass as a local variable to the other half of the fused process.
We wish to compute the application of @f@ only once.
The fused process has a jump instruction at label @l1@ which applies the function and passes it as an argument to the next instruction, where it can be shared between the two halves of the fused process.

The fused process has eleven instructions, while the inputs each had five.
The fused process has gotten bigger, and the four jump instructions comprise most of the growth.
If we can remove these jump instructions, the result will only be a little larger than each input process.

Looking at the jump instruction at label @l8@, we see that the close instruction at label @l7@ continues to @l8@, then follows the jump instruction to @l9@.
We modify @l7@ to continue straight to @l9@ without the intermediate step, and remove the instruction at label @l8@ entirely.
We can perform the same simplification at the jump instruction for label @l6@ by finding the occurence in the instruction for label @l5@ and replacing it with the jump's destination, @l0@.

The jump instruction at label @l3@ is parameterised and is referred to in the instruction for label @l2@.
For parameterised jumps we must substitute the parameter names for their arguments, and the call (@l3 x@) becomes (@l4 x@).

Essentially, we are inlining each jump instruction into its use sites.
This transform is more restricted than general inlining because the process language the destination continuations only allow applications of named labels.

The jump instruction at label @l1@ is referred to in the pull instruction at label @l0@.
The call to @l1@ is not applied to all its arguments, because the pull instruction fills in the last argument with the pulled value.
We cannot inline an under-applied continuation into the call-site at label @l0@, because the syntactic form for continuations has no way to introduce local functions.
We could instead look at the jump instruction and inline the definition of the push instruction at label @l2@, but this exposes two complications.

First, the push instruction refers to its parameter, @x@, twice.
If we inlined the push instruction, we would have to substitute both occurrences of parameter @x@ with the value (@f w@), which would duplicate work.
To avoid duplicating work, we can only inline if each parameter is mentioned at most once or is substituted by a variable or value which can be duplicated.
We cannot remove the jump instruction at label @l1@.

The second complication is that by inlining an instruction into the jump instruction, we might duplicate the instruction.
If there were two jump instructions which continued to the same push instruction and we inline the push instruction into both jumps, we have removed two simple instructions at the expense of duplicating the push instruction.
To avoid duplicating instructions, we can only inline into the body of a jump if the next instruction's label is only mentioned once.
The process representation our implementation uses means that counting mentions for a label is linear in the number of labels.
There are surely better representations, but these would require extra bookkeeping when modifying the instruction during inlining.
For now, our implementation only inlines references to jumps, and does not inline other instructions into jumps.

After removing the jumps, we get the following process with eight states.

\begin{lstlisting}[linebackgroundcolor={
  \hilineFst{8}
  \hilineCom{9}
  \hilineFst{10}
  \hilineSnd{11}
  \hilineFst{12}
  \hilineFst{14}
  \hilineSnd{15}
  \hilineCom{16}
}]
Process network:
  Sources: a = sourceOfFile "in.txt"
  Sinks:   c = sinkToFile  "out.txt"
  Processes:
    Process "map f a / map g b"
      Inputs:  a                  Outputs: b c                Initial: l0
      Instructions:
        l0   = Pull  a       l1 l7
        l1 w = Jump         (l2 (f w))
        l2 x = Push  b x    (l4 x)
        l4 z = Push  c (g z) l5
        l5   = Drop  a       l0

        l7   = Close b       l9
        l9   = Close c       l10
        l10  = Done
\end{lstlisting}

We must be careful to ensure the transform terminates.
If we have a jump instruction which jumps to itself, we cannot inline it recursively, because we would end up in an infinite loop.
The implementation must keep track of which labels have been inlined, and avoid doing the same thing multiple times.
% we construct a new process by starting from the initial label and exploring outwards.
% At each step 
% when inlining an instruction we keep a set of labels that we have inlined.
% If we encounter the same label twice, it is time to stop unfolding the definition.


\subsection{Cull outputs}
\label{section:implementation:cull-outputs}

When one process in the network pushes to an output stream but no other processes use the stream as an input, we can simplify the process network by removing the stream and replacing the instructions that mention the stream with jumps.
This situation is unlikely to occur in the original process network because a programmer would not introduce a channel without using it, but it does happen often in the intermediate process network after some processes have been fused.

Continuing with the same @map2@ example, we can remove the stream @b@ after the two processes have been fused together.
The fused process has two instructions that refer to the stream @b@: the push instruction at label @l2@, and the close instruction at label @l7@.
We replace each instruction with a jump to the original instruction's destination, as follows.

\begin{lstlisting}[linebackgroundcolor={
  \hilineFst{8}
  \hilineCom{9}
  \hilineFst{10}
  \hilineSnd{11}
  \hilineFst{12}
  \hilineFst{14}
  \hilineSnd{15}
  \hilineCom{16}
}]
Process network:
  Sources: a = sourceOfFile "in.txt"
  Sinks:   c = sinkToFile  "out.txt"
  Processes:
    Process "map f a / map g b"
      Inputs:  a                  Outputs: b c                Initial: l0
      Instructions:
        l0   = Pull  a       l1 l7
        l1 w = Jump         (l2 (f w))
        l2 x = Jump         (l4 x)
        l4 z = Push  c (g z) l5
        l5   = Drop  a       l0

        l7   = Jump          l9
        l9   = Close c       l10
        l10  = Done
\end{lstlisting}

By replacing each instruction with jumps, we have exposed more opportunities for jump contraction to remove states.
For exposition we performed jump contraction before culling outputs, but in our implementation we perform cull outputs first.
If we perform jump contraction again, we have the following result with six states.
There is only one superfluous instruction: the jump instruction at label @l1@.
We could remove this jump if we inlined the push instruction at label @l4@ into the jump, but this is not implemented yet.
This is still a significant improvement over the original fused process, which had eleven states.

\begin{lstlisting}[linebackgroundcolor={
  \hilineFst{8}
  \hilineCom{9}
  \hilineFst{10}
  \hilineFst{11}
  \hilineSnd{13}
  \hilineCom{14}
}]
Process network:
  Sources: a = sourceOfFile "in.txt"
  Sinks:   c = sinkToFile  "out.txt"
  Processes:
    Process "map f a / map g b"
      Inputs:  a                  Outputs: b c                Initial: l0
      Instructions:
        l0   = Pull  a       l1 l9
        l1 w = Jump         (l4 (f w))
        l4 z = Push  c (g z) l5
        l5   = Drop  a       l0

        l9   = Close c       l10
        l10  = Done
\end{lstlisting}


