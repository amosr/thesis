\section{Transforming process networks}
\label{s:Optimisation}

The fusion algorithm described in \cref{s:Fusion} operates on a pair of input processes.
For process networks which contain more than two processes, we repeatedly fuse pairs of processes in the network together until only one process remains.

When fusing a pair of processes, the fused process tends to have more states than each input process individually, because the fused process has to do the work of both input processes.
In general, the larger the input processes, the larger the fused process will be, and when we have many processes to fuse, the result will get progressively larger as we fuse more processes in.
If the fused process becomes too large such that the process does not fit in memory, then fusing in the next process will take longer, and code generation will take longer.
When repeatedly fusing the pairs of processes in a network, we perform some simplifications between each fusion step, to remove unnecessary states and simplify the input for the next fusion step.

% The fusion algorithm also introduces some spurious states.
% It is designed for simplicity of the fusion algorithm, at the expense of simplicity of the output.
% If the input process has a couple more states than necessary, this can turn into several unnecessary states in the fused process.
% When this fused process is used as the input to another fusion step, the unnecessary states compound.
% What started as a couple can become dozens, then hundreds, then thousands.
% As with compound interest on a loan, it is best to pay back early and often.


\subsection{Fusing a network}
\label{ss:Fusing:a:network}

\begin{figure}
\center
\begin{dot2tex}[dot]
digraph G {
  node [shape="none"];
  stock; index;

  stock -> pom_join;
  index -> pom_join;
  stock -> pot_tps;

  graph [style="rounded corners,filled"];

  subgraph cluster_z {
    graph [bgcolor="0.0 0.0 0.9"];
    subgraph cluster_priceAgainstMarket {
      lblstyle="right";
      label="priceOverMarket";
      graph [bgcolor="0.0 0.0 0.8"];
      pom_join [label="join"];
      subgraph cluster_x {
        label="";
        graph [bgcolor="0.0 0.0 0.7"];
        subgraph cluster_x {
          graph [bgcolor="0.0 0.0 0.6"];
          pom_price [label="map"];
          pom_cor [label="correlation"];
        };
      pom_reg [label="regression"];
      }
      pom_join -> pom_price;
      pom_price -> pom_cor;
      pom_price -> pom_reg;
    };

    subgraph cluster_priceOverTime  {
      lblstyle="left";
      label="priceOverTime";
      graph [bgcolor="0.0 0.0 0.8"];
      subgraph cluster_x1 {
        label="";
        graph [bgcolor="0.0 0.0 0.7"];
        pot_tps [label="map"];
        pot_cor [label="correlation"];
      }
      pot_reg [label="regression"];
      pot_tps -> pot_cor;
      pot_tps -> pot_reg;
    };

  }
}
\end{dot2tex}
\caption{Pairwise fusion ordering of the priceAnalyses network.}
\label{figs/procs/priceAnalyses-fusing-whole}
\end{figure}

As we shall see, when we fuse pairs of processes in a network, the order in which we fuse pairs can determine whether fusion succeeds.
Rather than trying all possible orders, of which there are many, we use a bottom-up heuristic to choose a fusion order.
% This heuristic is not guaranteed to choose the, but it .
\Cref{figs/procs/priceAnalyses-fusing-whole} shows the heuristically chosen fusion order for the @priceAnalyses@ example.
The processes are nested inside boxes; each box denotes the result of fusing a pair of processes, and inner-most boxes are to be fused first.
Each box is shaded to denote its nesting, and the more deeply nested a box is, the darker its shade.
In @priceOverTime@, we start by fusing the @correlation@ process with its producer, @map@; we then fuse the resulting process with the @regression@ process.
In @priceOverMarket@, we also start by fusing the @correlation@ process with @map@, then adding @regression@, and fusing in the @join@ process.
Finally, we fuse the result process for @priceOverTime@ with the result process for @priceOverMarket@.

% There are many orders in which we could fuse the pairs.
% For $n$ processes, there are $n!$ different permutations and $(n-1)!$ different ways to nest the parentheses.
% For $3$ processes there are $12$ orders; for $4$ processes there are $144$; for $5$ processes there are $2,880$.
% It does not take many processes for there to become too many orders to try.
% For $10$ processes, there are more than a trillion possibilities.
% Even if we could fuse one process in a single instruction, at 3GHz it would take forty minutes to try all orders.
% If the process network is fundamentally unfusable, it is unacceptable to force the user to wait forty minutes before telling them we cannot fuse it.

% We cannot try all the orders; we need some way to choose the order.
% As we shall see in \cref{s:extraction:future}, it is impossible to choose the right order by looking at the dependency graph alone; the correct order depends upon the implementation of each process.
% As future work we shall propose a fusion algorithm which is commutative and associative, which means we can fuse processes in any order.
% We shall start by explaining a heuristic which does not always choose the right order, but works for the benchmarks in \REFTODO{benchmarks}.
% \TODO{Maybe the explanation of why it has to be a heuristic should come first, but it makes it unappealing to start with an example where the heuristic doesn't work. If I explain the future work, it seems like there's no reason to explain the heuristic. Reorganise later.}

To demonstrate how fusion order can affect whether fusion succeeds, consider the following list program, which takes three input lists, appends them, and zips the appended lists together.

\begin{lstlisting}
append2zip :: [a] -> [a] -> [a] -> [(a,a)]
append2zip a b c =
  let ba = b ++ a
      bc = b ++ c
      z  = zip ba bc
  in  z
\end{lstlisting}

We use the more convenient syntax for list programs rather than the process network syntax introduced earlier, but in the discussion we interpret this program as a process network.
In the process network interpretation, each list combinator corresponds to a process, and each list corresponds to a stream.
The dependency graph for the corresponding process network is shown in \cref{figs/specconstr/append2zip}.


\FigurePdf{figs/specconstr/append2zip}{Dependency graph for append2zip}{Dependency graph for append2zip}

This example appends the input streams, then pairs together the elements in both appended streams.
The result of the two append processes, @ba@ and @bc@, both contain the elements from @b@ stream, followed by the elements of the second append argument; stream @a@ or stream @c@ respectively.
These two streams, @ba@ and @bc@, when paired together, will result in each element of the @b@ stream paired with itself, followed by elements of the two other streams paired together.


\Cref{figs/swim/append2zip} shows an example execution of @append2zip@, displayed as a sequence diagram.
In this diagram, we omit the @drop@ and @pull@ internal messages for all processes, and focus instead on the communication between processes.
Each input stream pushes its elements to its consumers; input stream @a@ has elements $[1, 2]$, input stream @b@ has elements $[3, 4]$, and input stream @c@ has elements $[5, 6]$.
Input stream @b@ has multiple consumers, so when it pushes elements, it pushes to both its consumers at the same time.


\begin{figure}
\center
\begin{sequencediagram}
\newthreadGAP{a}{@a@}{0.0}
\newthreadGAP{b}{@b@}{0.7}
\newthreadGAP{c}{@c@}{0.7}
\newthreadGAP{appba}{@b++a@}{0.7}
\newthreadGAP{appbc}{@b++c@}{0.7}
\newthreadGAP{zip}{@zip@}{0.7}
\newthreadGAP{z}{@z@}{0.7}

\messmessx{b}{push 3}{appba}{appbc}

\mess{appba}{push 3}{zip}
\mess{appbc}{push 3}{zip}
\mess{zip}{push (3,3)}{z}

\messmessx{b}{push 4}{appba}{appbc}

\mess{appba}{push 4}{zip}
\mess{appbc}{push 4}{zip}
\mess{zip}{push (4,4)}{z}

\messmessx{b}{close}{appba}{appbc}

\addtocounter{seqlevel}{3}

\mess{a}{push 1}{appba}
\mess{appba}{push 1}{zip}

\mess{c}{push 5}{appbc}
\mess{appbc}{push 5}{zip}

\mess{zip}{push (1,5)}{z}

\mess{a}{push 2}{appba}
\mess{appba}{push 2}{zip}

\mess{c}{push 6}{appbc}
\mess{appbc}{push 6}{zip}

\mess{zip}{push (2,6)}{z}

\addtocounter{seqlevel}{3}

\mess{a}{close}{appba}
\mess{appba}{close}{zip}

\mess{c}{close}{appbc}
\mess{appbc}{close}{zip}

\mess{zip}{close}{z}

\end{sequencediagram}
\caption[Concurrent sequence diagram for two `append2zip']{sequence diagram for execution of @append2zip@. }
\label{figs/swim/append2zip}
\end{figure}


The execution has three sections.
In the first section, all the values from the @b@ stream are pushed to both append processes, then paired together.
In the second section, execution alternates between the other streams, @a@ and @c@, with one value from each.
In the third section, the streams are closed, which propagates down to the consumers.
The bottom-most consumer process, @zip@, executes by alternately pulling from each of the append processes.
The order in which a process pulls from its inputs is called its \emph{access pattern}.
Each append process can only push when the @zip@ process' buffer for that channel is empty: append must wait for @zip@ to read the most recent element before pushing a new element.
When each append process is waiting, its producer---the input stream---must also wait before pushing the next element.
This waiting propagates the @zip@ process' access pattern upwards through the append processes and to the input streams.

This example contains three processes.
We could perform fusion in twelve different orders.
Of these twelve orders, there are two main categories, distinguished by whether we start by fusing the append processes with each other, or start by fusing the @zip@ process with one of the append processes.
% The first category, we fuse the two append processes together, then fuse with the @zip@ process.
% In the second category, we fuse the @zip@ process with one of the append processes, then fuse with the other append process.

If we fuse the two append processes together first, we interleave their instructions without considering the access pattern of the @zip@ process.
There are many ways to interleave the two processes; one possibility is that the fused process reads all of the shared prefix from stream @b@, then all of stream @a@, then all of stream @c@.
For the shared prefix, this interleaving alternates between pushing to streams @ba@ and @bc@.
After the shared prefix, this interleaving pushes the rest of the stream @ba@, then pushes the rest of the stream @bc@.
When we try to fuse the @zip@ process with the fused append processes with this interleaving, we get stuck.
The @zip@ process needs to alternate between its inputs, which works for the shared prefix, but not for the remainder.
By fusing the two append processes together first, we risk choosing an interleaving that works for the two append processes on their own, but does not take into account the access pattern of the @zip@ process.

Fusion does succeed if we fuse the @zip@ process with one of the append processes first, then fuse with the other append process.
The consumer, @zip@, must dictate the order in which the append processes push; fusing the @zip@ process first gives it this control.
We start from the consumer and fuse them upwards with their producers, because this allows the consumer to impose its access pattern on the producers.

To fuse an arbitrary process network, we consider a restricted view of the dependency graph, ignoring the overall inputs and outputs of the network.
We start at the bottom of the dependency graph, finding the \emph{sink} processes, or those with no output edges.
These sink processes are the bottom-most consumers which, like @zip@ in our @append2zip@ example, dictate the access pattern on their inputs.
For each sink process, we find its parents and fuse the sink process with its parents.
When the sink process has multiple parents, we need to choose which parent to fuse with first.
In the @append2zip@ example, we can fuse the @zip@ process with its append parents in any order.
In general, one parent may consume the other parent's output; in which case we first fuse with the consumer parent.
This order allows the consuming parent to impose its access pattern upon the producing parent.
We repeatedly fuse each sink process with its closest parent until there are no more parents.

After fusing each sink process with all its ancestors, there may remain multiple processes.
This only occurs if the remaining processes do not share ancestors.
The remaining processes also cannot share descendents, since if they had descendents they would not be sink.
This means the processes are completely separate and could be executed separately, in any order or even in parallel.
Having unconnected processes in the same process network is a degenerate case, as it could be represented as multiple process networks.
We err on the side of caution, telling the programmer about anything even slightly unexpected.
Rather than making the decision of which order to execute them in, we display a compile-time error and make the programmer separate the network.

The fusion algorithm for pairs of processes fails and does not produce a result process when two processes have conflicting access patterns on their shared inputs.
As the access patterns are determined statically, apparent conflicts may never occur at runtime; we instead make a pessimistic approximation.
In the implementation, if at any point we encounter a pair of processes which we cannot fuse together, we display a compile-time error telling the programmer that the network cannot be fused.

Unfortunately, this heuristic cannot always choose the correct fusion ordering.
It is not possible, in general, to choose the correct ordering based on the dependency graph alone.
Consider the following list program, @append3@, which appends three input lists in various orders, producing three output lists.
As with the @append2zip@ example, we present the example as a list program for syntactic convenience, while we interpret it as a process network.

\begin{haskell}
append3 :: [a] -> [a] -> [a] -> ([a],[a],[a])
append3 a b c =
  let ab = a ++ b
      ac = a ++ c
      bc = b ++ c
  in  (ab, ac, bc)
\end{haskell}

\begin{figure}
\begin{minipage}[t]{0.5\textwidth}
\center
\begin{dot2tex}[dot]
digraph G {
  node [shape="none"];
  b; a; c;
  app1 [label="append"];
  app2 [label="append"];
  app3 [label="append"];
  a -> app1; a -> app2;
  b -> app1; b -> app3;
  c -> app2; c -> app3;
  app1 -> ab;
  app2 -> ac;
  app3 -> bc;
}
\end{dot2tex}
\end{minipage}
\begin{minipage}[t]{0.5\textwidth}
\center
\begin{dot2tex}[dot]
digraph G {
  node [shape="none"];
  b; a; c;
  app1 [label="zip"];
  app2 [label="zip"];
  app3 [label="zip"];
  a -> app1; a -> app2;
  b -> app1; b -> app3;
  c -> app2; c -> app3;
  app1 -> ab;
  app2 -> ac;
  app3 -> bc;
}
\end{dot2tex}
\end{minipage}
\caption[Process network for `append3']{process networks for @append3@ and @zip3@.}
\label{figs/procs/append3-zip3}
\end{figure}




This process network can be executed with no buffering.
First, read all of the @a@ input stream, then read the @b@ stream, then read the @c@ stream.
There is no single consumer in this example which imposes its access pattern on its producers, so our heuristic fails.
This process network can only be fused if the process that produces @ab@ and the process that produces @bc@ are first fused together.
If we fused @ab@ and @ac@ together first, the fusion algorithm would make an arbitrary decision of whether to read the @b@ stream before, after, or interleaved with the @c@ stream.
The heuristic described will not necessarily choose the right order.

Looking at the dependency graph alone, it is impossible to tell which is the right order for fusion.
If we take the @append3@ example and replace the append processes with @zip@ processes, the dependency graph remains the same, but either fusion order would work.
\Cref{figs/procs/append3-zip3} shows the process networks of @append3@ and @append3@ replaced with @zip@ processes.



We propose to solve this in future work by modifying the fusion algorithm to be commutative and associative.
These properties would allow us to apply fusion in any order, knowing that all orders produce the same result.

The fusion algorithm is not commutative because when two processes are trying to execute instructions which could occur in either order, the algorithm must choose only one instruction.
Fusion commits too early to a particular interleaving, when there are multiple interleavings that would work.
By explicitly introducing non-determinism in the fused process, we can represent all possible interleavings, and do not have to commit to one too early.
We are moving the non-determinism from the order in which fusion occurs, and reifying it in the process itself.

Reifying the non-determinism in the processes will mean that all fusion orders produce the same process at the end.
Different orders will not affect the result, or whether things fuse.
Different orders do affect the size of the intermediate process, before all processes are fused together.
Fusing two unrelated processes which read from different streams introduces a lot of non-determinism: at each step of the fused process, either of the original processes can take a step.
The two processes do not constrain each other and the result process will have a lot of states.
Fusing related processes, for example a producer and a consumer, introduce less non-determinism because there are points when only one of the processes can run.
When the consumer is waiting for a value, only the producer can run.
Generally, fusing related processes will produce a smaller process than fusing unrelated processes.
The size of the overall result for the entire network is not any different, but the intermediate process will be smaller.
Larger intermediate programs generally take longer to compile, so some heuristic order which fuses related processes is likely to be useful, even if the order does not affect the result.


% \subsection{Fusing a pair of processes}
% Fusing a pair of processes is slightly different because variables are passed as function arguments, instead of a global heap.
% It also needs to take into account variable bindings per label, because of the locally-bound variables.
% For a pair of labels, the variables is the union of variables in the original processes, as well as any \emph{new} channel buffers which need to be bound.
% These are the ones which have an input state of \emph{have}.


