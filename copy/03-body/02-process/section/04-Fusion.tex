%!TEX root = ../Main.tex

% -----------------------------------------------------------------------------
\chapter{Fusion}
\label{s:Fusion}

The core fusion algorithm constructs a static execution schedule for a single pair of processes. To fuse a whole process group we fuse successive pairs of processes until only one remains. 

Fig.~\ref{fig:Fusion:Types} defines some auxiliary grammar used during fusion. We extend the exiting $Label$ grammar with an additional alternative, $Label_S \times Label_S$ which we use for the labels in a fused result process. Each $Label_S$ consists of a $Label$ from one of the  source processes, paired with a map from $Channel$ to the statically known part of that channel's current $\InputState$. The definition of $Label$ is now recursive. While fusing a whole process group, as we fuse pairs of individual processes the labels in the result collect more and more information. Each label of the final, completely fused process encodes the joint state that all the original source processes would be in at that point.

% These new labels $Label_S$ consist of a pair of source labels, as well as the static part of the $\InputState$ of each input channel. 

% If the static $\InputState_S$ is $@pending@_S$, there is a value waiting to be pulled, do not know the actual value. 

We also extend the existing $Variable$ grammar with a (@chan@ c) form which represents the stream buffer variable associated with channel @c@. We only need one stream buffer variable for each channel, and naming them like this saves us from inventing fresh variable names in the definition of the fusion rules (we used a fresh name @b1@ back in \S\ref{s:Fusion:FusingPulls} to avoid introducing a new mechanism at that point in the discussion) 

Still in Fig.~\ref{fig:Fusion:Types}, $\ChanType_2$ classifies how channels are used, and possibly shared, between two processes. Type @in2@ indicates that the two processes @pull@ from the same channel, and these actions must then be coordinated. Type @in1@ indicates that only a single process pulls from the channel. Type @in1out1@ indicates that one process pushes to the channel and the other pulls from the same. Type @out1@ indicates that the channel is pushed to by a single process. Each output channel is uniquely owned and cannot be pushed to by more than one process.

%!TEX root = ../Main.tex

\begin{figure}

\begin{tabbing}
@MMMMMMMMMMMM@   \TABDEF \kill

$\Label$        \> ::=  \> ~~~ $\ldots ~|~\LabelF~\times~\LabelF ~|~ \ldots$ \\
$\LabelF$      \> =    \> ~~~ $\Label~\times~\MapType{\Chan}{\InputStateF}$  \\
$\InputStateF$ \> ::=  \> ~~~ $@none@_F ~|~ @pending@_F ~|~ @have@_F ~|~ @closed@_F$    \\
$\Var$          \> ::=  \> ~~~ $\ldots ~|~@chan@~\Chan ~|~ \ldots$ \\
\\

$\ChanType_2$   \> ::=  \> ~~~ $@in2@~|~@in1@~|~@in1out1@~|~@out1@$
\end{tabbing}

\caption{Fusion type definitions.}
% The labels for a fused program consists of both of the original program labels, as well as the statically known part of the input state for each channel. The channels of both processes are classified into inputs and outputs, this describes what coordination is required between the two.
\label{fig:Fusion:Types}
\end{figure}



%!TEX root = ../Main.tex

\begin{figure}

\begin{tabbing}
\ti{fusePair}~@M M@   \TABDEF \kill

\ti{fusePair} \> $~:$ \> $\Proc \to \Proc \to  \Maybe~\Proc$ \\
$\ti{fusePair}~p~q$ \> $=$ \\
@    process@ \\
@        ins: @ $\sgl{c~|~c=t \in \cs,~t \in \sgl{@in1@,@in2@}} $ \\
@       outs: @ $\sgl{c~|~c=t \in \cs,~t \in \sgl{@in1out1@,@out1@}} $ \\
@       heap: @ $p[@heap@]~\cup~q[@heap@]$ \\
@      label: @ $l_0$ \\
@     instrs: @ $\ti{go}~\sgl{}~l_0$ \\
@ where@ \\
MM\=MM\=~=~\=\kill
 \> \cs \> $=$ \> $\ti{channels}~p~q$ \\
 \> $l_0$   \> $=$ \> $
      \big( 
      (p[@label@],~\sgl{c=none_S~|~c~\in~p[@ins@]}),~
      (q[@label@],~\sgl{c=none_S~|~c~\in~q[@ins@]})
      \big)$ \\
 \\
 \> $\ti{go}~\ti{bs}~(l_p,l_q)$ \\
 \> \> $~|$ \> $(l_p,l_q)~\in~\ti{bs}$ \\
 \> \> $=$  \> $\ti{bs}$ \\
 \> \> $~|$ \>
        $b~\in~\ti{tryStepPair}~\cs~l_p~p[@instrs@][l_p]~l_q~q[@instrs@][l_q]$ \\ 
 \> \> $=$ \> $\ti{fold}~\ti{go}~(\ti{bs}~\cup~\sgl{(l_p,l_q)=b})~(\ti{outlabels}~b)$
\end{tabbing}

\caption{Fusion of pairs of processes}

% Two processes are fused together by starting at the initial label for each process and computing the instruction based on one of the original process' instructions at that label. Instructions are added recursively until all reachable instructions are included.

\label{fig:Fusion:Def:Top}
\end{figure}






\smallskip
% -------------------------------------
Fig.~\ref{fig:Fusion:Def:Top} defines function \ti{fusePair} that fuses a pair of processes together, constructing a result process that does the job of both. We start with a joint label $l_0$ formed from the initial labels of the two source processes. We then use \ti{tryStepPair} to statically choose which of the two processes to advance, and hence which instruction to execute next. The possible destination labels of that instruction (computed with $outlabels$ from Fig.~\ref{fig:Fusion:Utils}) define new joint labels and reachable states. As we discover reachable states we add them to a map $bs$ of joint label to corresponding the instruction, and repeat the process to a fixpoint where no more states can be discovered.

%!TEX root = ../Main.tex

% Settle on a syntax for this later.
% Might be too many nested pairs.
\newcommand\nextStep[5]{\big((#1,~#2),~(#3,~#4),~#5 \big)}


% I tried this colour in a colour blindness simulator and it seems to be OK.
% Should still be readable when converted to grayscale.
\definecolor{notec}{HTML}{C03020}
\newcommand\note[1]{\textcolor{notec}{(#1)}}

\begin{figure}
\begin{tabbing}
M \= M \= M \= M \kill
$\ti{tryStepPair} ~:~ \ChanTypeMap \to \Label_1 \to \Instr \to \Label_1 \to \Instr \to \Maybe~\Instr$ \\
$\ti{tryStepPair} ~\cs~l_p~i_p~l_q~i_q$ \\

\> \note{PreferJump1} \\
\> $~|~i_p'~\in~\ti{tryStep}~\cs~l_p~i_p~l_q ~\wedge~@jump@~(l,u)~\in~i_p'$ 
   ~~~~~ $\to~i_p'$ \\
\> \note{PreferJump2} \\
\> $~|~i_q'~\in~\ti{tryStep}~\cs~l_q~i_q~l_p ~\wedge~@jump@~(l,u)~\in~i_q'$
   ~~~~~ $\to~\ti{swaplabels}~i_q'$ 
\\[0.5em]

\> \note{DeferPull1} \\
\> $~|~i_p'~\in~\ti{tryStep}~\cs~l_p~i_p~l_q ~\wedge~ i_q'~\in~\ti{tryStep}~\cs~l_q~i_q~l_p$ 
   ~~ $\wedge~@pull@~c~x~(l,u)~\not\in~i_p'$ 
   ~~~~~ $\to~i_p'$ \\
\> \note{DeferPull2} \\
\> $~|~i_p'~\in~\ti{tryStep}~\cs~l_p~i_p~l_q ~\wedge~i_q'~\in~\ti{tryStep}~\cs~l_q~i_q~l_p$
   ~~ $\wedge~@pull@~c~x~(l,u)~\not\in~i_q'$ 
   ~~~~~ $\to~\ti{swaplabels}~i_q'$ 
\\[0.5em]

\> \note{Run1} \\
\> $~|~i_p'~\in~\ti{tryStep}~\cs~l_p~i_p~l_q$ ~~~~~ $\to~i_p'$ \\
\> \note{Run2} \\
\> $~|~i_q'~\in~\ti{tryStep}~\cs~l_q~i_q~l_p$ ~~~~~ $\to~\ti{swaplabels}~i_q'$
\end{tabbing}
\caption{Fusion step coordination for a pair of processes.}
% Statically compute the instruction to perform at a particular fused label. Try to execute either process, preferring jumps and other instructions over pulling, as pulling can block while other instructions may perform ``useful work'' without blocking. If neither machine can execute, fusion fails.
\label{fig:Fusion:Def:StepPair}
\end{figure}



% -------------------------------------
Fig.~\ref{fig:Fusion:Def:StepPair} defines function \ti{tryStepPair} which decides which of two processes to advance.

Clauses (PreferJump1) and (PreferJump2) are heuristics which prioritize processes that can perform a @jump@. This heuristic helps to collect @jump@ instructions together in the result process, which are then easier for post-fusion optimization (\S\ref{s:Optimisation}) to handle. As clause (PreferJump2) calls $\ti{tryStep}$ with the instructions in swapped order, if \ti{tryStep} succeeds we use $\ti{swaplabels}$ (from Fig.~\ref{fig:Fusion:Utils}) to re-swap the labels in the result.

Clauses (DeferPull1) and (DeferPull2) are similar heuristics, which defer @pull@ instructions if possible. We do this because @pull@ instructions may block, while other instructions are more likely to produce immediate results.

Clauses (Run1) and (Run2) apply when one process can advance using some other instruction. We try the first process first, and if that can advance then so be it. This priority means that fusion is left-biased, preferring advancement of the left process over the second.

%!TEX root = ../Main.tex

\begin{figure*}

\begin{tabbing}
$\ti{tryStep} ~:~ \ChanTypeMap \to \LabelF \to \Instr \to \LabelF \to \Maybe~\Instr$ \\
$\ti{tryStep} ~\cs~(l_p,s_p)~i_p~(l_q,s_q)~=~@match@~i_p~@with@$ 
\end{tabbing}

\vspace*{-\baselineskip}
\vspace*{1ex}

\begin{tabular}{lr}
  $@jump@~(l',u')$  & \note{LocalJump} \\
  $\quad\to~\Just (@jump@~
      \nextStep
        {l'}{s_p}
        {l_q}{s_q}
        {u'})
      $ 
      \\[1ex]
  $@case@~e~(l'_t,u'_t)~(l'_f,u'_f)$ & \note{LocalCase} \\
$\quad\to~\Just (@case@~e~
      \nextStep
        {l'_t}{s_p}
        {l_q}{s_q}
        {u'_t}
      ~
      \nextStep
        {l'_f}{s_p}
        {l_q}{s_q}
        {u'_f})
      $ 
\\[1ex]

$@push@~c~e~(l',u')$ \\
  $\quad~|~\cs[c]=@out1@$ & \note{LocalPush} \\
$\quad\to~\Just (@push@~c~e~
      \nextStep
        {l'}
          {s_p}
        {l_q}
          {s_q}
        {u'})
      $ 
\\
  $\quad~|~\cs[c]=@in1out1@ ~\wedge~ s_q[c]=@none@_F$ & \note{SharedPush} \\
$\quad\to~\Just (@push@~c~e~
      \nextStep
        {l'}
          {s_p}
        {l_q}
          {\HeapUpdateOne{c}{@pending@_F}{s_q}}
        {\HeapUpdateOne{@chan@~c}{e}{u'}})
      $
\\[1ex]

$@drop@~c~(l',u')$ \\
  $\quad~|~\cs[c]=@in1@$ & \note{LocalDrop} \\
  $\quad\to~\Just (@drop@~c~
      \nextStep
        {l'}
          {s_p}
        {l_q}
          {s_q}
        {u'})
      $
      \\

  $\quad~|~\cs[c]=@in1out1@$ & \note{ConnectedDrop} \\
$\quad\to~\Just (@jump@~
      \nextStep
        {l'}
          {\HeapUpdateOne{c}{@none@_F}{s_p}}
        {l_q}
          {s_q}
        {u'})
      $
      \\

$\quad~|~\cs[c]=@in2@ ~\wedge~ (s_q[c]=@have@_F \vee s_q[c]=@pending@_F)$  & \note{SharedDropOne} \\
$\quad\to~\Just (@jump@~
      \nextStep
        {l'}
          {\HeapUpdateOne{c}{@none@_F}{s_p}}
        {l_q}
          {s_q}
        {u'})
      $
      \\


$\quad~|~\cs[c]=@in2@ ~\wedge~ s_q[c]=@none@_F$ & \note{SharedDropBoth} \\
$\quad\to~\Just (@drop@~c~
      \nextStep
        {l'}
          {\HeapUpdateOne{c}{@none@_F}{s_p}}
        {l_q}
          {s_q}
        {u'})
      $
\\[1ex]

$@pull@~c~x~(l'_o,u'_o)~(l'_c,u'_c)$ \\

$\quad~|~\cs[c]=@in1@$ & \note{LocalPull} \\
$\quad\to~\Just (@pull@~c~x~
      \nextStep
        {l'_o}{s_p}
        {l_q}{s_q}
        {u'_o}
      ~
      \nextStep
        {l'_c}{s_p}
        {l_q}{s_q}
        {u'_c})
    $ 
\\[1ex]

$\quad~|~(\cs[c]=@in2@ \vee \cs[c]=@in1out1@) ~\wedge~ s_p[c]=@pending@_F$ & \note{SharedPull} \\
$\quad\to~\Just (@jump@~
      \nextStep
        {l'_o}
          {\HeapUpdateOne{c}{@have@_F}{s_p}}
        {l_q}
          {s_q}
        {\HeapUpdateOne{x}{@chan@~c}{u'_o}})
        $ 
\\[1ex]

$\quad~|~\cs[c]=@in2@ ~\wedge~ s_p[c]=@none@_F ~\wedge~ s_q[c]=@none@_F$ & \note{SharedPullInject} \\
$\quad\to~\Just (@pull@~c~(@chan@~c)$ \\
@              @
      $\nextStep
        {l_p}
          {\HeapUpdateOne{c}{@pending@_F}{s_p}}
        {l_q}
          {\HeapUpdateOne{c}{@pending@_F}{s_q}}
        {[]}$
      \\
@              @
      $\nextStep
        {l_p}
          {\HeapUpdateOne{c}{@closed@_F}{s_p}}
        {l_q}
          {\HeapUpdateOne{c}{@closed@_F}{s_q}}
        {[]})
  $
\\[1ex]

$\quad~|~(\cs[c]=@in2@ \vee \cs[c]=@in1out1@) ~\wedge~ s_p[c]=@closed@_F$ & \note{SharedPullClosed} \\
$\quad\to~\Just (@jump@~
      \nextStep
        {l'_c}{s_p}
        {l_q}{s_q}
        {u'_c})
  $
\\[1ex]

$@close@~c~(l',u')$ \\
$\quad~|~\cs[c]=@out1@$ & \note{LocalClose} \\
$\quad\to~\Just (@close@~c~
      \nextStep
        {l'}{s_p}
        {l_q}{s_q}
        {u'})
    $
    \\

$\quad~|~\cs[c]=@in1out1@ ~\wedge~ s_q[c]=@none@_F$ & \note{SharedClose} \\
$\quad\to~\Just (@close@~c~
      \nextStep
        {l'}{s_p}
        {l_q}{\HeapUpdateOne{c}{@closed@_F}{s_q}}
        {u'})
    $ 
\\[1ex]

$@exit@$ & \note{LocalExit} \\
$\quad\to~\Just @exit@$
\\[1ex]

$\_~|~ @otherwise@ $ & \note{Blocked} \\
$\quad\to ~ \Nothing$
\end{tabular}

\caption{Fusion step for a single process of the pair.} 

% Given the state of both processes, compute the instruction this process can perform. This is analogous to statically evaluating the pair of processes. If this process cannot execute, the other process may still be able to.
\label{fig:Fusion:Def:Step}
\end{figure*}





% -------------------------------------
\smallskip
Fig.~\ref{fig:Fusion:Def:Step} defines function \ti{tryStep} which schedules a single instruction. This function takes the map of channel types, along with the current label and associated instruction of the first (left) process, and the current label of the other (right) process.

Clause (LocalJump) applies when the left process wants to jump.
In this case, the result instruction simply performs the corresponding jump, leaving the right process where it is. 

Clause (LocalCase) is similar, except that there are two destination labels.

Clause (LocalPush) applies when the left process wants to push to a non-shared output channel.
In this case the push can be performed directly, with no additional coordination required.

Clause (SharedPush) applies when the left process wants to push to a shared channel. Pushing to a shared channel requires the downstream process to be ready to accept the value at the same time. We encode this constraint by requiring the static input state of the downstream channel to be $@none@_S$. When this is satisfied, the result instruction stores the pushed value in the stream buffer variable $@chan@~c$'' and sets the static input state to ``$@pending@_S$'', which indicates that the new value is now available. 

Still in Fig.~\ref{fig:Fusion:Def:Step}, clause (LocalPull) applies when the left process wants to pull from a local channel, which requires no coordination.

Clause (SharedPull) applies when the left process wants to pull from a shared channel that the other process either pulls from or pushes to. We know that there is already a value in the stream buffer variable, because the state for that channel is $@pending@_S$. The result instruction copies the value from the stream buffer variable into a variable specific to the left source process, and the corresponding $@have@_S$ channel state in the result label records that it has done so.

Clause (SharedPullInject) applies when the left process wants to pull from a shared channel that both processes pull from, and neither already has a value. The result instruction is a @pull@ that loads the stream buffer variable.

Clause (LocalDrop) applies when the left process wants to drop the current value from that it read from an unshared input channel, which requires no coordination.

Clause (ConnectedDrop) applies when the left process wants to drop the current value that it received from an upstream process. As the value will have been sent via a heap variable instead of a still extant channel, the result instruction just performs a @jump@ while updating the static channel state.

Clauses (SharedDropOne) and (SharedDropBoth) apply when the left process wants to drop a value it gained from a channel which is shared by the other process. In (SharedDropOne) the channel states reveal that the other process is still using the value. In this case the result instruction is a @jump@ that only updates the static information that the left source process no longer requires this value. In (SharedDropBoth) then channel states reveal that the other process no longer needs the value either. In this case the result instruction is a real @drop@, because we are sure that neither process requires the value any longer.


% -------------------------------------
%!TEX root = ../Main.tex
\begin{figure}

\begin{tabbing}
$\ChanType_2$   \TABDEF \kill

\ti{channels} \> $:$ \> $\Proc \to \Proc \to \MapType{\Chan}{\ChanType_2}$ \\

  \> $=$    \> $\{ c=@MMMMMMM@~$\= \kill
$\ti{channels}~p~q$
  \> $=$    \> $\{ c=@in2@$
            \> $|~c~\in~(@ins@~p~\cap~@ins@~q) \}$ 
            \\

  \> $\cup$ \> $\{ c=@in1@$
            \> $ |~c~\in~(@ins@~p~\cup~@ins@~q)~\wedge~c~\not\in(@outs@~p~\cup~@outs@~q) \}$ \\

  \> $\cup$ \> $\{ c=@in1out1@$
            \> $|~c~\in~(@ins@~p~\cup~@ins@~q)~\wedge~c~\in(@outs@~p~\cup~@outs@~q) \}$ \\

  \> $\cup$ \> $\{ c=@out1@$
            \> $ |~c~\not\in~(@ins@~p~\cup~@ins@~q)~\wedge~c~\in(@outs@~p~\cup~@outs@~q) \}$ 
\end{tabbing}

\newcommand\funClauseDef[3]
{ $\ti{#1}~(#2)$ \> $=$ \> $#3$
}
\newcommand\outlabelsDef[2]
{ \funClauseDef{outlabels}{#1}{\sgl{#2}} 
}

\vspace{1ex}
\begin{tabbing}
$\ti{outlabels}~(@case@~e~(l_t,u_t)~(l_t,u_f))$ \TABSKIP $=$ \TABSKIP \kill
\ti{outlabels} \> $~:$ \> $\Instr \to \sgl{\Label}$ \\
\outlabelsDef{@pull@~c~x~(l,u)~(l',u')}{l,l'}              \\
\outlabelsDef{@drop@~c~(l,u)}{l}                \\
\outlabelsDef{@push@~c~e~(l,u)}{l}              \\
\outlabelsDef{@close@~c~(l,u)}{l}              \\
\outlabelsDef{@case@~e~(l,u)~(l',u')}{l, l'}    \\
\outlabelsDef{@jump@~(l,u)}{l}                  \\
\outlabelsDef{@exit@}{}
\end{tabbing}

\vspace{1ex}
\begin{tabbing}
$\ti{swaplabels}~(@case@~e~((l_1,l_2),u)~((l'_1,l'_2),u'))$ \TABSKIP $=$ \TABSKIP \kill
\ti{swaplabels} \> $~:$ \> $\Instr \to \Instr$ \\
\funClauseDef{swaplabels}
  {@pull@~c~x~((l_1,l_2),u)~((l'_1,l'_2),u')}
  {@pull@~c~x~((l_2,l_1),u)~((l'_2,l'_1),u')}    \\
\funClauseDef{swaplabels}
  {@drop@~c~((l_1,l_2),u)}
  {@drop@~c~((l_2,l_1),u)}      \\
\funClauseDef{swaplabels}
  {@push@~c~e~((l_1,l_2),u)}
  {@push@~c~e~((l_2,l_1),u)}    \\
\funClauseDef{swaplabels}
  {@close@~c~((l_1,l_2),u)}
  {@close@~c~((l_2,l_1),u)}    \\
\funClauseDef{swaplabels}
  {@case@~e~((l_1,l_2),u)~((l'_1,l'_2),u')}
  {@case@~e~((l_2,l_1),u)~((l'_2,l'_1),u')}     \\
\funClauseDef{swaplabels}
  {@jump@~((l_1,l_2),u)}
  {@jump@~((l_2,l_1),u)}  \\
\funClauseDef{swaplabels}
  {@exit@}
  {@exit@}
\end{tabbing}

\caption{Utility functions for fusion}
\label{fig:Fusion:Utils}
\end{figure}



\smallskip
Fig.~\ref{fig:Fusion:Utils} contains definitions of some utility functions which we have already mentioned.
Function \ti{channels} computes the $\ChanType_2$ map for a pair of processes.
Function \ti{outlabels} gets the set of output labels for an instruction, which is used when computing the fixpoint of reachable states.
Function \ti{swaplabels} flips the order of the compound labels in an instruction.
