\section{Constructing a process network}
\label{s:extraction:grepGood}

The following example constructs a process network in Folderol.
In due course we will inspect its generated code, but first it is necessary to see how process networks are constructed.

\begin{lstlisting}
grepGood :: FilePath -> FilePath -> IO ()
grepGood fileIn fileOut =
  $$(fuse $ do
     input  <- source [||sourceOfFile fileIn ||]
     goods  <- filter [||isPrefixOf   "Good" ||] input
     sink      goods  [||sinkToFile   fileOut||])
\end{lstlisting}

This example reads input lines from a file, filters out all except those starting with the string \lstinline/"Good"/, and finally writes the results to file.
In a Unix-style environment, we would write this as ``@grep ^Good@''; the caret (@^@) meaning ``starting with''.

The function @grepGood@ takes two arguments of type @FilePath@ for the input and output filenames.
Inside the definition, the Template Haskell splice \lstinline/$$(fuse ...)/ takes the process network as an argument, fuses the processes together, and generates output code.
We are constructing the process network inside the Template Haskell splice at compile-time, while we wish to execute it at runtime.
The process network is the static representation of the computation, and must be statically known and finite.

Inside the process network we start by creating a \emph{source} to read from the input file @fileIn@.
The @source@ function creates an input stream in the process network which can be used by other processes.
The @input@ binding in this case refers to the abstract name of the stream in the process network, rather than the runtime values of the stream.
This is an important distinction, as expressions operating over runtime values need to be quasiquoted to delay them from compile-time to runtime.
Similarly, the input file @fileIn@ will not be known until runtime.
The choice of source does not affect fusion, and does not need to be known at compile time.
Since most sources depend on runtime values in some way, the entire source is delayed until runtime and must be quasiquoted.

Now we take the values in the @input@ stream and filter them to those starting with the string \lstinline/"Good"/.
Again considering the compile-time/runtime distinction, the fact that the @filter@ process uses the stream named @input@ as its input is known at compile-time, and is not quasiquoted; while the predicate, which depends on the runtime stream values, must be quasiquoted.
We call the output filtered stream @goods@.

Finally, we send the filtered output to a file, by creating a \emph{sink}.
The @sink@ function is the opposite of @source@, and just like @source@ it requires the description of how to sink (writing to a file using @sinkToFile@) to be quasiquoted.

\TODO{Show the process network and the processes.}
Soon enough, we shall return to this.


% \begin{lstlisting}
% applyTransactions :: FilePath -> FilePath -> IO ()
% applyTransactions fileIn fileOut =
%   $$(fuse $ do
%      cust  <- source [||sourceOfFile fileCust||]
%      txns  <- source [||sourceOfFile fileTxns||]
%      cust' <- map    [||parseCust            ||] cust
%      txns' <- map    [||parseTxns            ||] txns
% 
%      (newCust, invalid) <- groupLeft [||applyTxn||] cust' txns'
% 
%      sink newCust [||sinkToFile fileOutCust   ||]
%      sink invalid [||sinkToFile fileOutInvalid||])
% \end{lstlisting}

% We start with the Template Haskell splice \lstinline/$$(fuse ...)/. In the code it is blue. 
% It has the following type.
% \begin{lstlisting}
% fuse :: Network () -> Q (TExp (IO ()))
% \end{lstlisting}
% That is, it takes a process network and returns the expression for the underlying @IO@ computation.
% The process network @Network ()@ is a monad as well.
% 
% The process network first constructs a source that reads from a file.
% \begin{lstlisting}
% source :: Q (TExp (Source a))                -> Network (Channel a)
% filter :: Q (TExp (a -> Bool))  -> Channel a -> Network (Channel a)
% map    :: Q (TExp (a -> b))     -> Channel a -> Network (Channel b)
% sink   :: Q (TExp (Sink a))     -> Channel a -> Network (Channel a)
% \end{lstlisting}
% The @source@ function takes a quasiquoted expression of how to construct the source at runtime.
% 
% Now show the generated code.
% 




