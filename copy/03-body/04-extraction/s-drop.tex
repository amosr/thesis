\section{Drop in synchrony}
\label{s:Drop:in:synchrony}
\TODO{Elsewhere.}

The @drop@ instructions exist to synchronise two consumers of the same input, so that both processes pull the same value at the same time.
When fusing two consumers, the fusion algorithm uses drops when coordinating the processes to ensure that one consumer cannot start processing the next element until the other has finished processing the current element.
Drop instructions are not necessary for correctness, or even for ensuring boundedness of buffers, but as we shall see they can lead to smaller processes.

The following example, @map2par@, has one input stream which is transformed twice.
The `@par@' part of the name is because the two @map@ processes are in parallel as opposed to series; both processes consume the same input.
This example is very similar to the @map2@ from earlier.
The processes are the same, except for the input to the second @map@, and there is an extra sink.

\begin{lstlisting}
map2par = $$(fuse $ do
  a <- source [||sourceOfFile "in.txt"||]
  b <- map    [|| f ||] a
  c <- map    [|| g ||] a
  sink b      [||sinkToFile   "b.txt"||]
  sink c      [||sinkToFile   "c.txt"||])
\end{lstlisting}

This function is represented by the following process network.

\begin{lstlisting}[linebackgroundcolor={
  \hilineFst{9}
  \hilineFst{10}
  \hilineFst{11}
  \hilineFst{12}
  \hilineFst{13}
  \hilineSnd{18}
  \hilineSnd{19}
  \hilineSnd{20}
  \hilineSnd{21}
  \hilineSnd{22}
  }]
Process network:
  Sources: a = sourceOfFile "in.txt"
  Sinks:   b = sinkToFile   "b.txt"
           c = sinkToFile   "c.txt"
  Processes:
    Process "map f a":
      Inputs:  a; Outputs: b; Initial: b0
      Instructions:
        b0   = Pull  a b1 b3
        b1 e = Push  b (f e) b2
        b2   = Drop  a b0
        b3   = Close b b4        
        b4   = Done

    Process "map g a":
      Inputs:  a; Outputs: c; Initial: c0
      Instructions:
        c0   = Pull  a c1 c3      
        c1 e = Push  c (g e) c2     
        c2   = Drop  a c0
        c3   = Close c c4         
        c4   = Done
\end{lstlisting}

\autoref{figs/swim/map2par} shows an example execution of @map2par@ with the input stream containing the elements $[1, 2]$.
This execution shows the communication between concurrent processes, where the input and output streams are also treated as concurrent processes.
The input stream pushes to both map processes at the start and cannot push again until after both map processes have dropped.
When the input stream pushes into the map processes, the value is stored in a single-element buffer in each map process.
The input stream can only push when both its consumers' buffers are empty, and the pull instruction on each map process can only execute when its single-element buffer is not empty.
The drop instruction clears the buffer, making room for the producer to push the next value.
\TODO{Would it be clearer if drop instructions were represented as sending a message back to the producer? This picture is missing some of the communication.}
At any point in time, both map processes are transforming the same element as each other because the next element is only available once both have dropped and agreed to read the next element.

\autoref{figs/swim/map2par-no-sync} shows a hypothetical execution if our process network semantics did not use drop instructions to synchronise between all consumers.
The previous example showed concurrent execution between the processes, while this example shows a particular interleaving to highlight a case where one consumer overtakes the other by one element.
As before, execution starts with the input stream pushing an element, and both map processes pull this element.
Without drop synchronisation, the single-element buffer is cleared as soon as the process pulls, which means the input stream is free to push another element since both processes have pulled.
Now the first map process executes and transforms both pushed elements.
The first map process is finished with both elements before the second map process has even looked at the first.
This is not a problem for concurrent execution: the execution results in the same value, and the buffer is still bounded.
However, when we fuse this network into a single process we do not wish to use this interleaving, because the process would need to keep track of two consecutive elements at the same time.
Keeping both elements means the process requires twice the number of live variables, which makes it less likely that both elements will fit in the available registers or cache.

\TODO{Other options:} what if the producer could only push when all consumers are trying to pull?
This is more restrictive and causes deadlock in @zip x y / zip y x@ case.

\TODO{Size:} show fused process with and without drop synchronisation. Drop synchronisation makes it smaller.

\TODO{SIMD?:} the bit above says we don't want to keep multiple consecutive elements at the same time.
Maybe we do want to, in order to use SIMD instructions.
In fact, for \autoref{figs/swim/map2par-no-sync} if we had vector instructions to compute @f@ and @g@ two elements at a time, that interleaving is exactly what we want.
There are many other interleavings though, and we are not guaranteed to get this one.
We do not want to rely on chance to find a SIMD interleaving.
Future work may involve looking for the right interleavings to exploit SIMD instructions.


\begin{figure}
\center
\begin{sequencediagram}
\newthreadGAP{a}{@a@}{0.2}
\newthreadGAP{mapf}{@map f@}{1.5}
\newthreadGAP{b}{@b@}{1.5}
\newthreadGAP{mapg}{@map g@}{1.5}
\newthreadGAP{c}{@c@}{1.5}

\parallelseq{
  \mess{a}{}{mapf}
}{
  \mess{a}{Push 1}{mapg}
}

\parallelseq{
  \mess{mapf}{Pull 1}{mapf}
  \mess{mapf}{Push (f 1)}{b}
  \mess{mapf}{Drop}{mapf}
}{
  \mess{mapg}{Pull 1}{mapg}
  \mess{mapg}{Push (g 1)}{c}
  \mess{mapg}{Drop}{mapg}
}

\parallelseq{
  \mess{a}{}{mapf}
}{
  \mess{a}{Push 2}{mapg}
}

\parallelseq{
  \mess{mapf}{Pull 2}{mapf}
  \mess{mapf}{Push (f 2)}{b}
  \mess{mapf}{Drop}{mapf}
}{
  \mess{mapg}{Pull 2}{mapg}
  \mess{mapg}{Push (g 2)}{c}
  \mess{mapg}{Drop}{mapg}
}
\end{sequencediagram}
\caption[Concurrent sequence diagram for parallel maps]{sequence diagram for concurrent execution of @map2par@. }
\label{figs/swim/map2par}
\end{figure}

\begin{figure}
\center
\begin{sequencediagram}
\newthreadGAP{a}{@a@}{0.2}
\newthreadGAP{mapf}{@map f@}{1.5}
\newthreadGAP{b}{@b@}{1.5}
\newthreadGAP{mapg}{@map g@}{1.5}
\newthreadGAP{c}{@c@}{1.5}

\parallelseq{
  \mess{a}{}{mapf}
}{
  \mess{a}{Push 1}{mapg}
}

\mess{mapf}{Pull 1}{mapf}

\mess{mapg}{Pull 1}{mapg}

\parallelseq{
  \mess{a}{}{mapf}
}{
  \mess{a}{Push 2}{mapg}
}

\mess{mapf}{Push (f 1)}{b}
\mess{mapf}{Pull 2}{mapf}
\mess{mapf}{Push (f 2)}{b}

\mess{mapg}{Push (g 1)}{c}
\mess{mapg}{Pull 2}{mapg}
\mess{mapg}{Push (g 2)}{c}
\end{sequencediagram}
\caption[Hypothetical sequence diagram for parallel maps without drop synchronisation]{sequence diagram for parallel execution of two map processes, using a hypothetical execution semantics without drop synchronisation.}
\label{figs/swim/map2par-no-sync}
\end{figure}


% By synchronising the two processes together, when we fuse we will only have one copy of the code that pulls each element.
% Because @P@ can only start pulling again by the time @Q@ has dropped, this means @P@ and @Q@ must both be trying to pull at the same time, which means we can reuse the same instructions generated from the previous time they both pulled.
% The example @PQ_drop@ shows the fused process with drop instructions.
%                                     ~~~~~[BL: highlight?]~~~
% Note that there is only one copy of each input process' code.
% On the other hand, @PQ_no_drop@ shows the fused process without drop instructions.
%                           ~~~~~~[BL: highlight?]~~~~~~~
% In @PQ_no_drop@ there are two copies of pushing to @CP@, though the main loop only executes one per iteration: pushing the current element to @CP@, and the previous element to @CQ@.
%                                                                                                     ~~~~[BL: examples]~~~~
% As the processes get larger, and more processes are fused together, the issue of duplicating code becomes more serious.
% There are two parts to this: first, we need to hold the entire process in memory to generate its code.
% Secondly, as the generated assembly code gets larger, it is less likely to fit into the processor's cache.
% Smaller code is generally better for performance.
% Having two copies of the push to @CP@ means that any consumers of @CP@ must in turn have their code duplicated, with the pull instructions from @CP@ copied into both sites of the pushes.
% 
% \TODO{diagrams}
% \begin{lstlisting}
% PQ_drop = process
%   P1Q1: c_buf <- pull C
%         c_p    = c_buf
%         c_q    = c_buf
%         push CP c_p
%         push CQ c_q
%         drop C
%         jump P1Q1
% 
% PQ_no_drop = process
%   P1Q1:  c_buf <- pull C
%          c_p    = c_buf
%          c_q    = c_buf
%          push CP c_p
%   P2Q2:  c_buf <- pull c
%          c_p    = c_buf
%          push CP c_p
%          push CQ c_q
%          c_q    = c_buf
%          jump P2Q2
% \end{lstlisting}
% 
% Without drops, the result program is still correct, but one process can overtake another by one element.
% Overtaking leads to larger processes because there end up being two copies of the processing code.
% 
% 
% 

