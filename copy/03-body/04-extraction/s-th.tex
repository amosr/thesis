\section{Template Haskell}
Template Haskell is a metaprogramming extension for Haskell \cite{sheard2002template} which provides a limited form of staged computation, where the only `stages' are compile-time and runtime.
There are two variants of Template Haskell: untyped and typed.
Untyped Template Haskell does not enforce well-typedness of the generated program until after the program is generated, which means it is quite easy to generate ill-typed programs.
Even though we can generate ill-typed programs, overall soundness is preserved because the generated program is still typechecked and compiled at compile-time.
On the other hand, typed Template Haskell enforces well-typedness of the generated program by lifting types in the generated program up to the meta-level as types in the generator program.

We use untyped Template Haskell for code generation.
Performing typed code generation is equivalent to proving that the result program is well-typed, whereas untyped code generation requires no such proof.
This makes the untyped version easier to implement.
While typed code generation is no doubt possible, it is an interesting research problem on its own.

The problem with using untyped code generation is that if there are type errors, they will not be found until quite late, once the generated code has been spliced into the main program.
This means that when the programmer sees the type error, the error location will be in the generated code.
That is, the error is shown in code the program wrote, rather than code the programmer wrote.
This complicates finding the underlying cause of the error and fixing the problem.
Type safety is a big problem when we wish to provide a library for other programmers: it is unacceptable to require a user of the library to understand the internal workings of code generation to understand type errors.
Typed Template Haskell allows us to provide a type-safe interface to users of the library.

\subsection{Untyped expressions}

Template Haskell extends regular Haskell with two syntactic constructs: quasiquoting and splicing, for moving between runtime and compile-time stages.
Quasiquoting converts a runtime expression to a compile-time value, while splicing performs the opposite conversion.
For example, we can quasiquote some arithmetic using the syntax \lstinline/[|1+2|]/, which produces the corresponding abstract syntax tree for an infix operator with two integer arguments:

\begin{lstlisting}
InfixE (Just (LitE (IntegerL 1))) (VarE GHC.Num.+) (Just (LitE (IntegerL 2)))
\end{lstlisting}

Quasiquoting is a purely syntactic convenience, as any quasiquoted expressions could be produced using the abstract syntax tree constructors directly.
That said, it is a particularly useful convenience which we will use extensively.

Splicing takes a compile-time abstract syntax tree and converts it to an expression to be evaluated at runtime.
For example, splicing back the quasiquoted arithmetic using the syntax \lstinline/$([|1+2|])/, will evaluate (@1+2@) at runtime.

Splicing and quasiquoting operate in the @Q@ monad.
The @Q@ monad gives a fresh name supply, so that when names are bound inside expressions they can be given unique names.
These fresh names ensure that newly bound names will not interfere with existing bindings.

Now let us look at a concrete example of how to use Template Haskell.
Suppose we wish to define a power function where the exponent is known at compile time, but the mantissa is not known until runtime.

We define @power@ as a function with the type (\lstinline/Int -> Q Exp/).
This means it takes an integer at compile-time, and produces a quoted expression in the quote monad.
This result expression will, once spliced in, have type (\lstinline/Int -> Int/).
Later, we shall use typed Template Haskell to make this extra type information explicit, but for now we must perform typechecking in our heads.

\begin{lstlisting}
power :: Int -> Q Exp
power 0 = [|\m -> 1                      |]
power e = [|\m -> $(power (e - 1)) m * m |]
\end{lstlisting}

The @power@ function pattern-matches on the exponent.
When the exponent is zero, we enter quasiquoting mode and construct a function that always returns one.
When the exponent is non-zero, we again enter quasiquoting mode and construct a function, taking the mantissa as its argument.
Inside the quasiquote, we need to handle the recursive case for the one-smaller exponent (\lstinline/e - 1/), so we enter into splicing mode with \lstinline/$(power (e - 1))/ to compute the one-smaller power.
The splice for the one-smaller power returns a function, to which we apply the mantissa.
Finally, we multiply the one-smaller power of the mantissa by the mantissa itself.

We can now define a specialised @power@ function to compute the square.
We define this as a top-level binding which performs a splice, and inside that splice we call @power@ with the statically known argument @2@.
The type inside the splice is \lstinline/Q Exp/, and once splicing is complete the result is an expression of type \lstinline/Int -> Int/.

\begin{lstlisting}
power2 :: Int -> Int
power2 = $(power 2)
\end{lstlisting}

The reason for this specialisation is to produce optimised code: by knowing the exponent at compile-time, we can perform the recursion once at compile-time rather than many times at runtime.
It is therefore worth inspecting the resulting code to check whether it is indeed optimal.
The compiler option (@-ddump-splices@) outputs the result splices, as follows.

\begin{lstlisting}
$(power 2) =
  \m0 -> (\m1 -> (\m2 -> 1) m1 * m1) m0 * m0
\end{lstlisting}

This is a roundabout way to multiply a number by itself, and there are a lot of opportunities to simplify that code.
While we expect the compiler to remove the extra lambdas by beta reduction, it would be even better to not introduce them in the first place.
If we do not introduce simplification opportunities in the first case, there is no uncertainty about whether the compiler will be able to remove them.
The problem is that @power@ introduces a lambda for each recursive step, while we only want one lambda at the top-level.
So let us fix this by defining a top-level function which introduces the lambda, and a recursive helper function to compute the power.

\begin{lstlisting}
powerS :: Int -> Q Exp
powerS e = [|\m -> $(powerS' e [|m|])|]
\end{lstlisting}

The top-level function @powerS@ introduces the lambda for the mantissa inside a quasiquote, then calls the helper function @powerS'@.
The exponent is a compile-time binding, while the mantissa is a runtime binding.
When the helper function is called at compile-time, the exponent can be passed as-is, while the mantissa must be quasiquoted to wrap the runtime binding into a compile-time expression.

The recursive helper function, @powerS'@, has type \lstinline/Int -> Q Exp -> Q Exp/.

\begin{lstlisting}
powerS' :: Int -> Q Exp -> Q Exp
powerS' 0 m = [|1                        |]
powerS' e m = [|$(powerS' (e-1) m) * $(m)|]
\end{lstlisting}

Like the original @power@ function, it pattern-matches on the exponent and in the recursive case multiplies by itself.
The difference is that the mantissa is bound as a compile-time expression with type (@Q Exp@) rather than inside the quasiquote, so it must be spliced when it is used.

The output for (@powerS 2@) is a lot simpler than that for (@power 2@).
Whereas before there was a lambda introduced and applied at each recursive step, now there is only a single lambda.

\begin{lstlisting}
$(powerS 2) =
  \m -> ((1 * m) * m)
\end{lstlisting}

There is still more we could do to improve the function: for example, (@1 * m@) could be replaced by @m@.
However, this is sufficient to show the core splicing and quoting ideas behind Template Haskell.
For more information on staging in general, \citet{rompf2010lightweight} takes this example further.

\subsection{Typed expressions}

% The problem with Template Haskell shown above is that there are no types attached to expressions.
% Quasiquoting a string \lstinline/[|"one"|]/ and quasiquoting an integer \lstinline/[|1|]/ both produce a value of the same type: @Q Exp@.
% We can then use these expressions to construct larger, completely untypable expressions; for example we could try to subtract a string from an integer, which should surely fail: \lstinline/[|1 - "one"|]/.

Typed Template Haskell extends the Template Haskell we have seen with typed expressions, typed splicing and typed quasiquoting.
The type of typed expressions, @TExp@, is annotated with a meta-level (compile-time) type argument denoting the object-level type of the expression.
Syntactically, typed quasiquotation uses two pipe symbols, where untyped quasiquotation uses one.
Likewise, typed splicing uses two dollar signs.
For example, using a typed quasiquote on a string will produce a @String@-typed expression: (\lstinline/[||"one"||] :: Q (TExp String)/).
Similarly, typed splicing eliminates the @Q@ monad and the @TExp@ wrapper, leaving only the result type: (\lstinline/$$([||"one"||]) :: String/).

Typed expressions are invaluable for providing a typesafe way to construct process networks.
We need to construct process networks at compile-time to fuse them at compile-time, while keeping the information required to execute them at runtime.
Consider the standard @filter@ function for lists, with type (\lstinline/(a -> Bool) -> [a] -> [a]/).
In order to implement a process network version of @filter@, the function argument of type (\lstinline/(a -> Bool/) must be converted to an expression, as it will be evaluated and applied at runtime.
Providing a process network @filter@ function which takes as an argument the typed expression (\lstinline/Q (TExp (a -> Bool))/) instead of the untyped expression (\lstinline/Q Exp/) means that type errors can be caught early.

We provide an untyped core for the processes and networks, then build on top of this to provide a typed interface for constructing networks.
For this reason, we need to be able to convert from typed expressions to untyped expressions.
We can convert from a typed expression to an untyped expression using the @unTypeQ@ function, which has type (\lstinline/Q (TExp a) -> Q Exp/).
This does not affect the underlying expression, it just throws away the meta-level type information.
The object-level type information remains, and once it is spliced in it will have the same type.

% Conversely, if one is very careful, one can convert an untyped expression to a typed one.
% This is an unsafe operation, because one can choose any type at all for the expression.
% The expression is not checked against the chosen type until it is spliced in.
% \begin{lstlisting}
% unsafeTExpCoerce :: Q Exp -> Q (TExp a)
% \end{lstlisting}
 
% By using untyped expressions for code generation, we do not lose any actual type safety, since the generated code will still end up being typechecked by the Haskell compiler.
% What we do lose are good error locations, but these errors will only occur if there are bugs in the code generator.
% Any type errors in the user of the library will be using the typesafe interface, with better error messages.

% Although it is possible to construct expressions with the wrong type, this generally requires explicitly unsafe operations.
% Most of the time, this is unlikely to occur by accident alone.


