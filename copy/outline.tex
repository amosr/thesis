\section{Introduction}

To learn interesting things from large datasets, we generally want to perform lots of queries.
When our query is small and our data is big, we might spend more time reading the data than we spend computing the answer.
In this case, we would like to amortise the cost of reading the data by performing multiple queries at the same time.

When querying datasets that do not fit in memory or disk, it can be hard to ensure that our query program's working set will fit in memory.
One way to transform large datasets in constant memory is to write the query as a \emph{streaming program}, which we can write by composing stream transformers together.
Composing stream transformers together adds some performance overhead, which is usually removed by \emph{fusing} together multiple transformers into a single transformer.

% The most common stream representation is a pull stream [ref]. Pull streams do not support multiple queries.

This thesis concerns low-overhead streaming models for executing multiple queries at a time.
We focus on two streaming models: push streams [ref taxonomy/push], and Kahn process networks [ref taxonomy/kpn].

Push streams support multiple queries, but these queries can be quite unwieldy to write as they must be constructed ``back-to-front'' [ref taxonomy/push/backwards].
In [ref icicle] we introduce a query language called Icicle, which allows programmers to write and reason about queries using a more familiar list-based semantics, while retaining the execution strategy of push streams.
The typesystem of Icicle aims to ensure that well-typed query programs have the same semantics whether they are executed as list programs or as stream programs.
% Icicle is a streaming query language, which means that queries can only perform a single pass over the input data.
% The Icicle compiler translates queries written in Icicle to push streams, and fuses multiple queries together to execute multiple queries at a time.

However, push streams do not support computations with multiple inputs except for non-deterministically merging two streams, and in some circumstances appending streams [ref taxonomy/push/many-inputs].
Kahn process networks support both multiple inputs and multiple queries, but require dynamic scheduling and inter-process communication, which can introduce significant overhead.
In [ref kpn/fusion] we introduce a method for taking multiple processes in a Kahn process network and fusing them together into a single process.
This fusion method generalises previous work on stream fusion [ref kpn/generalises] and demonstrates the connection between fusion and synchronised product of processes [ref kpn/synchronised-product], which is generally used as a proof technique rather than an optimisation.

[todo: clustering]
% \subsection{Outline}
% \subsection{Contributions}

\section{A brief taxonomy of streaming models}
There are many different streaming models, each with unique advantages and disadvantages.
This chapter classifies the most common and fundamental streaming models.
More

\subsection{Gold-panning}
A diverse and plausible collection of queries which a query programmer may desire to execute together.
To compare streaming models, we introduce a concrete example of queries we wish to run.

\subsection{Pull streams}
Pull streams support multiple inputs, but only a single query.

\subsection{Push streams}
Push streams support multiple queries but only one input, and must be written backwards.

@data Push a b = (a -> IO (), IO b)@

\subsection{Polarised streams}
Polarised streams support multiple inputs and queries, but must be manually separated into "pull" and "push" parts.

@unzip_ioi : Pull (a,b) r1 -> Push b r2 -> Pull a (r1,r2)@

\subsection{Kahn Process Networks}
Kahn Process Networks support multiple inputs and multiple queries, but have significant scheduling and communication overhead.

@data KPN = MVar …@

\section{Write many, run once}
Multiple queries with push streams
* the problem with pushing
* a rush, a push


\section{Related work}
Other models are interesting, but not necessarily applicable:

\subsection{The fusion system for streams named Stream Fusion}

\subsection{Push-pull streams}
Push-pull streams are sometimes called push streams, but support neither multiple inputs nor multiple queries.

@PushPull a r = Push (Pull a r') (r,r')@

\subsection{Iteratee, conduit and pipes}
Iteratee is a monadic representation of a push stream.
\begin{lstlisting}
enumPair :: Monad m => Iteratee el m a -> Iteratee el m b -> Iteratee el m (a,b)
enumPair i1 i2 = enum2 i1 i2 >>= runI2
\end{lstlisting}

\begin{lstlisting}
i1 :: Monad m => Iteratee el1 (Iteratee el2 m) (el1, el2)
i1 = do
     e1 <- head 
     e2 <- lift head
     return (e1,e2)
\end{lstlisting}


, Conduit and Pipes are monadic representations of pull streams.
They get around the linearity problem of reusing stream names, to some extent, by not allowing streams to be named.
This monadic representation allows a small set of primitives for reading from straight lines, emitting values etc, which can be composed together.
Implementing ``map'' for pull streams requires delving into the details of the stream representation, while implementing map for iteratee-style streams can be done with no knowledge of the underlying implementation.

This monadic interface allows the structure of the computation graph to depend on the values.
This expressiveness has a price: if the computation graph can change dynamically, we cannot statically fuse it.
Cite Arrows paper about how monads mix up the static and dynamic parts of a system, reducing the ability to optimise based on static information.
To work around these performance penalties, Conduit supports short-cut fusion for some operations.
Cerain operations are implemented using non-monadic pull streams, which can be converted to Conduit streams.
Rewrite rules then remove superfluous conversions between Conduit streams and pull streams.

A pedantic note on terminology: Conduit uses the term \emph{fusion} for connecting two stream transformers together, while in this thesis we use fusion to mean removing the overhead of connecting two streams together.
(So what --- is this even worth mentioning?)


\subsection{Synchronous streams}
Synchronous streams support multiple inputs and multiple queries but require all streams to be defined at the same time, precluding "append" and "merge".

@data Sync a = Time -> a@

