
\section*{Timeline}

\newcommand\when[2] {: \emph{#1\% done; finish in #2 weeks}}
\newcommand\done {: \emph{done}}
\newcommand\ugh {\when{75}{16}}
\newcommand\uns[1] {\when{0}{#1}}
\newcommand\pap[2] {\when{exists as paper; #1}{#2}}


\begin{enumerate}
\item Introduction \ugh
  \begin{enumerate}
  \item Outline \ugh
  \item Contributions \ugh
  \end{enumerate}

\item Streaming models background \ugh
  \begin{enumerate}
  \item An example of a streaming program \done
  \item Pull streams \done
  \item Push streams \done
  \item Polarised streams \when{80}{1}
  \item Streaming overhead \when{50}{1}
  \end{enumerate}

\item Icicle: push streams \pap{50}{4}
  \begin{enumerate}
  \item Elements and aggregates \pap{50}{2}
    \begin{enumerate}
    \item Stage restriction \pap{50}{2}
    \item Finite streams and synchronous data flow \pap{50}{2}
    \item Incremental update \pap{50}{3}
    \item Bounded buffer restriction \pap{50}{3}
    \item Source language \pap{50}{3}
      \begin{enumerate}
      \item Type system \pap{50}{3}
      \item Evaluation semantics \pap{50}{4}
      \end{enumerate}
    \end{enumerate}
  \item Intermediate language \pap{50}{4}
  \item Benchmarks \pap{50}{4}
  \item Related work \pap{50}{4}
  \end{enumerate}

\item Kahn process networks as a streaming model
  \begin{enumerate}
  \item Processes and networks \pap{50}{5}
    \begin{enumerate}
    \item FilterMax \pap{50}{5}
    \item Process definition \pap{50}{5}
    \item Execution \pap{50}{5}
      \begin{enumerate}
      \item Injecting \pap{50}{6}
      \item Advancing \pap{50}{6}
      \item Feeding \pap{50}{6}
      \end{enumerate}
    \item Non-deterministic evaluation order \pap{50}{6}
    \end{enumerate}

  \item Statically fusing processes together
    \begin{enumerate}
    \item Synchronised product \pap{50}{6}
    \item Fusion type definitions \pap{50}{6}
    \item Fusion of pairs of processes \pap{50}{7}
    \item Fusion step coordination for a pair of processes \pap{50}{7}
    \item Fusion step for a single process of a pair \pap{50}{7}
    \item Fusing a network \pap{50}{7}
    \item When we cannot fuse \pap{50}{7}
    \end{enumerate}

  \item Proving it correct \uns{8}
    \begin{enumerate}
    \item Mechanised soundness \uns{8}
    \item Not complete \uns{8}
    \end{enumerate}

  \item Evaluation benchmarks \when{90}{9}
    \begin{enumerate}
    \item Gold-panning \when{90}{9}
    \item QuickHull \done
    \item Dynamic range compression \done
    \item File operations \done
    \item Partition / append \done
    \item Result size \pap{50}{9}
    \end{enumerate}

  \item Implementation and code generation \done
    \begin{enumerate}
    \item Template Haskell \done
    \item Constructing a process network \done
    \item All this boxing and unboxing \done
      \begin{enumerate}
      \item Mutable references \done
      \item Constructor specialisation \done
      \end{enumerate}
    \item Sources and sinks \done
    \item Code generation \done
    \item Vector endpoints \done
    \item Size hints \done
    \item Transforming process networks \done
      \begin{enumerate}
      \item Fusing a network \done
      \item Jump contraction \done
      \item Cull outputs \done
      \end{enumerate}
    \item Synchronised dropping \done
    \end{enumerate}

  \item Related work \when{50}{12}
    \begin{enumerate}
    \item Kahn process networks \when{80}{10}
    \item Polarised streams \when{90}{10}
    \item Active and passive streaming \when{0}{11}
    \item Online and offline \when{0}{11}
    \item Monadic streaming \when{50}{11}
    \item Comparison \when{0}{12}
    \end{enumerate}

  \end{enumerate}

\item Clustering over multiple passes \pap{50}{13}
  \begin{enumerate}
  \item Clustering introduction \pap{50}{13}
  \item Combinators \pap{50}{13}
  \item Size inference \pap{50}{14}
  \item Integer linear programming \pap{50}{14}
  \item Results \pap{50}{14}
  \item Related work \pap{50}{14}
  \end{enumerate}

\item Conclusion \when{0}{15}
\item Bibliography \pap{50}{16}

\end{enumerate}


\iffalse
\subsection{Iteratee, conduit and pipes}
Iteratee is a monadic representation of a push stream.
\begin{lstlisting}
enumPair :: Monad m => Iteratee el m a -> Iteratee el m b -> Iteratee el m (a,b)
enumPair i1 i2 = enum2 i1 i2 >>= runI2
\end{lstlisting}

\begin{lstlisting}
i1 :: Monad m => Iteratee el1 (Iteratee el2 m) (el1, el2)
i1 = do
     e1 <- head 
     e2 <- lift head
     return (e1,e2)
\end{lstlisting}


, Conduit and Pipes are monadic representations of pull streams.
They get around the linearity problem of reusing stream names, to some extent, by not allowing streams to be named.
This monadic representation allows a small set of primitives for reading from straight lines, emitting values etc, which can be composed together.
Implementing ``map'' for pull streams requires delving into the details of the stream representation, while implementing map for iteratee-style streams can be done with no knowledge of the underlying implementation.

This monadic interface allows the structure of the computation graph to depend on the values.
This expressiveness has a price: if the computation graph can change dynamically, we cannot statically fuse it.
Cite Arrows paper about how monads mix up the static and dynamic parts of a system, reducing the ability to optimise based on static information.
To work around these performance penalties, Conduit supports short-cut fusion for some operations.
Cerain operations are implemented using non-monadic pull streams, which can be converted to Conduit streams.
Rewrite rules then remove superfluous conversions between Conduit streams and pull streams.

A pedantic note on terminology: Conduit uses the term \emph{fusion} for connecting two stream transformers together, while in this thesis we use fusion to mean removing the overhead of connecting two streams together.
(So what --- is this even worth mentioning?)


\fi
